\chapter[Beyond Classical Physics]{Beyond Classical Physics: Photons
and Wave-Particle Duality}
\label{chapter:beyond_classical}

\section{Introduction to Quantum Mechanics}


In the early 1900s, a series of experiments and theoretical breakthroughs
dramatically changed our understanding of how the universe works,
including our conceptions of space and time, predictability versus
randomness, and the limits imposed on measurement at the atomic scale.
Not only did these developments --- referred to as {\it modern physics}
--- significantly overturn and revise the known laws of physics, they also
became the foundation for a complete understanding of the foundations of
chemistry (and therefore biology as well), and they led to a series of
significant applications that have already resulted in  an explosion of
modern technology, e.g., semiconductor physics and modern electronics;
computer technology; communications (including cell phones); numerous
medical diagnostic devices, surgical techniques and radiation therapies;
new and significantly enhanced forms of microscopy; ultra-precise
navigation devices; and nuclear power generation and weaponry.

And the technological applications of modern physics will continue well
into the future. In particular, there is a significant on-going research
effort into the development of nanotechnological devices that will likely
revolutionize the fields of medicine and engineering during the next
50 years.  Imagine, for instance, nanometer-scale devices that could be
programmed to search and destroy cancer cells or repair internal injuries
at the cellular scale within the human body. This idea is currently
speculative (as of the year 2021), but successful development of something
like this --- which would almost certainly require an understanding of
principles of modern physics --- could revolutionize cancer treatment in
a way that will cause society to look back at chemotherapy and radiation
treatment the way we currently look back at the use of leeches as
medical ``devices'' in the middle ages. Other potential future quantum
applications include quantum computing and encryption; replacements for
semiconductor switches based on new graphene materials or quantum optical
devices; or even molecular and biological electronic devices based on DNA.

In PHYS 211, you learned about one of the two pillars of modern physics,
Einstein's Theory of Relativity, which extends classical (Newtonian)
physics to systems that travel at speeds approaching the speed of light
and also sheds important light on the nature of matter and energy.
In this unit, you will learn about the other pillar of modern physics:
the theory of quantum mechanics.  Quantum effects are most noticeable on
a microscopic scale, yet quantum behavior has critical effects on
the macroscopic world that we see around us every day. As an example,
atoms wouldn't be possible without quantum principles, so all matter
around us would be dramatically different without this subatomic
behavior (actually, there wouldn't be any ``us'').  And we have developed
techniques that enable us to manipulate the subatomic world in ways that
have significant technological applications, including much of what forms
the basis of modern chemistry, chemical engineering, materials science,
and electronics.

What we've learned about the microscopic world is that it is {\it
  unlike anything we can picture!}  Things in the microscopic world,
such as electrons or protons or electric fields, can sometimes act as
particles and sometimes act as waves.  
Nothing in our macroscopic world is like this, so it is not easy
to build simple mental pictures for it.  To give you a sense of how
strange this is: when a ``particle'' such as an electron is ``acting
as a wave'' it does not have a definite position!  Its very existence
is spread out in some manner over a region of space.

So be it.  It's one of the greatest triumphs of science that we have
developed the experimental and theoretical tools to uncover the
behavior of nature even when our fundamental intuition can no longer
aid us.  The theory of quantum mechanics, as we shall see, is
necessarily somewhat abstract.  But it is very much a {\it physical}
theory, well-grounded in experimental evidence.


To be an educated,
scientifically aware citizen in the 21st century, it is essential to
have a grounding in the basic laws of quantum mechanics.

\section{Three Great Failures of Classical Physics}

What emerged in the early years of the 20$^\text{th}$ century were puzzles 
and mysteries that our well-developed and successful theories of Newtonian
mechanics, electricity and magnetism, and thermodynamics --- what we
now call {\it classical physics} --- were unable to solve.  Three of
these mysteries stood out above the others.
\begin{description}
\item[The ultraviolet catastrophe.] Combining classical electromagnetism with
  thermodynamics led to the (wrong) result that the thermal motion of
  matter --- the acceleration of jiggling charges --- would create an
  infinite amount of electromagnetic energy.  Oops!
\item[The stability of atoms.] Combining classical electromagnet\-ism (EM) with
  classical mechanics led to the result that electrons orbiting the
  nucleus of an atom would radiate away their energy and collapse into
  the nucleus after about $10^{-12}\units{s}$.  So atoms shouldn't be
  stable.  Oops again!
\item[Atomic spectral lines.] Experiments showed that when energy is
  pumped into atoms, say by heating a gas, the atoms then radiate EM
  waves back out, but only at certain distinct wavelengths.  Hydrogen
  atoms emit one set of wavelengths, helium atoms a different set, and
  so on.  Nothing in classical physics could come close to explaining
  this phenomenon.  Strike three!
\end{description}

The attempt to resolve these issues led to the revolutionary,
paradigm-changing, development of quantum mechanics.  In this unit we
will show how a new quantum theory was able to address successfully
these shortcomings of the classical theory.

The story begins with Planck and Einstein, who introduced the notion
that EM waves, for example light waves, can also act as particles.
This is a phenomenon called \textit{wave-particle duality}: light is
neither just a wave nor just particles, but has aspects of both.  In
this chapter we will look closer at the ultraviolet catastrophe ---
the incompatibility of thermodynamics and electromagnetism --- and
show that the particle aspect of light resolves the problem.  Then
we will explore the implications for the interaction between
light and matter (explaining, for example, why you should wear
sunscreen when you are outside for long periods on a sunny day).  
Finally, we conclude with de~Broglie's stunning and
ultimately correct hypothesis that not just light but all matter in
the universe exhibits wave-particle duality.


\section{Ultraviolet Catastrophe}

Electromagnetism and thermodynamics were well-developed and successful
theories by the end of the 19$^\text{th}$ century.  The theory of
electromagnetism (Maxwell's equations) demonstrated the 
unification of electricity and magnetism, and led to the realization
that light is an electromagnetic wave.  This, in turn, led to the
technology of generating and receiving radio waves, which was the
second great step in the information technology revolution.\footnote{The
printing press was the first.}  Similarly, our theories of
thermodynamics explained the states of matter and provided an 
understanding of the engines that powered the industrial revolution.

There was just one big problem: classical E\&M and thermodynamics
are incompatible.  Here is the basic issue.  Picture some substance
at a temperature $T$.  The particles in that substance are vibrating
around with thermal motion.  This amounts to accelerating charges, and
accelerating charges emit EM waves.  Conclusion: we expect matter at a
temperature $T$ to be radiating away some of its energy into electric
and magnetic fields.

But the energy exchange goes both ways.  These EM fields exert forces
on the charged particles, giving them back some energy.  So all
together we see that energy sloshes between the moving particles and
the EM fields, much like it sloshes from particle to particle within
the substance.  Then we should expect the EM fields, like the particle
motion, to have some thermal equilibrium values determined by the
temperature $T$.

So far, so good.  Let us now try to calculate what the thermal
equilibrium EM fields should be for the simplest case we can
construct.  Imagine a cavity bounded by a pair of walls separated by a
distance $L$ (i.e., a one-dimensional box), and to keep things simple 
we will only have one spatial
dimension.  The situation is illustrated in Fig.~\ref{fig:cavity}. The
charges in the walls are in thermal equilibrium at temperature $T$, so
they are moving and creating electric and magnetic fields.  What form
can these EM fields take?  Basically, we get all the possible standing
wave modes, just like waves on a string or sound waves in a tube.  As
we found in the waves unit, the longest wavelength is $\lambda=2L$,
the second longest $\lambda=L$, the third longest is $\lambda=2L/3$,
and so on.  The actual EM fields contain all those modes and can be
expressed as some superposition of these waves.

\begin{figure}
\begin{center}
\includegraphics[width=4.5in]{beyond_classical/cavity}
\caption{EM waves in cavity of length $L$.  These EM waves are
created by the thermal motion of the charges in the walls.}
\label{fig:cavity}
\end{center}
\end{figure}

Now let's try to calculate the amplitude of these waves.  First,
recall that the energy of a wave is proportional to the amplitude
squared.  For a superposition of wave modes the energy turns out to be
the superposition of the energies of each individual wave:
\begin{equation}
E = \alpha A_1^2 + \alpha A_2^2 + \alpha A_3^2 + \dots
\end{equation}
where $\alpha$ is some constant and $A_n$ is the amplitude of the
$n^\text{th}$ longest wavelength mode of the EM wave.  Now comes the
thermodynamics: the equipartition theorem says that in thermal
equilibrium any quadratic term in the energy has an average value of
$\textstyle\frac{1}{2}k_BT$.  Therefore we can conclude
\begin{equation}
\langle \alpha A_1^2\rangle = \langle \alpha A_2^2\rangle = 
\langle \alpha A_3^2\rangle = \dots = \frac{1}{2}k_BT.
\label{eq:EM_wave_equipartition}
\end{equation}
The equipartition theorem implies that each of those modes has
to have the same average value, and they are all related to the
temperature $T$.  It's a powerful theorem!

And now we are ready to compute the total energy in the EM fields.
All we need to do is count the number of modes in the cavity ($n=1$,
$2$, $3$, \dots) and then multiply by $\textstyle\frac{1}{2}k_BT$.  But here is
the problem: \textit{There are an infinite number of modes!}

We can keep drawing waves with shorter and shorter wavelength, and we
never run out of modes.  Each new one we draw brings another
$\textstyle\frac{1}{2}k_BT$ to the energy.  So we are led to conclude that there
is an infinite amount of energy in the EM fields.  This is clearly
wrong --- thankfully, or we would all be blasted by the infinite
radiation all around us.  And now we can see why this is called the
ultraviolet catastrophe: it is the infinite piling up of shorter and
shorter wavelengths, or higher and higher frequencies, that is where
something in the theories are breaking down.

So where did it go wrong?


\section{Photons}

The resolution of the ultraviolet catastrophe, reached in stages
by Planck and Einstein, is that Maxwell's theory of electricity
and magnetism is incomplete.  In 1905\footnote{1905 was Einstein's
``Miracle Year'' --- the year that he published his first paper on
relativity, a paper that explained molecular diffusion, and a paper
introducing the concept of a photon.} Einstein introduced the concept
of a photon: a ``particle'' of light whose energy is related to the EM
wave frequency via 
\begin{equation} 
E_\text{ph} = hf 
\end{equation}
where 
\begin{equation} h = 6.63\times 10^{-34}\units{J$\cdot$s} 
                   = 4.14\times 10^{-15}\units{eV$\cdot$s} 
\end{equation} 
is known as
Planck's constant.  Einstein's claim was that an EM wave of frequency $f$
can be viewed as a gas of photons, each with this energy.  Photons are
not part of Maxwell's equations; they are in some sense a supplementary
property of EM waves.\footnote{The complete theory of
  electricity and magnetism, including the full reconciliation with
  quantum mechanics and relativity, is called \textit{quantum
    electrodynamics} and wasn't fully developed until the 1950s.  We
  will get a flavor of this theory in Unit 4.}

\begin{example}{Number of photons.}
  An EM wave of wavelength $\lambda=420\units{nm}$ has a total energy
  of $1.80\units{J}$.  How many photons are in this wave?

  \solution Each photon is contributing an energy
  $E_\text{ph} = hf$.  Using $f = c/\lambda$ we can calculate the
  energy per photon,  
\begin{eqnarray}
E_\text{ph} &=& \frac{hc}{\lambda} \nonumber \\
            &=& \frac{(6.63\times 10^{-34}\units{J$\cdot$s}) 
                (3.00\times 10^8\units{m/s})}
                {420\times 10^{-9}\units{m}} \nonumber \\
            &=& 4.74\times 10^{-19}\units{J}.
\end{eqnarray}
  The total energy $E$ is given by $N E_\text{ph}$, where $N$ is the
  number of photons, so
\begin{equation}
N = \frac{E}{E_\text{ph}} = \frac{1.80\units{J}}{4.74\times
  10^{-19}\units{J}} = 3.80\times 10^{18}.
\end{equation}
Evidently, EM waves with macroscopic energies contain a lot of
photons!
\end{example}

The idea of light being composed of particle-like photons is tremendously
important with numerous implications throughout all of science and
engineering.  The photon picture also resolves the problem of infinite
energy in a cavity.  Imagine sorting all the EM wave modes in the cavity
from lowest frequency to highest frequency.  Since $E_\text{ph}=hf$,
the photons in the lowest frequency mode have low energy, while the
photons in the high frequency modes have higher energy.  But the total
energy of each mode must be $\textstyle{\frac{1}{2}}k_B T$, which means each EM wave
mode has exactly the same energy.

How can higher energy photons add up to the same energy?  There must
be fewer of them.  So as we go to higher and higher frequency modes,
we must have fewer and fewer photons in each mode.  Eventually we will
reach a high frequency mode with only one photon, and that's the end
of the line.  There can be no higher frequency modes.  And so the
ultraviolet catastrophe is averted.

In Problems \ref{prob:cavity_freq} and \ref{prob:largest_mode}
at the end of this chapter, you will do this quantitatively. You will find
that the mode number $j_\text{max}$ where the number of photons drops to below
$1$ in a one-dimensional cavity is
\begin{equation}
j_\text{max} =  \frac{L k_B T}{h c}.
\label{eq:jmax}
\end{equation}
For modes with larger $j$, there is not enough thermal energy to produce
even a single photon. 
Given this result, we can find the total energy in the cavity.  We have modes $j=1$,
$2$, \dots, $j_\text{max}$, each contributing an energy
$\textstyle{\frac{1}{2}}k_BT$.  That adds up to
\begin{equation}
E =  j_\text{max}\times \Bigl(\frac{1}{2}k_BT\Bigr) = \frac{L (k_BT)^2}{2hc}.
\label{eq:EM_wave_thermal_energy_1D}
\end{equation}
This answer is a much nicer result for the thermal energy of the EM
fields, because it's not infinite!

Let's reflect a moment on what we have just done.  We argued that EM
waves, which in classical physics can have any continuous value of
energy, instead come in energy chunks.  When the number of these
photon chunks is large, some $10^{18}$ or so, then we can ignore the
chunkiness and treat the EM waves classically.  Then the equipartition
holds and we can use
Eq.~(\ref{eq:EM_wave_equipartition}). 

But for really high frequency (UV) waves, the photons have so much
energy that there are only a few of them in a mode.  At this point,
the classical physics result Eq.~(\ref{eq:EM_wave_equipartition}) 
no longer holds.  Essentially, there is simply not enough thermal
energy available to create even a single photon for these modes.
So we counted an energy of $\textstyle{\frac{1}{2}}k_BT$ for each mode until
we hit this wall, and then we stopped counting.  The result is 
Eq.~(\ref{eq:EM_wave_thermal_energy_1D}).

Note that this result only applies to a one-dimensional system. 
Real materials, of course, are three dimensional.  We won't go through
the full derivation for a three-dimensional cavity.  That would require
working with the thermodynamics of a photon gas, something
that we do in our PHYS 317 Thermodynamics course.  

\section{Photons Interacting with Matter}

In the previous section we argued why the photon picture resolved the
dilemma of the UV catastrophe,  but many physicists at the time that
Einstein proposed photons were skeptical of the idea.  But contained
in Einstein's brilliant suggestion was the potential for a quantitative
test of the idea, through a phenomenon called the photoelectric effect.  
This is the work for which he was awarded his only Nobel 
Prize.\footnote{Einstein
  should have received five Nobel Prizes, for special relativity, the
  photoelectric effect, Brownian motion (which demonstrated the
  existence of atoms), general relativity, and Bose-Einstein
  condensation.}

\begin{figure}
\begin{center}
\includegraphics[width=2.8in]{beyond_classical/photoelectric_sketch}
\caption{A sketch of the photoelectric effect: light shines on a metal,
and electrons are released from the surface.}
\label{fig:photoelectric_sketch}
\end{center}
\end{figure}

\subsection{Photoelectric Effect}

Metals are materials in which electrons are free to move around ---
this is why metals are good conductors, why they are shiny, etc. But
the electrons cannot just freely hop off the surface.  They are bound
to remain inside the material by a binding energy $U_\text{bind}$.
It was discovered at the close of the 19$^\text{th}$ century that some
electrons can be made to fly off the surface of a metal by shining light
on it, as sketched in Fig.~\ref{fig:photoelectric_sketch}.  This should
seem reasonable: the EM wave gives some extra energy to the electrons,
allowing them to break free of their binding to the metal.  The electron
``pays off'' its binding energy debt and leaves with the remaining energy
in the form of kinetic energy.

Let's focus on how this effect should depend on the intensity and the
frequency of the light.  With classical EM waves, increasing the
amplitude of the wave should cause a stronger force on the electrons
and give them more energy.  This would suggest that increasing the
intensity of light should increase the energy given to the electrons
and they should come off with a higher final speed.  Also, the
classical EM wave picture would predict that it should be possible to
eject electrons from the metal for any frequency of light, as long as
the intensity is high enough.

By 1905 there were qualitative indications that this was not what
was happening in experiments, and Einstein's photon hypothesis
provided a quantitative theory that could be tested experimentally.
In Einstein's theory, which was quantitatively verified by experiments
in 1916, the intensity of the light has {\it no effect} on the kinetic
energy of the emitted electrons --- it is the frequency of the light
that is related to the kinetic energy.  Shining low frequency light on
the material results in no electrons being emitted, regardless of the
intensity, and when electrons are emitted, the kinetic energy increases
linearly with the frequency of the light. The experimental results are
illustrated graphically in the Fig.~\ref{fig:photoelectric_data}.

\begin{figure}
\begin{center}
\includegraphics[width=5in]{beyond_classical/photoelectric_data}
\caption{Left: the maximum kinetic energy of the emitted electrons is
  independent of the intensity of light.  Also, electrons are not
  emitted regardless of the intensity of light if the frequency $f$ is
  below some cutoff value $f_c$.  Right: the maximum kinetic energy of
  the emitted electrons does depend on frequency, and the slope of the
  graph is equal to Planck's constant.}
\label{fig:photoelectric_data}
\end{center}
\end{figure}

Einstein's theory of the photoelectric effect is based on the following
key assumption: 

\boxittext{ In the microscopic world of atoms and subatomic particles,
  light interacts with matter in the form of \textit{single photons}.  }

This means light of frequency $f$ can only give a single photon's
energy to an electron.  That photon, with energy $E_\text{ph} =hf$,
might or might not have enough energy to free the electron from the
metal.  If the photon energy is below the binding energy ($E_\text{ph} < 
U_\text{bind}$), then no
electrons will escape.  Since $E_\text{ph} = hf$, this sets the
cutoff frequency $f_c$ when $E_\text{ph} = hf_c = U_\text{bind}$.  If the
photon energy is greater than the binding energy holding the electron
to the metal ($E_\text{ph} > U_\text{bind}$), then the electron might escape (it can always squander
the energy and head off in the wrong direction, or bump into an
impurity in the metal).  But of those electrons that do escape, there
will be an \textit{upper limit} to their kinetic energy given by
\begin{eqnarray}
K_\text{max} &=& E_\text{ph} - U_\text{bind} \nonumber \\
             &=& hf - U_\text{bind}.
\end{eqnarray}
That upper limit is obtained when the electron uses the energy
absorbed from the photon optimally.  Note that the maximum electron
kinetic energy depends only on the frequency of the light and not on
its intensity.  This was Einstein's bold prediction, and it was
confirmed experimentally by Millikan, who measured $K_\text{max}$ for
various frequencies and found data of the form sketched on the right
in Fig.~\ref{fig:photoelectric_data}.

Thus we see that the photon picture explains the experimental results
for the photoelectric effect perfectly.  It also provides a quantitative
measurement of Planck's constant $h$ from the slope of the $K_\text{max}$
versus $f$ plot.

\begin{example}{Electrons emitted from copper.}
Copper has a binding energy of $4.7\units{eV}$.  For light of {\bf
  (a)} $200\units{nm}$ and {\bf (b)} $400\units{nm}$ shining on a
piece of copper, determine whether electrons are emitted and, in the
case they are emitted, calculate their maximum kinetic energy.

\solution First we calculate the photon energy.  A handy trick for
getting photon energies in electron volts is the
following:\footnote{The relation $hc=1240\units{eV$\cdot$nm}$ can save
  a lot of calculator typing for problems with photon energies in eV and
  wavelengths in nm.}
%
\begin{equation}
  E_\text{ph} = hf = \frac{hc}{\lambda}
  =\frac{1240\units{eV$\cdot$nm}}{\lambda}.
\end{equation}
%
For (a), $\lambda = 200\units{nm}$, so the photon energy is
%
\begin{equation}
  E_\text{ph}= \frac{1240\units{eV$\cdot$nm}}{200\units{nm}} =
  6.2\units{eV}.
\end{equation}

When the photon is absorbed by the electron, the electron gains enough
energy to pay off its binding energy debt and escape with
\begin{equation}
  K_\text{max} = E_\text{ph} - U_\text{bind} = 6.2 - 4.7 = 
\boxed{1.5\units{eV}}.
\end{equation}
For (b) we have a photon energy of $1240/400= 3.1\units{eV}$.  This
is less than the binding energy, so no electrons will be emitted.
\end{example}

\subsection{Ionization}

Einstein's basic idea that photons are the energy chunks in the
interaction of light with matter is able to explain more phenomena
than just the photoelectric effect.  For example, an electron in an
atomic orbital is held in the atom by some binding energy.  For the
case of hydrogen --- which is just a single electron orbiting a single
proton --- the electron is usually in the lowest energy state
possible, called the ground state.  In the hydrogen ground state the
binding energy for the electron is $13.6\units{eV}$.  When lower
frequency radiation (EM waves) shines on a gas of hydrogen, then none
of the electrons are freed from their atomic orbitals.  Radiation of
higher frequency is able to liberate electrons from their host
nucleus, creating \textit{ionized hydrogen}.  This higher frequency
radiation is called, rather sensibly, ionizing radiation.

This idea extends beyond simple hydrogen.  EM waves frequently induce
chemical reactions, from a photographic plate to photosynthesis to
mutation of DNA.  That is, the energy of the absorbed photon is being
used to break and possibly re-arrange some chemical bonds. In the case of DNA mutation, the
relevant energy scale of the chemical bonds is in the 0.1 to $10\units{eV}$
range; EM waves with photon energies comparable to or greater than 
these binding energies can cause mutations. {\bf This is a known mechanism
for explaining why exposure to EM radiation with certain frequencies can
cause cancer in humans.} The photon nature of light is also critical
for understanding various photo-detectors, including the sensors
in the digital camera in your cell phone. We will explore these ideas in the
assigned problems and in subsequent chapters.

Finally, we conclude this section with an important point: light had
previously been considered to be exclusively a wave phenomenon, but
with photons it has been shown to have particle characteristics.  But
it's not an either/or situation.  It is not the case that the wave
description is wrong; after all, you have seen the interference
phenomena yourself in lab.  So we are led to acknowledge that light
can be both wave-like and particle-like.  What it does in a particular
experiment depends on what the experiment is measuring.  This is very
strange! 

\section{Wave-Particle Duality}

Planck and Einstein made the first big steps towards quantum mechanics
by introducing photons and showing the role they play when light
interacts with matter.  The next breakthrough was made by de~Broglie
(pronounced de-BROY), who proposed what is now called {\it
  wave-particle duality.}

Basically, de~Broglie noticed how strange the wave and photon
character of light is and pondered whether this was perhaps not
limited to light, but rather was a new feature of nature.  Since
light, which had been considered a wave, could turn out to have
particle properties, perhaps things which are considered particles,
like protons and electrons, could turn out to have wave properties.
Perhaps {\it everything} in the microscopic world exhibits
wave-particle duality.

\subsection{Electron Interference}

What would this mean for an electron or a proton to have wave-like
properties?  The clear signature of waves is interference, so we could
try to make electrons interfere like waves.  Imagine an experiment
where a beam of electrons is sent towards a double-slit apparatus,
much like you did with a beam of light in lab.  If the electrons are
strictly particles with no wave character, we would expect to find two
spots on the screen where the electrons are striking: one bright spot
is the electrons that passed through the left slit and the other is
the electrons that passed through the right slit.  However, if the 
electrons are acting as waves, we should expect to see a full
interference pattern on the screen, with a sequence of bright and
dark spots.


\begin{figure}
\begin{center}
\includegraphics[width=4in]{beyond_classical/simulated_electron_double_slit}
\caption{Electrons are sent through a double slit apparatus and then
  strike a screen.  The dots represent where the electrons hit the
  screen.  As the number of electron detections grows, the
  interference pattern becomes clear. (Simulated data.)}
\label{fig:electron_double_slit}
\end{center}
\end{figure}

These experiments have been done.  We show here a simulation of the data in
Fig.~\ref{fig:electron_double_slit}.  This figure demonstrates that
once enough electrons have reached the screen, an interference pattern
develops.  Electrons can act as waves!

There is an extremely peculiar aspect to this experiment.  First, note
that the bright spots in the interference pattern are simply the
regions where an electron hit more often.  Equivalently, they are
regions where an electron has a higher probability to hit.  The
electrons go through the double slit apparatus one at a time, and
somehow they know where the higher and lower probability regions are.
But the interference pattern is a property of both slits.  For
example, if the slit spacing is changed, the distance between
interference bright spots is changed.  This implies that somehow each
single electron ``experiences'' both slits, since it passes through
them knowing the probabilities of where to hit on the screen.  This is
very strange!

But very real.  In fact, the wave-like property of electrons is the
basis of a handy tool called the \textit{electron
  microscope}.\footnote{Bucknell has one.  Some of you may have used
  it.}  This microscope can probe length scales significantly shorter
than visible light to study, for example, cell organelles.

\subsection{The de~Broglie Relation}

de~Broglie went further than just proposing this wave-particle
duality; he made a specific prediction for what the wavelengths
should be for things we normally think of as particles.  The 
de~Broglie relation,
\begin{equation}
\lambda = \frac{h}{p}
\label{eq:deBroglie_first}
\end{equation}
says that the wavelength of a ``particle'' depends on its momentum and
on Planck's constant.

Let us check that this relation works for photons: recall from the
relativity unit of PHYS 211 that massless photons have an energy
$E_\text{ph} = c|p_\text{ph}|$.  The de~Broglie relation says that the
photon momentum and wavelength should be related by $p=h/\lambda$, so
all together this gives
\begin{equation}
E_\text{ph} = c |p_\text{ph}| = c\frac{h}{\lambda} = hf.
\end{equation}
Indeed, for photons Eq.~(\ref{eq:deBroglie_first}) is equivalent to
Einstein's photon energy relation.

So de~Broglie's proposal was that $\lambda=h/p$ holds for all
particles, not just for photons.  Let's see how this works out for an
electron microscope.

\begin{example}{Electron microscope.}
You wish to use an electron microscope to resolve features of a cell
organelle that are on the scale of $1\units{nm}$.  To do this, you
``shine'' a beam of electrons on the sample.  What is the minimum
speed that these electrons could have?  \solution Recall that to
resolve features on a certain scale, we need a wavelength at least
that small, so we will need $\lambda \leq 1\units{nm}$.

According to the de~Broglie relation, this implies
\begin{equation}
  p = \frac{h}{\lambda} \geq \frac{h}{1\units{nm}}  
\end{equation}
or
\begin{equation}
  p \geq \frac{6.63\times 10^{-34}\units{J$\cdot$s}}{10^{-9}\units{m}}
  = 6.6\times 10^{-25}\units{kg$\cdot$m/s}.
\end{equation}
Now that we have the minimum for the momentum, we can find the 
minimum speed simply by $p=mv$ (we are neglecting relativistic
effects):
\begin{equation}
  v = \frac{p}{m} \geq \frac{6.6\times 10^{-25}\units{kg$\cdot$m/s}}
  {9.11\times 10^{-31}\units{kg}} = \boxed{7.3\times 10^5\units{m/s}.}
\end{equation}
\end{example}

So, we have seen that de~Broglie argued (successfully) that particles
act like waves.  But {\it waves of what}?  We will discuss this further
in the next chapter.

\vfill

\newpage

\section*{Problems}
\markright{PROBLEMS}


\begin{problem}
Calculate: 
\begin{enumerate}
\item the energy (in eV) of a $500\units{nm}$ wavelength photon,
and 
\item the non-relativistic kinetic energy for a $500\units{nm}$
electron.
\end{enumerate}
\label{prob:photonenergy}
\end{problem}

\begin{problem}
A certain electron has the same wavelength as orange light, $\lambda \simeq
600\units{nm}$.   Calculate the speed of this electron.
\end{problem}

\begin{problem}
The electron binding energy for a particular metal is
$2.25\units{eV}$.  Calculate the cutoff frequency for the
photoelectric effect in this metal.
\label{prob:cutoff_freq}
\end{problem}

\begin{problem}
\begin{enumerate}

\item Let's say that you want to probe the structure of a bacteriophage
T4 virus.\footnote{These viruses --- which attack common bacteria ---
are amazingly cool. They are like little lunar landers that land on the
surface of a bacterium, after which they inject DNA into the bacterium,
which then produces many more of the viruses before exploding.  See,
e.g., Kanamuru et al., Nature {\bf 415}, p. 553 (2002).} One way to probe
the virus is to look at it with electromagnetic waves.  If one wants
to ``see'' the T4 structure with a resolution of about 
$1\units{nm}$, what must be the wavelength of the EM wave used?
What is the energy of a single photon with this wavelength?

\item Another way to probe the T4 is with an electron microscope.
  What must the wavelength and energy of an electron used by the 
  microscope if it is to resolve the T4 virus down to the same resolution of
  $10^{-9}\units{m}$?
\end{enumerate}
\label{prob:T4virus}
\end{problem}

\begin{problem}
Use the de~Broglie relation to calculate the momentum of an X-ray
photon of frequency $f=1.0\times 10^{18}\units{Hz}$.
\label{prob:deBmomentum}
\end{problem}

\begin{problem}
Photons of frequency $6.0\times 10^{14}\units{Hz}$ are directed
towards a metal.  As a result, electrons are ejected with kinetic
energies up to $1.4\units{eV}$.  Determine the binding energy for this
metal.
\label{prob:BindingEnergy}
\end{problem}

\begin{problem} % [A52]
A few years ago there was a flurry of attention given to the potential
hazards of electromagnetic fields from overhead power lines.  The
concern is that the alternating current (AC) in these power lines was
emitting radiation that could cause cancer.
\begin{enumerate}
\item The AC current in power lines alternates with a frequency of
  $60\units{Hz}$.  Use this to determine the energy of photons emitted
  from the power lines (express your answer in eV).
\item The weakest molecular bonds have binding energies around
  $0.1\units{eV}$.  Use this explain why the scientific community is
  highly skeptical of the claims of cancer dangers.
\end{enumerate}
\label{prob:EMFhazards}
\end{problem}

\begin{problem} %[A55]
When a particular metal is illuminated with infrared radiation of
wavelength $700\units{nm}$, electrons are emitted with kinetic
energies that range up to $0.25\units{eV}$.  Calculate the largest
kinetic energy for ejected electrons if the same surface is
illuminated with light of wavelength $400\units{nm}$.
\end{problem}

\begin{problem}
A one-dimensional cavity of length $L$ is filled with electromagnetic
standing waves.  Show that the frequency of the $j^\text{th}$ longest
wavelength mode is given by $f_j = cj/(2L)$.  
\label{prob:cavity_freq}
\end{problem}


\begin{problem}
\begin{enumerate}
\item Given your results from Problem~\ref{prob:cavity_freq},
calculate the energy of a single photon for the $j^\text{th}$ longest 
wavelength mode in a one-dimensional (1D) cavity with length $L$.
\item Considering that the Equipartition Theorem predicts an average
total energy of $\textstyle{\frac{1}{2}}k_BT$ for each mode, calculate
the number of photons that you would expect to find (on average)
for the $j^\text{th}$ longest wavelength mode in a 1D cavity with length $L$.
\item Based on your answer to (b), what would be the largest mode
number $j$ for which you would expect to find (on average) one or more
photons in the cavity?
\item In one or two sentences, explain why the photon nature of light
resolves the problem of the UV catastrophe (i.e., the prediction of
an infinite total energy). 
\end{enumerate}
\label{prob:largest_mode}
\end{problem}

\begin{problem}
As discussed in the reading, the Equipartition Theorem says that, classically,
each electromagnetic wave mode should have an average energy of
$\textstyle{\frac{1}{2}}k_BT$. For simplicity, we'll assume a one-dimensional cavity
with a length of $2.0\units{cm}$.
\begin{enumerate}
\item Calculate the value of $\textstyle{\frac{1}{2}}k_BT$ at
room temperature of $22^{\circ}\units{C}$.
\item Calculate the wavelength and frequency of the lowest frequency
normal mode electromagnetic wave for this cavity.
\item Calculate the energy of one photon of light with this
wavelength.
\item Calculate the number of photons that you would expect (on the
average) for the lowest frequency mode in this cavity.
\item How many photons would you expect (on average) for the
second lowest frequency mode in this cavity?
\item How many photons would you expect for the $10^\text{th}$ lowest
frequency mode in this cavity? The $100^\text{th}$ lowest mode? The
$1000^\text{th}$ mode? The $10000^\text{th}$ lowest mode?
\item Do you think that the equipartition theorem holds for all of your
answers in part (f)? Why or why not?
\end{enumerate}
\end{problem}

\begin{problem}
The electromagnetic field in a one-dimensional cavity is in thermal
equilibrium, and the longest wavelength mode contains $4500$ photons.
\begin{enumerate}
\item Calculate the number of photons in the second-longest wavelength
mode.
\item Calculate the number of photons in the third-longest wavelength
mode.
\item Calculate $j_\text{max}$, the largest mode number.
\end{enumerate}
\end{problem}


\begin{problem}
An X-ray photon of wavelength $60\units{nm}$ ionizes a hydrogen atom
with an electron initially in its ground state (with energy $-13.6\units{eV}$).
Calculate the kinetic energy of the resulting free electron.
\label{prob:IonizeHAtom}
\end{problem}

\begin{problem}
Wave-particle duality means that all the fundamental building blocks
in the quantum microscopic world have both wave and particle
properties.  This is demonstrated in the electron double slit
experiment shown in Fig.~\ref{fig:electron_double_slit}.

\begin{enumerate}
\item Describe an aspect of the experiment that involves electrons
  acting as particles.
\item Describe an aspect of the experiment that involves the same
electrons acting as waves.
\end{enumerate}
\end{problem}

\begin{problem}
A He-Ne laser emits light of wavelength $633\units{nm}$.  Calculate
how many photons per second are emitted by a $5\units{mW}$ He-Ne
laser.
\end{problem}

