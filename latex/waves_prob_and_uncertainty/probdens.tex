\section{Probability Density}
\label{sec:ProbDensity}

Now that we have a method for obtaining the wavefunction $\psi(x)$ of
a particle, let us return to the discussion of how to determine the
probability density.  Simply put, the probability density $P(x)$ for a
particle whose wavefunction is $\psi(x)$ is defined to be the square 
of the magnitude of the wavefunction $|\psi(x)|$.  The concept of
probability density can be tricky --- it's not the same as
probability!  Probability density is really a {\em probability per
  unit length}. So to get probability from $|\psi(x)|^2$, you must
multiply by a length interval. Specifically
\begin{equation}
\label{eq:probDensity}
P(x) dx = |\psi(x)|^2\, dx = \left[ \begin{array}{l}
         \mbox{Probability of finding the particle in an} \\
         \mbox{interval of length $dx$ centered at position $x$}
                    \end{array} \right] .
\end{equation}

While a general wavefunction may be very complicated, the prescription
for its interpretation is always the same: from $\psi(x)$, calculate
$|\psi(x)|^2$, its magnitude squared.  Then the probability density is
given by $P(x)=|\psi(x)|^2$.

One of the complications of a general wavefunction is that it may be
complex-valued, involving factors of the imaginary number $ i = \sqrt{-1}$.  
When this occurs $|\psi(x)|^2$ does not just mean ``square the wavefunction''
to get the magnitude squared.  Rather you should follow this three-step
process:

\begin{enumerate}

\item[(1)] Write the {\em complex conjugate} of the wavefunction, $\psi^*$,
  by replacing every occurrence of ``$i$'' with ``$-i$'' in the
  wavefunction.  For example if $\psi(x) = 2 e^{i3x} - 5 i x^2$, then
  $\psi^*(x) = 2 e^{-i3x} + 5 i x^2$.

\item[(2)] Multiply $\psi(x)$ times $\psi^*(x)$.

\item[(3)] Replace any occurrences of ``$i^2$'' with ``$-1$'' and
  simplify. Your result should be real (no left-over ``$i$''s) and
  non-negative for all values of $x$, as required by a probability
  density.

\end{enumerate}

\begin{example}{Calculating probability density}
\label{exam:CalcProbDensity}
A particle in a region of space is represented by the wavefunction
\begin{equation}
\psi(x) = \sqrt{5} e^{ikx} \sin(2x) e^{-x} \nonumber
\end{equation}

\noindent where $k$ is a real constant.  Calculate the probability
density for this particle and determine at what position the particle
is most likely to be found.

{\bf Solution:} According to the prescription given above, the
probability density for this particle is calculated to be
\begin{eqnarray}
\left| \psi(x) \right| ^2 & = & \psi(x) \cdot \psi^*(x) \nonumber \\
 & = & \left( \sqrt{5} e^{ikx} \sin(2x) e^{-x} \right) \left( \sqrt{5} e^{-ikx} \sin(2x) e^{-x} \right) \nonumber \\
 & = & 5 \sin^2(2x) e^{-2x} \nonumber
\end{eqnarray}

\noindent where we have used the fact that
\begin{eqnarray}
(e^{ikx}) (e^{ikx})^* &=& (e^{ikx}) (e^{-ikx}) \nonumber \\
&=& e^{(ikx-ikx)} = e^0 = 1. \nonumber
\end{eqnarray}

\begin{figure}[!hb]
\begin{center}
\includegraphics[width=3in]{wavefunctions/fig42.eps}
\end{center}
\caption{Probability density for the wavefunction of Example~\ref{exam:CalcProbDensity}.}
\label{fig:prob_density_example}
\end{figure}

Fig.~\ref{fig:prob_density_example} shows a plot of $P(x)$ versus
$x$. Using the interpretation for the probability density as given in
Eq.~(\ref{eq:probDensity}), we see from
Fig.~\ref{fig:prob_density_example} that $P(x)$ has a maximum near the
position $x \approx 0.55$ and therefore the probability of finding the
particle is the greatest at this position. So, if a measurement of the
position of the particle is made, it will most likely yield the value
$x = 0.55$.
\end{example}

In most cases, we are not interested in knowing the probability of
finding the particle at any precise position, but rather the
probability of being located within a finite range of positions.
Fig.~\ref{fig:prob_density_integral} is the graph of a probability
density for a particle in some quantum state.  The quantity $P(x_o)
dx$ represents the probability for finding the particle within a
narrow interval $dx$ about position $x_o$, shown as the shaded region
in the figure.  If we would like to know the probability of finding
the particle anywhere between the two positions $x_a$ and $x_b$, then
we would add probabilities such as $P(x_o) dx$ for all positions $x_o$
from $x_o = x_a$ to $x_o = x_b$.  Mathematically, this is the same as saying
we need to {\em integrate} the probability density:

%\begin{figure}[!hb]
\begin{figure}[!h]
\begin{center}
\includegraphics[width=4in]{wavefunctions/prob_density_integral.eps}
\end{center}
\caption{Determining the probability over a range of positions.}
\label{fig:prob_density_integral}
\end{figure}

\begin{equation}
\label{eq:IntegralprobDensity}
\left( \begin{array}{l}
         \mbox{Probability of finding the particle} \\
         \mbox{in the region $[x_a, x_b]$}
                    \end{array} \right) = \int_{x_a}^{x_b} P(x) dx
                                        = \int_{x_a}^{x_b} |\psi(x)|^2 dx .
\end{equation}
Eq.~(\ref{eq:IntegralprobDensity}) brings up an important
property of wavefunctions and probability densities.  What if the
region we are interested in is the entire $x$-axis, that is, 
$[x_a, x_b] = (-\infty, +\infty)$?  
Another way of saying this is, `What is
the probability of finding the particle anywhere?'  Certainly the
answer to this question is $100\%$ !  This implies that the
wavefunction must be given such that
\begin{equation}
\label{eq:normalization}
\int_{-\infty}^{+\infty} |\psi(x)|^2 dx = 1 .
\end{equation}
When a wavefunction satisfies this requirement, it is said to be {\em
  normalized}.  Normalization is usually accomplished by multiplying
the wavefunction by a suitable constant factor, as we will see in the
following examples.

\begin{example}{Probability of finding a particle I.}
Fig.~\ref{fig:example_wavefunction} shows the wavefunction for a
certain particle, where $A$ is a positive constant. (a) Determine a
value for the constant $A$ such that the wavefunction is properly
normalized. (b) What is the probability of finding the particle in the
region between $x = 2\units{nm}$ and $x = 4\units{nm}$?

\begin{figure}[!hb]
\begin{center}
\includegraphics[width=3in]{wavefunctions/prob_density_example.eps}
\end{center}
\caption{Example wavefunction.}
\label{fig:example_wavefunction}
\end{figure}

{\bf Solution: (a)} To determine probabilities we must first determine
$|\psi(x)|^2$ for the given wavefunction. We do this by computing the
value of $|\psi(x)|^2$ at every position on the graph which results in
the graph of Fig.~\ref{fig:example_prob_density}.

\begin{figure}[!hb]
\begin{center}
\includegraphics[width=3in]{wavefunctions/prob_density_example_a.eps}
\end{center}
\caption{Probability density for wavefunction.}
\label{fig:example_prob_density}
\end{figure}
The total probability of finding the particle anywhere is the total
area under the curve of Fig.~\ref{fig:example_prob_density}:
\begin{eqnarray} 
\int_{-\infty}^{+\infty} |\psi(x)|^2 dx &=& A^2 (2 - 1) + \left( - \frac{A}{2} \right)^2 (4 - 2) \nonumber \\
&=& A^2 + \frac{A^2}{2} = \frac{3}{2} A^2 .
\end{eqnarray}

For the wavefunction to be properly normalized, this area must equal
one.  Therefore, the value for the constant $A$ must be
\begin{equation}
\frac{3}{2} A^2 = 1 \hspace{0.3in} \rightarrow \hspace{0.3in} A = \sqrt{\frac{2}{3}}.
\end{equation}
{\bf (b)} Now that we know the value for the constant $A$ that
properly normalizes the wavefunction, the probability of finding the
particle in the region between $x = 2\units{nm}$ and $x = 4\units{nm}$ is just the area
under the curve of the probability density between $x = 2\units{nm}$ and $x =
4\units{nm}$:
\begin{equation}
\text{Prob}([2, 4]) = \frac{A^2}{4} (4 - 2) = \frac{A^2}{2} = \frac{1}{3}. 
\end{equation}
\end{example}

Now that we understand the concept of normalization, we know how
to assign a value to the constant $A$ in the wavefunction solutions of
the infinite square well potential given in 
Eq.~(\ref{eq:squarewell_7}). Given that the wavefunction is zero outside
of the well, the total probability of finding the particle within the
well should be one.  Therefore,

\begin{equation}
\int_0^L |\psi_2(x)|^2 dx = \int_0^L A^2 \sin^2\left(\frac{n
  \pi}{L}x\right) dx = 1 .
\end{equation}
Using the table of integrals in Appendix A of your Wolfson text, we
can evaluate the integral using
\begin{equation}
\int_0^L \sin^2\left(\frac{n \pi}{L} x\right) = \frac{L}{2} .
\end{equation}
Therefore
\begin{equation}
A^2 \left( \frac{L}{2} \right) = 1 \hspace{0.3in} \mbox{and} \hspace{0.3in} A = \sqrt{\frac{2}{L}} .
\end{equation}
Notice that this value of $A$ is the same for all of the solutions,
independent of $n$.

\begin{example}{Probability of finding a particle II.}
A particle of mass $m$ is placed in an infinite square well of width
$L$ in a quantum state for which $n = 2$. (a) At what position(s) is
the particle most likely to be found within the well?  (b) What is the
probability of finding the particle between positions $x = L/4$ and $x
= L/2$?

{\bf Solution: (a)} The wavefunction for this particle is shown as
$\psi_2(x)$ in Fig.~\ref{fig:infinite_sq_well_solutions}.  Using
Eq.~(\ref{eq:squarewell_7}), the probability density for this
wavefunction is
\begin{equation}
|\psi_2(x)|^2 = \left(\sqrt{\frac{2}{L}} \right)^2 \sin^2\left(\frac{2 \pi}{L}x\right) ,
\end{equation}

\begin{figure}[!tb]
\begin{center}
\includegraphics[width=3in]{wavefunctions/ProbState_2.eps}
\end{center}
\caption{Probability density for a particle in the n=2 state of an infinite square well.}
\label{fig:ProbState_2}
\end{figure}

\noindent as shown in Fig.~\ref{fig:ProbState_2}. The positions
where the particle is most likely to be found are where the
probability density is the greatest. By examining the graph of
probability density, these positions are located at $x = L/4$ and $x =
3L/4$.  {\bf (b)} To calculate the probability of finding the particle
between $x = L/4$ and $x = L/2$, we use 
Eq.~(\ref{eq:IntegralprobDensity}):
\begin{equation}
Prob\left[\frac{L}{4}, \frac{L}{2}\right] = \int_{L/4}^{L/2} |\psi_2(x)|^2 dx \nonumber 
\end{equation}
\begin{equation}
 = \int_{L/4}^{L/2} \left(\frac{2}{L}\right) \sin^2\left(\frac{2 \pi}{L}x\right) dx
\nonumber 
\end{equation}
\begin{equation}
 = \frac{2}{L} \left[\frac{x}{2} - \frac{L}{8 \pi}\sin \left(\frac{4 \pi}{L}x\right) \right]_{L/4}^{L/2} =
 \frac{1}{4} \nonumber
\end{equation}
\noindent where we have used the table of integrals in Appendix A of
your Wolfson text.

Because of the symmetry of the probability density, the area under the
curve of $\left|\psi\right|^2$ between $x = L/4$ and $x = 3L/4$ is
one-quarter of the total area, therefore it makes sense that the
probability is $\frac{1}{4}$.
\end{example}
