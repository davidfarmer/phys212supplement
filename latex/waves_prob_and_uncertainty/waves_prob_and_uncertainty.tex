\chapter[Waves, Probabilities, and Uncertainty]{Waves, Probabilities, and 
  Uncertainty}

%\chapter[Uncertainty and Stability of Atoms]{Heisenberg's Uncertainty Principle 
%  and the Stability of Atoms}
\label{chapter:uncertainty}

\section{Introduction}
\label{sec:uncertainty_intro}

We concluded the last chapter with de~Broglie's hypothesis that all
matter exhibits wave-particle duality, not just light.  
%More
% specifically, he proposed that the particle property of momentum and
%the wave property of wavelength are related by
%\begin{equation}
%\label{eq:deBroglie}
%\lambda = \frac{h}{p}.
%\end{equation}
De Broglie's proposal is radical.  Arguing that all matter has wave-like
properties turns out to have far-reaching implications whose answers
fundamentally change the way we view the laws of the universe.
%The story is somewhat like Einstein's postulates for relativity, which
%were simple to state but ended up revising our fundamental notions of
%spacetime, mass, and energy when followed to their logical conclusion.
%One difference, though, is that the story in quantum mechanics had
%many authors: de~Broglie made the crucial hypothesis, but he was
%already building on the work of Einstein, Planck, and Bohr, and it took
%further developments by Heisenberg, Schr\"odinger, and others to
%turn quantum mechanics into a complete theory.
One important question is the following: waves are spread out over regions
of space; so, {\it where is the particle} if it is acting like a wave?

In this chapter, we will discuss the interpretation of de~Broglie's waves
in terms of {\it probability}. An implication of this probabilistic
description of matter waves is the well-known \textit{Heisenberg
uncertainty principle}, which states that a particle cannot simultaneously
have a precise position and momentum.  We will show that the uncertainty
principle ends up solving one of the failures of classical physics:
it explains why atoms are stable. The uncertainty principle also limits
our ability to measure and manipulate matter at small, sub-atomic scales.

We finish the chapter with a generalization of de~Broglie's relation --- 
Schr\"odinger's equation --- that can be used to \textit{calculate} 
wavefunctions for particles.

\section{Quantum Waves}
\label{sec:quantum_waves}

Let's consider a particle moving with a well-defined momentum
$p$. According to de~Broglie, this particle can also be
thought of as a wave with a wavelength $\lambda = h/p$. We can
represent this graphically as a simple sine wave, as shown in
Fig.~\ref{fig:wave_localization}(a).\footnote{Actually, a state with
well-defined momentum has a complex wavefunction $\psi(x) = Ae^{2\pi
ipx/h}$ whose real and imaginary parts are simple cosine and sine waves.}
But a wave doesn't have to be a simple sine wave. The {\it wavefunction}
$\psi(x)$ describing a particle's wave-like nature can take any shape.
Figure~\ref{fig:wave_localization}(b) shows a wavepacket-shaped
wavefunction, and Fig.~\ref{fig:wave_localization}(c) shows a pulse-like
wavefunction. Later in this chapter, we will explain how the specific
wavefunction can be determined for different physical situations.

\begin{figure}
\begin{center}
\includegraphics[width=5.4in]{waves_prob_and_uncertainty/wave_localization2}
\caption{Wavefunctions $\psi(x)$ discussed in the text: (a) a pure
sine wave, (b) a partially localized function, and (c) a localized blip.}
\label{fig:wave_localization}
\end{center}
\end{figure}

But first: what exactly {\it is} the wave here? An answer to that
question was provided in 1926 by Max Born who argued that the wavefunction
contains information about the probability for finding the particle.
Specifically, given a wavefunction $\psi(x)$ that describes the
wave properties of a particle, Born argued that $\psi(x)$ represents
\textit{probability amplitude}, and taking the magnitude squared of
the wavefunction produces the \textit{probability density} $P(x)$ ---
probability per unit distance for a one-dimensional problem --- for
finding the particle:
\begin{equation}
P(x) = |\psi(x)|^2.
\label{eq:probdensity}
\end{equation}
This tells us that according to quantum physics, a particle has a range
of possible positions, described by a probability density. We don't know
where the particle is located --- there is a possibility that it could
be found anywhere where the wavefunction $\psi(x)$ is non-zero. But it
isn't just a statement that we don't {\it know} where the particle is;
rather, {\it the particle doesn't HAVE a position} until we make a
measurement and locate it.

The interpretation of matter waves as probability amplitudes has
tremendous implications, which we will explore later in this chapter
and in the next few chapters. But before we explore those implications,
we need to develop the  mathematics of probability densities.

\subsection{Probability Densities}

The concept of probability density can be tricky --- it's not the same
as probability!  In one spatial dimension we write the probability
density as $P(x)$, since it can vary with the position $x$.  To get
probability from $P(x)$, you must multiply by a length interval since
it is really {\em probability per unit length}.  Specifically,
\begin{equation}
P(x)\, dx = \left(\begin{array}{l}
          \text{Probability of finding the particle within} \\
          \text{an interval from $x$ to $x+dx$. }
                    \end{array}\right).
\end{equation}

If the probability density is uniform, then it is easy to find
probability for any interval.  To illustrate, suppose you are playing
golf on a dark night (don't ask why) and after driving the ball you
know it landed somewhere along the left edge of the fairway, between
200 to 250 meters from the tee.\footnote{This is physics, so golf is
  played using the metric system.}  You want to find it and learn
about probability density at the same time.

The total probability of finding your ball between $ =200\units{m}$ 
and $x = 250\units{m}$ is 1 --- it is definitely
somewhere in that interval.  Furthermore, the probability density is
uniform.  That is, the ball is equally likely to be found anywhere in
the interval.  Then the probability density $P(x)$ is given by
\begin{equation}
P(x) = \frac{\text{Total Probability}}{\text{Total Length of Interval}}
\hspace{0.2in}\text{(uniform $P(x)$ only)},
\end{equation}
so in this case
\begin{equation}
P(x) = \frac{1}{(250\units{m}-200\units{m})} =
\frac{1}{50}\units{m$^{-1}$}.
\end{equation}
If you limit your search to the region between
$x_1 = 210\units{m}$ and $x_2 = 215\units{m}$, then your
probability of finding the ball is
\begin{align}
\text{Probability} &= (\text{Probability Density})\times
(\text{Length of Interval of Interest}) \nonumber \\ 
% &=& P(x)(x_2-x_1) \nonumber \\ 
 &= \frac{1}{50\units{m}} \left(215\units{m}-210\units{m}\right) 
 % \nonumber \\ &=&
 = \boxed{0.1 \text{ or 10\%}.}  %\frac{1}{10} %\nonumber \\ &=& 10\%
\label{eq:prob_density_uniform}
\end{align}

On the other hand, if the probability density is not uniform, you have
to integrate: % and Eq.~(\ref{eq:prob_density_uniform}) becomes
\begin{equation}
\text{Probability of finding ball between $x_1$ and $x_2$} 
   = \int_{x_1}^{x_2} P(x)\, dx.
\end{equation}
This is the general relation you should use to determine probabilities
for 1-D problems.

Finally, we note a useful graphical connection to probability.  Since
integrals have the nice interpretation of being area under a curve,
lots of understanding of these concepts can be gained by plotting
$P(x)$ vs.\ $x$, and simply calculating areas.
\newpage

\begin{example}{Probability Density I.}
A plot of a non-uniform probability density for finding your ball near
a different hole is shown below, in Fig.~\ref{fig:prob_density_golf}.

\begin{figure}[!h]
\begin{center}
\includegraphics[width=4.8in]{waves_prob_and_uncertainty/prob_density_golf}
\end{center}
\caption{Probability density for location of a golf ball.}
\label{fig:prob_density_golf}
\end{figure}



\begin{enumerate}
\item Show that the total probability of finding your ball is 1.
\item Use the plot to find the probability of finding your ball
  between 150 and 200 m from the tee.
\end{enumerate}
%\solution 
\begin{solution}
\begin{enumerate}
\item Since the total probability is the sum over the probabilities
 of being at all possible positions, we must integrate the probability 
density over all possible values of $x$, i.e., find the total area under 
the graph.  The total net area is a triangle with 
a base of $100\units{m}$ and a height of $0.02\units{m$^{-1}$}$, so 
\begin{equation}
\text{Probability} = \text{Area} = {\textstyle\frac{1}{2}}\times 100
\times 0.02 = 1  .
\end{equation}.
\item The probability is equal to $\int_{150}^{200}P(x)\, dx$, which is the 
area of the shaded trapezoid.  The trapezoid can be considered as a rectangle 
(of base $b = 50$ and height $h = 0.01$) plus a triangle (of base $b = 50$ 
and height $h = 0.01$). Therefore the total probability is
\begin{align}
\text{Probability} = \text{Area under curve} %\nonumber\\
  &= bh +  {\textstyle\frac{1}{2}b h}\nonumber \\
      &= 50 (0.01) + {\textstyle\frac{1}{2}}50 (0.02 - 0.01) \nonumber \\
                                 &= \boxed{0.75 \text{ or 75\%}.}
\end{align}
\end{enumerate}
\end{solution}
\end{example}

\subsection{Probabilities and Wavefunctions}
\label{sec:ProbDensity}
While a general wavefunction may be very complicated, the prescription
for its interpretation is always the same: from $\psi(x)$, calculate
$|\psi(x)|^2$, its magnitude squared.  Then the probability density is
given by $P(x)=|\psi(x)|^2$.

One of the complications of a general wavefunction is that it may be
complex-valued, involving factors of the imaginary number $ i = \sqrt{-1}$.  
When this occurs $|\psi(x)|^2$ does not just mean ``square the wavefunction''
to get the magnitude squared.  Rather you should follow this three-step
process:

\begin{enumerate}

\item[1.] Write the {\em complex conjugate} of the wavefunction, $\psi^*$,
  by replacing every occurrence of ``$i$'' with ``$-i$'' in the
  wavefunction.  For example if $\psi(x) = 2 e^{i3x} - 5 i x^2$, then
  $\psi^*(x) = 2 e^{-i3x} + 5 i x^2$.

\item[2.]  Multiply $\psi(x)$ times $\psi^*(x)$.

\item[3.] Replace any occurrences of ``$i^2$'' with ``$-1$'' and
  simplify. Your result should be real (no left-over ``$i$''s) and
  non-negative for all values of $x$, as required for a probability
  density.

\end{enumerate}

\begin{example}{Calculating probability density.}
\label{exam:CalcProbDensity}
A particle in a region of space with $x > 0$ is represented by 
the wavefunction
\begin{equation}
\psi(x) = \sqrt{5} e^{ikx} \sin(2x) e^{-x} \nonumber
\end{equation}
where $k$ is a real constant.  Calculate the probability
density for this particle.

{\bf Solution:} According to the prescription given above, the
probability density for this particle is calculated to be
\begin{eqnarray}
\left| \psi(x) \right| ^2 
   & = & \psi(x) \cdot \psi^*(x) \nonumber \\
   & = & \left( \sqrt{5} e^{ikx} \sin(2x) e^{-x} \right) 
         \left( \sqrt{5} e^{-ikx} \sin(2x) e^{-x} \right) \nonumber \\
   & = & 5 \sin^2(2x) e^{-2x}, \nonumber
\end{eqnarray}
where we have used the fact that
\begin{eqnarray}
(e^{ikx}) (e^{ikx})^* &=& (e^{ikx}) (e^{-ikx}) \nonumber \\
                      &=& e^{(ikx-ikx)} = e^0 = 1. \nonumber
\end{eqnarray}

%\begin{figure}[!hb]
%\begin{center}
%\includegraphics[width=3in]{waves_prob_and_uncertainty/fig42}
%\end{center}
%\caption{Probability density for the wavefunction of Example~\ref{exam:CalcProbDensity}.}
%\label{fig:prob_density_example}
%\end{figure}

%Fig.~\ref{fig:prob_density_example} shows a plot of $P(x)$ versus
%$x$. Using the interpretation for the probability density as given in
%Eq.~(\ref{eq:probDensity}), we see from
%Fig.~\ref{fig:prob_density_example} that $P(x)$ has a maximum near the
%position $x \approx 0.55$ and therefore the probability of finding the
%particle is the greatest in the vicinity of this position. 
\end{example}

%To determine the probability of finding a particle in a region of
%space, we integrate the probability density:
Figure~\ref{fig:prob_density_integral} is the graph of a probability
density for a particle in some quantum state.  The quantity $
P(x_0)\, dx$ represents the probability for finding the particle within a
narrow interval from $x_0$ to $x_0+dx$, shown as the shaded area
in the figure.  If we would like to know the probability of finding
the particle anywhere between the two positions $x_a$ and $x_b$, then
we need to {\em integrate} the probability density:

%\begin{figure}
\begin{figure}
\begin{center}
\includegraphics[width=4in]{waves_prob_and_uncertainty/prob_density_integral}
\end{center}
\caption{Determining the probability over a range of positions.}
\label{fig:prob_density_integral}
\end{figure}

\begin{equation}
\label{eq:IntegralprobDensity}
\left( \begin{array}{l}
         \mbox{Probability of finding the particle} \\
         \mbox{in the region $[x_a, x_b]$}
                    \end{array} \right) = \int_{x_a}^{x_b} P(x)\, dx
                                        = \int_{x_a}^{x_b} |\psi(x)|^2\, dx ,
\end{equation}
which is the area under the curve from $x_a$ to $x_b$.
Eq.~(\ref{eq:IntegralprobDensity}) brings up an important property of
wavefunctions and probability densities.  What if the region we are
interested in is the entire $x$-axis, that is, $[x_a, x_b] = (-\infty,
+\infty)$?  Another way of saying this is, `What is the probability of
finding the particle anywhere?'  Certainly the answer to this question
is $100\%$ !  This implies that the wavefunction must be given such that
\begin{equation}
\label{eq:normalization}
\int_{-\infty}^{+\infty} |\psi(x)|^2\, dx = 1 .
\end{equation}
When a wavefunction satisfies this requirement, it is said to be {\em
  normalized}.  Normalization is usually accomplished by multiplying
the wavefunction by a suitable constant factor, as we will see in the
following examples.

\begin{example}{Probability of finding a particle I.}
\label{ex:probabilityI}
Figure \ref{fig:example_wavefunction} shows the wavefunction for a
certain particle, where $A$ is a positive constant. (a) Determine a
value for the constant $A$ such that the wavefunction is properly
normalized. (b) What is the probability of finding the particle in the
region between $x = 2\units{nm}$ and $x = 4\units{nm}$?

\begin{figure}
\begin{center}
\includegraphics[width=3.5in]{waves_prob_and_uncertainty/prob_density_example}
\end{center}
\caption{Wavefunction for Example \ref{ex:probabilityI}.}
\label{fig:example_wavefunction}
\end{figure}

{\bf Solution: (a)} To determine probabilities we must first determine
$|\psi(x)|^2$ for the given wavefunction. We do this by computing the
value of $|\psi(x)|^2$ at every position on the graph which results in
the graph of Fig.~\ref{fig:example_prob_density}.

\begin{figure}[b]
\begin{center}
\includegraphics[width=3.5in]{waves_prob_and_uncertainty/prob_density_example_a}
\end{center}
\caption{Probability density for Example \ref{ex:probabilityI}.}
\label{fig:example_prob_density}
\end{figure}
The total probability of finding the particle anywhere is the total
area under the curve of Fig.~\ref{fig:example_prob_density}:
\begin{eqnarray} 
\int_{-\infty}^{+\infty} |\psi(x)|^2\, dx 
  &=& A^2 (2 - 1) + \left( - \frac{A}{2} \right)^2 (4 - 2) \nonumber \\
  &=& A^2 + \frac{A^2}{2} = \frac{3}{2} A^2 .
\end{eqnarray}

For the wavefunction to be properly normalized, this area must equal
one.  Therefore, the value for the constant $A$ must be
\begin{equation}
\frac{3}{2} A^2 = 1 \hspace{0.3in} 
              \longrightarrow \hspace{0.3in} A = \sqrt{\frac{2}{3}}.
\end{equation}
{\bf (b)} Now that we know the value for the constant $A$ that
properly normalizes the wavefunction, the probability of finding the
particle in the region between $x = 2\units{nm}$ and $x = 4\units{nm}$ 
is just the area under the curve of the probability density between 
$x = 2\units{nm}$ and $x = 4\units{nm}$:
\begin{equation}
\text{Prob}([2, 4]) = \frac{A^2}{4} (4 - 2) = \frac{A^2}{2} = \frac{1}{3}. 
\end{equation}
\end{example}

\section{Heisenberg's Uncertainty Principle}
\label{sec:heisenberg_principle}

Okay, back to Fig.~\ref{fig:wave_localization}. We have already
commented that a particle moving with a well-defined momentum
$p$ has a wavefunction $\psi(x)$ that is a pure sine wave, as in
Fig.~\ref{fig:wave_localization}(a).  But where is the particle when
it is in this state? The answer: it doesn't {\it have} a well-defined
position.  There is a non-zero probability density for finding the
particle anywhere where the wave is non-zero. But a pure sine wave with
a definite wavelength extends all the way from $x = -\infty$ to $x =
+\infty$. So a particle with a well-defined momentum could be found {\it
anywhere}. Phrased another way, the spread in the particle's location
is infinite.  So, if the momentum of a particle is defined precisely,
the particle's position is completely undetermined --- it could be found
anywhere. (And remember: it isn't just that we don't know where the
particle is located, but rather that the particle simply doesn't {\it
have} a location.)

But what if the particle's wavefunction looks like the one in
Fig.~\ref{fig:wave_localization}(c)? In this case, $\psi(x) = 0$
everywhere except in a narrow region where the sharp blip is. So, its
position is well-defined and predictable. It is theoretically possible to
make an arbitrarily narrow pulse-like wavefunction, i.e., with a spread
in position which is arbitrarily small.  {\it But what is the wavelength
--- and therefore the momentum --- of this particle?}

Formally, we use the symbol $\sigma_x$ to denote the spread in a
particle's position and $\sigma_p$ to denote the spread in a particle's
momentum.\footnote{Mathematically, $\sigma_x$ is the same as the standard
deviation
\begin{equation}
 \sigma_x = \sqrt{\langle x^2 \rangle - \langle x\rangle^2 }.
\end{equation}
where the averages $\langle x\rangle$ and $\langle x^2\rangle$ are
defined as
\begin{equation}
\langle x\rangle = \int x\, P(x)\, dx \qquad
\langle x^2\rangle = \int x^2\, P(x)\, dx
\end{equation}
with $P(x)$ being the probability density defined in the previous section.}
For waves, there is an important theorem called {\it Fourier's
theorem} that says that any function $f(x)$ --- which
might be describing the height of a water wave or the pressure of a
sound wave as a function of position or a quantum wavefunction $\psi$
--- can be decomposed into a sum of sines and cosines.\footnote{Fourier's
theorem is {\it really} cool and forms the basis for the theory of music
among other things. The basilar membrane in your ear performs an operation
very close to a Fourier transform, with different portions of the membrane
responding to different sound frequencies. This process helps
you to identify, say, the individual sounds produced by flutes,
cellos and bass guitars that are all blended together during a concert. 
And Fourier analysis is used to determine the characteristic ``fingerprint''
of different musical instruments and, in fact, is used by designers
of electronic music synthesizers to mimic  the sounds of real instruments.}
We won't prove it here, but Fourier's theorem shows that a pulse-like
wave with $\sigma_x = 0$ can be produced by adding sine waves with
wavelengths ranging from 0 up to $\infty$. But since each wavelength
corresponds to a different momentum, that means that a particle with
a well-defined position (i.e., $\sigma_x = 0$) could have any momentum
from infinity down to zero (i.e., $\sigma_p = \infty$).

So, a particle can have a well-defined momentum (but completely uncertain
position) or a well-defined position (but completely uncertain momentum).
There is a third possibility: Fig.~\ref{fig:wave_localization}(b) shows
a wavepacket that is reasonably (but not perfectly) localized and one
that can be decomposed into the addition of sine waves with a limited
(but non-zero) spread of wavelengths. In this case, the particle
is neither perfectly localized nor does it have a perfectly determined
momentum, but neither $\sigma_x$ nor $\sigma_p$ are infinite either.

A strict application of Fourier's theorem to matter waves shows that 
there is a minimum total spread in the particle's position and momentum:
\begin{equation}
  \sigma_x \sigma_p \geq \frac{\hbar}{2} ,
  \label{eq:heisenberg_uncertainty}
\end{equation}
where $\hbar \equiv h/2\pi$.
This is known as the Heisenberg uncertainty relation.  It is a profound
result that caused Newton to roll over in his grave\footnote{Well, okay,
maybe Newton didn't literally roll over in his grave, but this result
is {\bf very} different than anything envisioned by classical physics},
because it says that we cannot simultaneously know a particle's position
and momentum.  A precise position and a precise momentum would mean both
$\sigma_x$ and $\sigma_p$ are zero, but that is not allowed.

Recall when we solved for the motion of an object in classical
mechanics, say a blow dart shot straight upward, we needed to know
the initial position of the dart and the initial velocity.  The
Heisenberg uncertainty relation says that is not possible.  The
more precisely you are able to determine the position of some
object, the less you will be able to know about its momentum.

Now, $\hbar$ is a really small constant.  For distances and momenta on
our macroscopic scale, Heisenberg's uncertainty places no practical
limitations.  But down at the atomic scale the Heisenberg uncertainty
relation has a very big impact.  And, of course, experiments have
confirmed the uncertainty relation, so it is not just a proposal but
it is part of reality.

Many people, when encountering the uncertainty relation for the first
time, assume that this is a statement about the limits of our ability
to do experiments.  But that is not the case.  It's a property of
nature obeyed by all matter, even when we're not looking.  As we shall
see, the Heisenberg uncertainty relation is essentially the reason
that atoms are stable.

\section{The Stability of Atoms}
\label{sec:atom_stability}

Now we are ready to address the second great failure of classical physics.
Maxwell's equations make a crystal clear prediction: accelerating charges
send out energy in the form of EM waves.  This is a big problem for the
classical physics model of the atom.  If the electrons orbit the nucleus
like the planets orbit the sun, then they definitely have acceleration
(equal to $v^2/R$), and so they are radiating away their energy.

But where would that energy come from?  It comes from the potential energy
of the electron's electric force interaction with the proton.  To lower
its potential energy, the electron must come closer to the proton,
just like an object coming closer to the Sun lowers its gravitational
potential energy.\footnote{Actually, only half of the potential energy
is radiated away.  The other half goes into increased kinetic energy.
But the total mechanical energy is still decreasing.} So that is the
classical solution: the electron must radiate away energy by coming ever
nearer to the nucleus.

Working out the numbers involved for a hydrogen atom, we
find that the electron should spiral in and essentially crash into the
nucleus after a time of about $10^{-12}\units{s}$.  That is,
classical mechanics and electromagnetism predict that atoms
aren't stable and shouldn't last long enough for them (and us) to
still be around.  All the matter in our universe should be simply
electrons sitting in the nucleus with the protons and neutrons.

Heisenberg's uncertainty principle resolves this dilemma.  An electron
cannot sit on top of the nucleus because that would be a very precise
position and a very precise momentum (namely, at rest), which violates
the uncertainty relation, so the best the electron can do is some
compromise.  It accepts the minimal spread in position and momentum
that it can and then is stuck there.  Since it cannot lower its
energy further, it quits radiating and is stable.\footnote{Thank
  goodness!}

Put another way, whenever a particle is confined -- i.e., if $\sigma_x$
is less than $\infty$ -- then there is a minimum spread $\sigma_p
\ne 0$ in the momentum.  Assume the product $\sigma_x\sigma_p$ in the
Heisenberg uncertainty relation, Eq.~(\ref{eq:heisenberg_uncertainty}),
to be as small as possible, namely, $\sigma_x\sigma_p\approx \hbar/2$.
This gives us the inverse relation
\begin{equation}
 \sigma_p \approx\frac{\hbar}{2\sigma_x}.
 \label{eq:sigmap_from_sigmax}
\end{equation}
If $\sigma_p \ne 0$, there is a non-zero kinetic energy.  The average 
momentum must again be zero by symmetry, $\langle p\rangle=0$, so 
\begin{equation}
\sigma_p^2 = \langle p^2\rangle - \cancelto{0}{\langle p\rangle^2}
 = \langle p^2\rangle.
\end{equation}
The average kinetic energy can be related to the spread in momentum:
\begin{equation}
  \langle K\rangle = \frac{1}{2}m\langle v^2\rangle =
  \frac{1}{2m}\langle p^2\rangle = \frac{\sigma_p^2}{2m} .
  \label{eq:ke_from_spread}
\end{equation}
Conceptually, the since the electron is confined, it 
is forced to have some spread in velocity or momentum.  The
larger this spread, the larger the resulting average kinetic energy.

So, what does this mean? If an electron in an atom were to spiral into
the center, the spread in its position $\sigma_x$ would get smaller
and smaller. But smaller $\sigma_x$ would result in a larger and larger
$\sigma_p$ (by the uncertainty principle) and a larger kinetic energy
(larger than the classical increase in $K$ from spiraling inward). For
an electron in an atom, the reduction in its potential energy (by moving
closer to the center of the atom) would be countered by an increase in
its kinetic energy, which grows toward infinity if the electron spirals
all the way in. There is an optimal radial distance where the mechanical
energy $E = \langle K\rangle + \langle U\rangle$ reaches its minimum.
So, the classical problem of an electron spiraling in doesn't apply,
because eventually the energy {\it increases} if the electron gets closer
and closer to the center of the atom.

So, the condition for the electron to spiral in all the way is gone. It can't
keep radiating radiating EM waves, losing energy, and spiraling in to the
nucleus of the atom. Heisenberg's principle indicates that the energy of the
electron is not a minimum at the center of the atom but rather at
a finite radius. So, an electron would need {\bf more} energy to
fall further into the center of the atom, 

This is, of course, a simplification of the quantum explanation for
the stability of atoms. As we'll see in the next chapter, the energy of
an electron in an atom (and, in fact, the energy of any confined particle)
is quantized, having only certain, discrete values.

In the meantime, an important consequence of Heisenberg's Uncertainty
is the following:  

\boxittext{A confined particle can never have zero kinetic energy.} 
\break We will say a lot more about this in the
next chapter.

\section{Schr\"odinger's Wave Equation}
\label{sec:schrodinger_eqn}

We have talked about the idea of a wavefunction $\psi(x)$ that describes
the wave-nature of a particle, and we have introduced the idea that the
wave represents probability amplitudes, with the probability density
$P(x)$ for locating a particle being given by $P(x) = |\psi(x)|^2$. But
how can we determine what that wavefunction is for a specific problem?

In 1926, Erwin Schr\"odinger proposed a differential equation whose
solutions give the wavefunctions corresponding to matter waves.  For a
particle of mass $m$ and total energy $E$, Schr\"odinger's equation
written in one spatial dimension is
\begin{equation}
-\frac{\hbar^2}{2m}\frac{d^2\psi(x)}{dx^2} + U(x)\psi(x) = E\psi(x),
\label{eq:schrodinger}
\end{equation}
where $U(x)$ is the potential energy function of the
particle. Schr\"odinger's equation is similar in form to the types of
equations that describe other wave phenomena, such as sound waves and
electromagnetic waves.  The basic idea here is that, given a particle
with mass $m$ and a potential energy $U(x)$, Schr\"odinger's equation
will allow us to determine the possible energies $E$ and associated
wavefunction solutions $\psi(x)$.  

\subsection{Testing Solutions to an Equation.}

There are advanced techniques from the theory of differential equations
that we could use to start finding solutions to Schr\"odinger's equation
from scratch.  But for this course, we use only the ``guess and check''
method, which is also a standard technique in solving differential
equations. (In lab, you will solve Schr\"{o}dinger's equation using a
numerical iteration method.)  The approach is quite simple: you guess
a particular function, you put it into the equation and see if it works.
If it works for all values of $x$, then you say that it is a solution. If
it doesn't, then you say that it {\bf isn't} a solution.

We will illustrate the ``guess and check'' approach with an example:

% As a simple example of the guess-and-check approach, let's
% say, hypothetically, that you had never heard of the quadratic
% formula to figure out the solution to an equation of the form
% $ax^2+bx+c=0$.\footnote{We aren't going to write that formula down
% here because we want you to pretend that you had never heard of it.}
% If someone gave you an equation, say, $2x^2+10x+12 = 0$, you wouldn't be
% able to solve it since you don't know the quadratic formula.  But let's
% say that person said, ``Determine whether or not $x=-3$ is a solution to
% this equation.'' You might say, ``okay'' and substitute $x=-3$ into the
% equation to see if it works.  You'd get $2 \times (-3)^2 + 10 \times
% (-3) +12$ which gives you $18 - 30 + 12$ which {\it does}, in fact,
% equal 0. ``So, yes,'' you would say, ``$x = -3$ {\it is} a solution to
% the equation $2x^2+10x+12=0$.'' If you were asked to test if $x = 4$
% is a solution, you'd get $2 \times 4^2 + 10 \times 4 + 12$ which gives
% you $32+40+12$ which is 84.  ``84 isn't the same as 0, so $x = 4$ is
% {\bf not} a solution to the equation $2x^2+10x+12=0$.

% This kind of approach, i.e., substituting in a test solution to
% an equation to see if it works, can work with any kind of equation,
% including complicated-looking differential equations.  We will do another
% (longer) example to show how this works, and then apply the same approach
% to Schr\"odinger's equation.

\begin{example}{Testing solutions to an equation.}
Some equations have solutions which are functions, instead of just a number.
Let's say you are trying to determine the function $f(x)$ that solves
the following equation:
\begin{equation}
\left[f(x)\right]^2 - x^2 + 8x - 16 = 0.
\label{eq:TestEx1}
\end{equation}
You can probably solve this with algebra techniques, but let's pretend
that you don't know how to solve it. Instead, we are going to test trial
solutions.
\begin{enumerate}
\item[(a)]
Test the solution $f(x) = ax+b$ to see if it satisfies Eq.~(\ref{eq:TestEx1}).
If it does, determine the values of the constants $a$ and $b$ for the solution.
\item[(b)]
Test the solution $f(x) = ax$ to see if it satisfies Eq.~(\ref{eq:TestEx1}).
If it does, determine the value of the constant $a$ for the solution.
\item[(c)]
Test the solution $f(x) = ax^2+b$ to see if it satisfies Eq.~(\ref{eq:TestEx1}).
If it does, determine the values of the constants $a$ and $b$ for the solution.
\end{enumerate}
{\bf Solution: (a)} Substituting $f(x) = ax+b$ into Eq.~(\ref{eq:TestEx1}), 
we get
\begin{eqnarray}
\left[f(x)\right]^2-x^2+8x-16 & = & (ax+b)^2-x^2+8x-16 \nonumber \\
& = & a^2x^2+2abx+b^2-x^2+8x-16 \nonumber \\
& = & (a^2-1)x^2+x(2ab+8)+(b^2-16) \nonumber
\end{eqnarray}
The question is whether this can all add up to zero for all values of $x$.
The answer to that question is yes {\bf only} if the constants, the $x$-terms 
and the $x^2$ terms all separately add to zero.  That means that all of 
the terms in parentheses above need to add to zero separately.  (You can't
satisfy this by making $x=0$, because then the function would only work for
that particular value of $x$). So
\begin{eqnarray}
b^2-16 = 0 \rightarrow b = \pm 4 \nonumber \\
2ab+8 = 0 \rightarrow 2ab=-8 \rightarrow a=-4/b \nonumber \\
a^2-1=0 \rightarrow a= \pm1 \nonumber
\end{eqnarray}
All three of these relations work if $b=4$ and $a=-1$ or if $b=-4$ and
$a=1$. So, we can say that $f(x)=ax+b$ {\it is} a solution to 
Eq.~(\ref{eq:TestEx1}) if $b=4$ and $a=-1$ or if $b=-4$ and $a=1$, i.e.,
$f(x) = -x+4$ and $f(x)=x-4$ are both solutions.

{\bf (b)} Now let's substitute $f(x) = ax$ into Eq.~(\ref{eq:TestEx1}):
\begin{eqnarray}
\left[f(x)\right]^2-x^2+8x-16 & = & (ax)^2-x^2+8x-16 \nonumber \\
& = & a^2x^2-x^2+8x-16 \nonumber \\
& = & (a^2-1)x^2+8x-16 \nonumber
\end{eqnarray}
We could get rid of the $x^2$ term by making $a = \pm 1$, but we would still
be left with $8x - 16 = 0$. The only way to make that work would be to say
$x$ must be equal to 2. But that means that the test solution $f(x) = ax$
(with $a = 1$) would only work at $x = 2$. If $f(x)$ only works at
one particular value of $x$, then it isn't a solution of the equation.
(Solutions must work for {\bf every} value of $x$.)

{\bf (c)} Now let's substitute $f(x) = ax^2+b$ into Eq.~(\ref{eq:TestEx1}):
\begin{eqnarray}
\left[f(x)\right]^2-x^2+8x-16 & = & (ax^2+b)^2-x^2+8x-16 \nonumber \\
& = & a^2x^4+2abx^2+b^2-x^2+8x-16 \nonumber \\
& = & a^2x^4+(2ab^2-1)x^2+ (8)x +(b^2-16) \nonumber
\end{eqnarray}
This would work only if the coefficients of the $x^4$, $x^2$, $x$ and
constant terms were all zero. We can make the $x^4$ term disappear by
making $a=0$ and we can make the constant term disappear by making 
$b=\pm 4$, but that would leave non-zero coefficients for the
$x^2$ and $x$ terms. There is no combination of $a$ and $b$ that can
make all four terms disappear, so there is no function of the form
$f(x) = ax^2+b$ that can satisfy Eq.~(\ref{eq:TestEx1}).

\end{example}

The same approach works for {\bf any} equation, including {\it differential
equations} (i.e., equations that have derivatives in them).

\subsection{Testing Solutions of Schr\"odinger's Equation}
\label{sec:test_Solution}

Okay, let's look at Schr\"odinger's equation now. We won't ``solve'' this
equation (that requires mathematics beyond the scope of this course),
but we will test possible solutions to see if they work. First, since
Schr\"odinger's equation is a generalization of de~Broglie's result,
we should get the same result for a free particle (i.e., if $U(x)=0$).
De~Broglie says that a free particle with a momentum $p$ can be described
as a sine wave with a wavelength $\lambda=h/p$. So, let's try a test
solution $\psi_1(x)=A\sin(kx)$ and see if this satisfies Schr\"odinger's
equation.

With $U(x) = 0$, Schr\"odinger's equation becomes
\begin{equation}
-\frac{\hbar^2}{2m}\frac{d^2\psi_1(x)}{dx^2} + 0 = E\,\psi_1(x) .
\label{eq:schroed_with_0pe}
\end{equation}
The Schr\"odinger equation has the second derivative of the
wavefunction in it, so let's calculate that derivative for our trial
solution:
\begin{equation}
\frac{d\psi_1}{dx} = kA\cos(kx) \hspace{0.5in}  \mbox{and} \hspace{0.5in} 
\frac{d^2\psi_1}{dx^2} = -k^2A\sin(kx). \nonumber
\end{equation}
Next, substitute the second derivative and the wavefunction
into Eq.~(\ref{eq:schroed_with_0pe})
\begin{equation}
-\frac{\hbar^2}{2m}\left[-k^2A\sin(kx)\right] + 0 = E\left[A\sin(kx)\right] 
\end{equation}
and combine the $\sin(kx)$ terms
\begin{equation}
A\sin(kx)\left[\frac{\hbar^2k^2}{2m}-E\right] = 0.
\end{equation}
This works for $A=0$, but that isn't interesting because then $\psi(x)$ would
just be $0$. But $\psi_1(x)$ also works if the terms in brackets
add up to zero, i.e., if 
\begin{equation}
\frac{\hbar^2k^2}{2m} - E = 0 \longrightarrow E = \frac{\hbar^2k^2}{2m} 
\end{equation}
and
\begin{equation}
k=\pm\frac{\sqrt{2mE}}{\hbar},
\end{equation}
where $k$ is the wavenumber $k=2\pi/\lambda$.  The mechanical energy $E$ is
equal to the kinetic energy $K$ in this case (since $U = 0$ everywhere), so
\begin{equation}
\frac{2\pi}{\lambda}=\pm \frac{\sqrt{2m\left(\frac{1}{2}mv^2\right)}}{h/2\pi} 
= \frac{2\pi\sqrt{m^2v^2}}{h} \
= \frac{2\pi p}{h}. 
\end{equation}
We have dropped the $\pm$ since a negative wavelength doesn't have any
physical relevance.)  So, we can say that a test solution $\psi_1(x) =
A\sin(kx)$ {\it does} satisfy Schr\"odinger's equation for a free particle
($U = 0$), assuming the wavenumber $k$ has a value that corresponds to
a wavelength $\lambda = h/p$, as expected by de~Broglie's relation.

(Note that this process has not determined the value of $A$ in our test
solution.  The trial function $\psi_1(x) = A\sin(kx)$ works for any value
of $A$ (assuming the correct $k$). The value of $A$ needs to be determined
by normalization conditions. We'll discuss this in the next chapter.)

For contrast, let's try a test solution that does {\it not} satisfy
Schr\"odinger's equation.

\begin{example}{Trying another test solution for a free particle.}
Test the trial solution $\psi_1(x) = Bx^2$ to see whether or not 
it satisfies Schr\"odinger's equation for a free particle.

{\bf Solution:} With $U=0$, Schr\"odinger's equation is
\begin{equation}
-\frac{\hbar^2}{2m}\frac{d^2\psi_1(x)}{dx^2} + 0 = E\,\psi_1(x) . \nonumber
\end{equation}
Calculate the second derivative for our trial solution:
\begin{equation}
\frac{d\psi_1}{dx} = 2Bx \hspace{0.5in}  
  \mbox{and} \hspace{0.5in} \frac{d^2\psi_1}{dx^2} = 2B. 
\end{equation}
Next, substitute the second derivative and the wavefunction
into Schr\"odinger's equation:
\begin{equation}
-\frac{\hbar^2}{2m}\left[2B\right] + 0 = E\left[Bx^2\right], 
\end{equation}
and cancel common factors and simplify, giving
\begin{equation}
E B x^2 + \frac{\hbar^2 B}{m} = 0.
\end{equation}
In this equation there is no physically-meaningful choice 
of $E$ and $B$ that could make this true for \textbf{all} $x$,\footnote{$B=0$ 
would work, but then you'd have $\psi_1(x) = 0$, which isn't a 
physically relevant solution, because $\psi_1(x) = 0$ means that 
there is no probability of finding the particle anywhere.} so 
$\psi_1(x)$ is \textbf{not} a solution.
\end{example}
\newpage

Finally, let's illustrate a harder problem, the quantum harmonic
oscillator.  

\begin{example}{Test solution for a particle in a harmonic oscillator potential.}
Recall that the spring potential energy is given by 
$U(x) = \frac{1}{2}k_{\rm sp}x^2 = \frac{1}{2}m\omega^2x^2$, where 
$\omega = \sqrt{k_{\rm sp}/m}$.  Inserting this into the 
Schr\"odinger equation, we find
\begin{equation}
\frac{-\hbar^2}{2m}\frac{d^2\psi(x)}{dx^2} + \frac{1}{2}k_\text{sp}x^2\,\psi(x)
= E\,\psi(x).
\label{eq:schroed_spring}
\end{equation}
Try the test function $\psi_2(x)=xe^{-ax^2}$ to see if this is a solution
to Schr\"odinger's equation for a harmonic oscillator.  If it is,
determine the unknown constant $a$ in the solution and the unknown energy $E$.

%\solution 
\begin{solution}
Again, we must calculate the second derivative of this
function.  We will leave the calculation of the second derivative for
you to do (remember to be careful to correctly use the product rule and
the chain rule).
%We will need to use the product rule, so let's write $\psi_3(x)$ as
%\begin{equation}
%\psi_3(x)  =  xe^{-ax^2} =  x \left(e^{-ax^2}\right) 
%\end{equation}
%Then the first derivative is
%\begin{align}
%\frac{d\psi_3(x)}{dx} &= \left[\frac{d}{dx}x\right]\left(e^{-ax^2}\right) 
%+ x\frac{d}{dx}\left(e^{-ax^2}\right) \nonumber\\
%&= 1\left(e^{-ax^2}\right) + x e^{-ax^2} (-2ax)
%\end{align}
%In the second line, the factor of $(-2ax)$ came from the chain rule.
%Now we can factor out the common exponential to get
%\begin{equation}
%\frac{d\psi_3(x)}{dx} =  \left(1 - 2ax^2\right)e^{-ax^2} . 
%\end{equation}
%Whew!  Now we go to the second derivative.  We will need to use the
%product rule and chain rule once again:
%\begin{align}
%\frac{d^2\psi_3(x)}{dx^2} % &= \frac{d}{dx}\left[\left(1 - 2ax^2\right)e^{-ax^2}\right] \\
%&= \left[\frac{d}{dx}\left(1 - 2ax^2\right)\right]e^{-ax^2}
%+ \left(1 - 2ax^2\right)\frac{d}{dx}\left(e^{-ax^2}\right) \nonumber\\
%&= \left(-4ax\right)e^{-ax^2} + \left(1-2ax^2\right)e^{-ax^2}\left(-2ax\right)\nonumber\\
%&= \left(4a^2x^3 - 6ax\right)e^{-ax^2} .
%\end{align}
%Now substitute this and $\psi_3$ into the Schr\"odinger equation, Eq.~(\ref{eq:schroed_spring}):
%\begin{align}
%\frac{-\hbar^2}{2m}\left(4a^2x^3 - 6ax\right)\cancel{e^{-ax^2}}
%+\frac{1}{2}m\omega^2x^2\left(x\cancel{e^{-ax^2}}\right) &= E\left(x\cancel{e^{-ax^2}}\right) \\
%\Rightarrow\quad \frac{-2\hbar^2a^2}{m}x^3 + 3\frac{a\hbar^2}{m}x + \frac{1}{2}m\omega^2x^3
%- Ex &= 0 .
%\end{align}
The final result for the second derivative is
\begin{equation}
\label{eq:2ndDerivative}
\frac{d^2 \psi_2(x)}{dx^2} = \left(4 a^2 x^3 - 6 a x \right) e^{- a x^2}.
\end{equation}
Now substitute this and $\psi_2(x)$ into Schr\"odinger's equation,
Eq.~(\ref{eq:schroed_spring}):
\begin{equation}
-\frac{\hbar^2}{2m}\left(4a^2x^3 - 6ax\right) e^{-ax^2}
+\frac{1}{2}k_\text{sp}x^2\left(x \,e^{-ax^2}\right) = E\left(x \,e^{-ax^2}\right) .
\end{equation}
Every term has the same exponential factor in it, so we cancel those
to get
\begin{equation}
\frac{-2\hbar^2a^2}{m}x^3 + 3\frac{a\hbar^2}{m}x + \frac{1}{2}k_\text{sp}x^3
= Ex .
\end{equation}
Now to finish, move $Ex$ to the left hand side and collect common
powers of $x$:
\begin{equation}
x^3\left(\frac{1}{2}k_\text{sp} - \frac{2\hbar^2a^2}{m}\right)
+ x\left(3\frac{a\hbar^2}{m} - E\right) = 0.
\end{equation}

For this equation to hold for all $x$, we must make the coefficient of 
\textit{each} power of $x$ vanish separately.  So both the coefficient 
of $x^3$ \textit{and} the coefficient of $x$ are set to zero:
\begin{align}
\frac{1}{2}k_\text{sp} - \frac{2\hbar^2a^2}{m} = 0 \quad \text{and} \quad 
3\frac{a\hbar^2}{m} - E = 0.
\end{align}
Solving these for $a$ and $E$, we find
\begin{equation}
a = \frac{m\omega}{2\hbar} \quad 
\mbox{\bf and} \quad E = 3\frac{a\hbar^2}{m}
   = 3\left(\frac{m\omega}{2\hbar}\right)\frac{\hbar^2}{m} 
   = \frac{3}{2}\hbar\omega .
\end{equation}
We see that $\psi_2(x) = x e^{-ax^2}$ \textit{is} a solution to the
Schr\"odinger equation if the constants $a$ and $E$ are as given in
the previous equation.  We see that Schr\"odinger's equation gives
us the energy $E$ for this state, and also determines the unknown
constant $a$ in our trial wavefunction, $\psi_2$.  This wavefunction
solution is actually the first excited state of the quantum oscillator:
$\psi_2(x) = x\,e^{-(m\omega/2\hbar)x^2}$ with definite energy $E_2 =
\frac{3}{2}\hbar\omega$.  In one of the assigned problems at the end of
the chapter, you will test the wavefunction solution corresponding to
the ground state of the harmonic oscillator.
\end{solution}
\end{example}

We'll say a lot more about quantization of energy states in the next chapter.

\newpage

\section*{Problems}
\label{sec:uncertainty_problems}
\markright{PROBLEMS}


%one

\begin{problem} 
 {\bf Probabilities with dice.} This problem gets at the idea of a
 probability distribution. Take two of your dice, roll them, and
 record the sum of the spots showing on the top faces.  (This should
 be a number between 2 and 12!)  Repeat until you have recorded 20
 results.
   \begin{enumerate}
   \item  Collect your results in three ``bins.''  How many trials do
   you find in the bin containing numbers 2, 3, 4, or 5? How many in
   the 6, 7, or 8 bin?  In the 9, 10, 11, or 12 bin?  Try to explain any
   patterns.
   \item  In problem session, combine your results with others to
   estimate the  probabilities of getting a roll in each of the three
   bins.
   \item Can you calculate theoretical probabilities?  How do they
    compare with individual or class results?
   \end{enumerate}
\label{prob:dice}
\end{problem}

%two

\begin{problem}
{\bf Classical probabilities for finding your car.}  John, an aspiring
physics student, works part-time parking cars at a downtown hotel.
The lot is a long, underground tunnel, with all the cars parked in a
single long row, $600\units{m}$ long.  When owners return for their
cars, instead of telling them exactly where to find their cars, he
describes the location in terms of probability and probability
density.
\label{prob:parking_lot_1}
   \begin{enumerate}
   \item Mr.~Vanderbilt is told that his car ``could be anywhere in
     the lot,'' which means that the probability density is constant.
     Calculate the value of this uniform probability density $P(x)$
     for Mr.~Vanderbilt to find his car a distance $x$ from one end of
     the lot.  (Answer in units of probability/m.)
    \item Find the probability that Mr.~Vanderbilt's car is in the first 
     $100\units{m}$ of the lot.
    \item Mrs.~Reeve is told that the probability density to find her
      car is a constant $P_1$ from $x = 0$ to $x=200\units{m}$, and
      a second constant \mbox{$P_2$ = $P_1/3$} in for $x=200$ to 
      $x=600\units{m}$.  Find the different constant probability densities
      $P_1$ for $0 < x < 200\units{m}$ and $P_2$ for $200\units{m}
      < x < 600\units{m}$.
     \item Based on your results from part (c), find the probability
       that Mrs.~Reeve's car is in the first $400\units{m}$ of the
       lot.
     \end{enumerate}
\end{problem}

\newpage

%three

\begin{problem}
The probability amplitude $\psi(x)$ for a certain particle to be at 
position $x$ is
\[ 
\psi(x) = 
\frac{\sqrt{x}}{a\sqrt{2}}, \hspace{0.2in}\text{for $0\leq x \leq a$.}
% \begin{cases} \sqrt{2x}/a & \text{for $0\leq x \leq a$} \\
%                     0    & \text{for $x<0$ or $x>a$.} 
% \end{cases} 
\]
\begin{enumerate}
\item Explain what the quantity $|\psi(x)|^2$ tells us about the particle.

\item Calculate the probability that the particle is found between $x=0$ 
and $x = a$. 

\item Calculate the probability that the particle is found anywhere else
(i.e., {\bf not} between $x=0$ and $x=a$).
\end{enumerate}
\end{problem}

%four

\begin{problem}
Determine the probability of finding a particle between $x = 1.0$ and
$x = 2.0$ for the following wavefunctions:
\begin{enumerate}
\item $\psi(x) = 0.10x+0.50$
%
% $\int_1^2 (0.1x+0.5)^2dx = \int_1^2 (0.01x^2+0.1x+0.25)dx = (0.01x^3/3
% +0.05x^2+0.25x)_1^2 = (0.08/3-0.01/3)+(0.2-0.05)+(0.5-0.25)
% = 0.25+0.15+0.02 = 0.42$
%
\item $\psi(x) = 0.10x+0.50i$
%
% $\int_1^2 (0.1x+0.5i)(0.1x-0.5i)dx = \int_1^2 (0.01x^2+0.25)dx
% = (0.08/3-0.01/3)+(0.5-0.25)$ The difference is the (0.15) from the
% cross terms. i.e., 0.27.
\end{enumerate}
\label{prob:WavefunctionProbs}
\end{problem}

% Old problem 4
% \begin{problem}
% A particle's wavefunction is given as $\psi(x) = A e^{-x^2/a^2}$, 
% where $A$ and $a$ are constants.
% \label{prob:GausWavefunction}
% \begin{enumerate}
% 
% \item In the vicinity of what position(s) $x$ is the particle most 
% likely to be found?
% 
% \item At what position(s) $x$ is the probability density half of 
% its maximum value?
%
% \end{enumerate}
% \end{problem}

%five 

\begin{problem}
  An electron in a hydrogen atom has a spread in position typically 
  around $\sigma_x = 5\times 10^{-11}\units{m}$.
  \begin{enumerate}
  \item Use the Heisenberg uncertainty relation to find a lower bound
    on the spread in momentum, $\sigma_{p_x}$, for the electron.

  \item Now find a lower bound on the spread in velocity, $\sigma_{v_x}$.

%  \item Since the electron is stays within the atom, the average
%    velocity must be zero.  Therefore the nonzero spread, $\sigma_v$, tells
%    us the scale of typical speeds that we would find if we measured
%    the electron's speed.  Based on your answer to part (b), do
%    electrons in hydrogen atoms reach relativistic speeds?
  \end{enumerate}
\label{prob:UncertaintyHelectron}
\end{problem}

%six

\begin{problem}
  The Heisenberg uncertainty relation does not pose much of a
  limitation on our macroscopic scale.  Consider a blow dart of mass
  $2.5\units{g}$.  Let's imagine a measurement of the dart's position
  with a precision limited to $1\units{$\mu$m}$.  Determine the lower 
  bound on the spread in velocity imposed by the Heisenberg uncertainty 
  relation.
\label{prob:UncertaintyBlowDart}
\end{problem}

\newpage

%seven

\begin{problem}
Fabrication of nano-scale devices requires the ability to position atoms
and molecules with very small spatial spread $\sigma_x$.

\begin{enumerate}

\item Use Heisenberg's uncertainty principle to make a rough
estimate of the precision by which a carbon atom (mass $2.00 \times
10^{-26}\units{kg}$ and radius $70\units{pm}$) can be confined (i.e.,
determine the smallest $\sigma_x$) to keep its minimum kinetic energy
below typical molecular binding energies of around $1\units{eV}$.
Would you expect the uncertainty principle to be a problem if you wanted
to arrange carbon atoms in a nanotech device with a precision of around
one-hundredth of the radius of a carbon atom?

\item Repeat the calculations in part (a) for a proton (mass 
$1.67 \times 10^{-27}\units{kg}$), also with $\langle K\rangle$ 
below $1\units{eV}$. 

\item 
Compare your result from (b) to the
size of the nucleus of a carbon atom ($2.7 \times 10^{-15}\units{m}$).
What does this imply about a proton confined inside a carbon nucleus?
I.e., what must be true for a proton to remain bound inside a carbon 
nucleus?

\end{enumerate}
\label{prob:NanoscaleUncertainty}
\end{problem}

%eight

\begin{problem}
Back in 1975, Gordon Moore proposed that the number of transistors
per area on integrated circuits roughly doubles every two years.
This principle (``Moore's Law'') has worked surprisingly well for
40 years now, with transistors introduced in 2012 as small as $22\units{nm}$
and techniques are continually being developed to make them even 
smaller. But Moore's Law will eventually fail due to limitations 
imposed by the uncertainty principle.

\begin{enumerate}
\item Use the uncertainty principle to calculate the smallest spread
$\sigma_x$ for an electron such that its minimum kinetic energy (due to
uncertainty) is below the work function (binding energy) of silicon,
which is $4.05\units{eV}$.  
\item Let's assume that the smallest
possible transistor has an area 100 times the square of the $\sigma_x$
that you calculated in part (a).  Given the area (approximated as the square
of $22\units{nm}$) for the best transistors from 2012, 
if Moore's Law continues to hold into the future, roughly what year will
transistors reach this quantum limit?  
\end{enumerate}
\end{problem}

%nine

\begin{problem}
Explain in a few sentences why classical physics (Newtonian mechanics and 
Electricity and Magnetism) predict that atoms with electrons
orbiting around a nucleus are unstable and can't exist
indefinitely.
\end{problem}

\newpage

%ten

\begin{problem}
Use the uncertainty principle to estimate the minimum $\langle K\rangle$ 
for 
\begin{enumerate}
\item an electron confined to a region of $50\units{pm}$ (roughly the 
radius of a hydrogen atom); 
\item a DNA molecule ($1.0 \times 10^{-25}\units{kg}$) confined 
to the nucleus of a cell $3.0\units{$\mu$m}$ radius; and
\item a $5.0\units{mg}$ grain of sand in a pill box with width 
$2.0\units{cm}$.
\end{enumerate}
\label{prob:MinK}
\end{problem}

%eleven

\begin{problem}
Use the uncertainty principle to explain in 2 or 3 sentences 
\begin{enumerate}
\item why it is not possible for a confined particle to have zero
kinetic energy; and
\item why everyday-size objects (even something as small as a speck
of dust) often seem to have zero kinetic energy, even when confined
to a small region.
\end{enumerate}
\end{problem}

%twelve

\begin{problem}
Given the equation $df/dx = 4.0\sin(0.25x)$, test (by direct substitution)
to determine if the following functions are solutions.  If so, determine
possible values of any constants.  
\begin{enumerate}
\item $f(x) = Bx^2$,
\item $f(x) = B\sin(kx)$,
\item $f(x) = B\cos(kx)$.
\end{enumerate}
\end{problem}

%thirteen

\begin{problem}
Given the equation $d^2f/dx^2 = 5x + 6$, test (by direct substitution)
to determine if the following functions are solutions.  If so, determine
possible values of any constants.  
\begin{enumerate}
\item $f(x) = Ax^3+Bx^2+Cx$;
\item $f(x) = A\sin(kx)$.
\end{enumerate}
\label{prob:SubstitutionPractice}
\end{problem}

\newpage

%fourteen

\begin{problem}
{\bf Schr\"{o}dinger equation for a classically allowed situation.}
Consider a particle of mass $m$ in a region in which the potential energy
is constant, i.e., $U(x)=U_0$, and assume that the total energy of
the particle $E$ is greater than the potential energy, i.e., $E>U_0$.
(This is the case for classically allowed motion.)  To determine
the wave function we must find a function $\psi(x)$  that satisfies
the one-dimensional Schr\"{o}dinger equation \[ -\frac{\hbar^2}{2m}
\frac{d^2\psi(x)}{dx^2} + U(x)\psi(x) = E\psi(x).  \]

In this problem you will try three ``guesses'' for $\psi(x)$ 
and see if they satisfy Schr\"{o}dinger's equation. The 
three ``guesses'' are
\begin{itemize}
\item $\psi_1(x) = Ax^2$ 
\item $\psi_2(x) = B\sin(kx)$
\item $\psi_3(x) = Ce^{-\kappa x}$,
\end{itemize}
where $A$, $B$, $C$, $k$, and $\kappa$ are undetermined real
constants.
    \begin{enumerate}
    \item Rearrange Schr\"{o}dinger's equation so that the
    second derivative $d^2\psi/dx^2$ is alone on the left.
    \item Plug $\psi_1(x)$ into Schr\"{o}dinger's equation and 
    see if there is any choice for the constant $A$ that will
    make $\psi_1(x)$ satisfy the equation for all values of $x$.
    \item Plug $\psi_2(x)$ into Schr\"{o}dinger's equation and see
    if there is any choice for the constants $B$ and $k$ that will
    make $\psi_2(x)$ satisfy the equation for all values of $x$ .  
    \item  Plug $\psi_3(x)$ into Schr\"{o}dinger's equation and see
    if there is any choice for the constants $C$ and $\kappa$ that will
    make $\psi_3(x)$ satisfy the equation for all values of $x$. 
    \item You should have found that $\psi_2(x)$ can be a solution for
    the proper choice of $k$.  Determine the wavelength of the
    oscillations in terms of $\hbar$, $m$, $E$, and $U_0$. (i.e., solve
    for $k$ and remember from the waves unit that $k=2\pi/\lambda$.)  
    Is your result consistent with that predicted from the de~Broglie 
    relationship?  (Hint: $E-U_0$ is the kinetic energy $K = p^2/2m$.  
    Re-write things in terms of the momentum and the answer should drop 
    into your lap.)
    \end{enumerate}
\label{prob:TestSchrod}
\end{problem}

% \newpage
% fifteen

\begin{problem}
{\bf Schr\"{o}dinger equation for classically forbidden
situation.}  Consider a particle of mass $m$ in a region with a
constant potential energy $U_0$, and assume that the total energy of
the particle $E$ is {\em less} than the potential energy,
i.e., $E<U_0$.  (This isn't possible for classical motion, but continue
anyway.)  To determine the wave function we must find a function
$\psi(x)$ that satisfies the one-dimensional Schr\"{o}dinger equation
\[ -\frac{\hbar^2}{2m} \frac{d^2\psi(x)}{dx^2} + U(x)\psi(x) 
= E\psi(x).  \]
In this problem you will try three ``guesses'' for $\psi(x)$ 
and see if they satisfy Schr\"{o}dinger's equation. The 
three ``guesses'' are 
\begin{itemize}
\item $\psi_1(x) = Ax^2$ 
\item $\psi_2(x) = B\sin(kx)$
\item and $\psi_3(x) = Ce^{-\kappa x}$,
\end{itemize}
where $A$, $B$, $C$, $k$, and $\kappa$ are undetermined real
constants.
    \begin{enumerate}
    \item Rearrange Schr\"{o}dinger's equation so that the
    second derivative $d^2\psi/dx^2$ is alone on the left.
    \item Plug $\psi_1(x)$ into Schr\"{o}dinger's equation and 
    see if there is any choice for the constant $A$ that will
    make $\psi_1(x)$ satisfy the equation for all values of $x$.
    \item Plug $\psi_2(x)$ into Schr\"{o}dinger's equation and see
    if there is any choice for the constants $B$ and $k$ that will
    make $\psi_2(x)$ satisfy the equation for all values of $x$.  
    \item  Plug $\psi_3(x)$ into Schr\"{o}dinger's equation and see
    if there is any choice for the constants $C$ and $\kappa$ that will
    make $\psi_3(x)$ satisfy the equation for all values of $x$. 
    \end{enumerate} 
\end{problem}

%sixteen

\begin{problem}
A special case of the quantum harmonic oscillator is described by 
Schr\"{o}dinger's equation of the form
\[ -\frac{d^2\psi}{dx^2} + 4x^2 \psi = E\psi.  \]
Substitute the trial solution $\psi = Ae^{-x^2}$ and determine the value 
of the energy $E$.
\label{prob:harmonic_oscillator}
\end{problem}

%seventeen

\begin{problem}
\label{prob:harmonicOsc_ground}
 The general case Schr\"{o}dinger equation for the one-dimensional
 harmonic oscillator is
\[ -\frac{\hbar^2}{2m}\frac{d^2\psi}{dx^2} + \frac{1}{2}m\omega^2 x^2 \psi
   = E\psi.      \]
Substitute the trial solution $\psi(x) = Ae^{-ax^2}$ (with $a>0$) 
into this equation and determine the constants $a$ and $E$. This 
is the ground state of the oscillator.  
\end{problem}

%eighteen, but probably leave out

%\begin{problem} \label{prob:harmonicwavefn}
%In section (4.6) we examined a trial wavefunction solution for the Schr\"{o}dinger
%Equation describing a quantum harmonic oscillator.  The trial wavefunction was given
%as
%\[ \psi_2(x) = x e^{-a x^2} . \]

%The Schr\"{o}dinger equation, given by Eq.(4.28) requires taking the second
%derivative $$\psi_2^{''}(x)=\frac{d^2 \psi_2}{dx^2}$$ of this trial solution. 
%Perform this calculation and compare your result with Eq.~(\ref{eq:2ndDerivative}).
%\end{problem}


