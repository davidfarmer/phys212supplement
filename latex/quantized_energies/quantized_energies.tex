\chapter[Quantized Energies and Spectra]{Quantized Energies and Spectra}
\label{chapter:quantized_energies}
%\setcounter{ex}{0}

\section{Introduction}
\label{sec:quantized_energies_intro}

\indent We have seen that particles have wave-like
properties, and that the de~Broglie relation, connecting the
properties of momentum and wavelength, has far-reaching consequences
for the behavior of atoms and other microscopic systems. 
The main consequence is that the idea of a particle as a
point-like object having a precise position at any given time has to
be replaced with a probability density describing a distribution of
positions where the particle is likely to be found. 

In the previous chapter, we introduced the idea of a wavefunction
$\psi(x)$ from which we can determine the probability density
$P(x) = |\psi(x)|^2$. We also discussed Heisenberg's uncertainty
principle, that states that there is a minimum combined spread in 
a particles position and momentum, a principle that also leads to
a minimum kinetic energy for any confined particle.
And at the end of the previous chapter, we
introduced the Schr\"odinger equation which enables us to determine
wavefunctions for particles in regions of given potential energy.

In this chapter, we show how Schr\"odinger's Equation predicts
the {\it quantization} of energies for confined particles, i.e., only
certain, well-defined energies are allowed.\footnote{This
is where the ``quantum'' in the name ``quantum mechanics'' comes from.}
Quantization of energies is a profound and distinctly quantum principle
(i.e., not predicted at all from classical mechanics) which has extensive
applications in modern technology. We will illustrate quantization of
particle energies with a simple system --- the {\it particle in a box} (also
known as the {\it infinite square well potential}) --- for which
solutions of Schr\"odinger's Equation are readily obtained. 
We will discuss the properties of this system, which has recently
led to the development of {\it quantum dots} for use in numerous
applications.  

We will also discuss how quantum systems absorb and emit
light, a result that explains not only the colors that we see from
many physical systems but which has also become a common tool for
identifying the material constituents of a range of systems, from
biological systems in microscopic studies up to stars 
and star-forming regions
millions (and billions) of light years from the Earth.

\section{Solutions for the Infinite Square Well}
\label{sec:infinite_sq_well}

To demonstrate how we use Schr\"odinger's equation to derive
wavefunction solutions, let's consider the simplest example of a
``particle in a box,'' i.e., a
particle trapped in one dimension (and unable to move in the other
two dimensions) between two impenetrable walls
located at $x = 0$ and $x = L$.  Since the particle feels no force in
the region between the walls, the potential energy in this region is
constant, and therefore we will take it to be zero.  Since the walls
are impenetrable, the potential energy must become infinite at $x = 0$
and $x = L$.  Therefore the graph of the potential energy $U(x)$ as a
function of position is shown in Fig.~\ref{fig:infinite_sq_well}.
This configuration is usually referred to as the {\it infinite square
  well potential}.  For this example we will consider solutions for
which the energy of the particle $E$ is positive. (Solutions for $E <
0$ for this potential actually do not exist due to the fact that the
potential energy goes to infinity at the boundaries of the well.)

\begin{figure}[!tb]
\begin{center}
\includegraphics[width=3.2in]{quantized_energies/infinite_sq_well}
\end{center}
\caption{Graph of the infinite square well potential as a function of
  position.}
\label{fig:infinite_sq_well}
\end{figure}


Since there is no chance that the particle can penetrate either of the
barriers at $x = 0$ or $x = L$, the wavefunction $\psi(x)$ must be
exactly zero in the regions $x \le 0$ and $x > L$.  Inside the well,
where $U(x) = 0$, Schr\"odinger's equation can be written as

\begin{equation}
\frac{d^2\psi(x)}{dx^2}  = - \left(\frac{2mE}{\hbar^2}\right)\psi(x).
\label{eq:squarewell_1}
\end{equation}

Our job is to determine what mathematical function is consistent with
Eq.~(\ref{eq:squarewell_1}), that is, can we find a function
$\psi(x)$ whose second derivative is equal to {\it minus} a positive
constant $(2 m E/ \hbar^2)$ times the original function?  
We already testing a solution to this equation in Section 
\ref{sec:test_Solution}.
In that section, we found that the wavefunction 
$\psi_1(x) = A\sin(kx)$
works if 
\begin{equation}
\label{eq:squarewell_2}
E = \frac{\hbar^2 k^2}{2m}.
\end{equation}
from which it follows that $k = \pm\sqrt{2mE}/\hbar$. 
This isn't the only possible solution --- the wavefunction 
$\psi_1(x) = B\cos(kx)$ also satisfies Eq.~(\ref{eq:squarewell_1}).
Since either of these functions satisfies Schr\"odinger's equation, we
can write a general solution as a linear combination of these two
functions:

\begin{equation}
\label{eq:squarewell_3}
\psi(x) = A \sin{(kx)} + B \cos{(kx)} .
\end{equation}

In order for this solution to work, we must adjust the values of the
constants $A$, $B$, and $k$ such that the wavefunction inside the well
matches up with the value of the wavefunction at positions $x = 0$ and
$x = L$, where we previously concluded that the value of the
wavefunction should be zero.  This process is called {\it matching the
 boundary conditions}.  At $x = 0$, the wavefunction inside the well given
by Eq.~(\ref{eq:squarewell_3}) becomes $\psi(0) = B$, so in order
for the wavefunction to equal zero at the boundary $x = 0$, we must
choose the value $B = 0$.  Thus the interior wavefunction must be
given simply as $\psi(x) = A \sin{(kx)}$.

Continuing, we now look at the boundary condition at $x = L$.  Again,
the wavefunction must be zero here, so this requires that

\begin{equation}
\label{eq:squarewell_4}
A \sin{(k L)} = 0.
\end{equation}

\noindent From our knowledge of the sine function, we know that this
function is zero for values such that $k L = 0, \pi, 2\pi, 3\pi,
\cdots = n \pi$ for any integer $n$.  Therefore this second boundary
condition tells us that the possible values the constant $k$ can
assume are

\begin{equation}
\label{eq:squarewell_5}
k_n = \frac{n \pi}{L} . \hspace{0.5in} (n = 1, 2, 3, \dots)
\end{equation}
We have thrown out the $k = 0$ solution because that would correspond
to a wavefunction $\psi(x) = 0$ which corresponds to a particle that has
zero probability of being found anywhere, (i.e., the particle doesn't
exist).
Combining these values for $k$ with Eq.~(\ref{eq:squarewell_2}),
we find that the possible energies $E_n$ that the particle can have
inside the well are given by

\begin{equation}
\label{eq:squarewell_6}
E_n = n^2 \frac{\pi^2 \hbar^2}{2mL^2} 
= n^2 \frac{h^2}{8 m L^2} \hspace{0.2in} (n = 1, 2, 3, \dots),
\end{equation}

\noindent and the wavefunctions associated with each of these energies are

\begin{equation}
\label{eq:squarewell_7}
\psi_n(x) = A \sin{\left(\frac{n \pi}{L}x\right)} .
\end{equation}

Fig.~\ref{fig:infinite_sq_well_energy} shows a graph of the
allowed energies $E_n$ for a particle in an infinite square well
potential superimposed on the graph of potential energy $U(x)$ and
Fig.~\ref{fig:infinite_sq_well_solutions} shows the wavefunctions
$\psi_n(x)$ for the four lowest allowed energies.

A few general remarks are in order about these solutions. 
\begin{figure}[!b]
\begin{center}
\includegraphics[width=3.2in]{quantized_energies/infinite_sq_well_energy}
\end{center}
\caption{The allowed energies for a particle bound in an infinite
  square well potential. The energies are expressed in terms of the
  lowest (ground state) energy $E_1$.}
\label{fig:infinite_sq_well_energy}
\end{figure}

\begin{figure}[!t]
\begin{center}
\includegraphics[width=4.8in]{quantized_energies/infinite_sq_well_sols}
\end{center}
\caption{The wavefunction solutions for the infinite square well
  potential for the four lowest energy states.}
\label{fig:infinite_sq_well_solutions}
\end{figure} 

\begin{itemize} 
\item Notice that the lowest allowed energy for a particle is {\bf
  not} zero!  This coincides with our discussion in
  chapter~\ref{chapter:uncertainty} in which Heisenberg's uncertainty
  principle requires a minimum non-zero energy for a particle that is
  confined to a finite region of space.

\item There are only certain values of the energy $E$ for which there are
  well-behaved solutions to Schr\"odinger's equation. When a particle is trapped 
  in a finite region of space, its energy cannot be any continuous value
  but rather can only be one of many discrete values of energy. In the
  language of quantum mechanics, the energy is {\it quantized}.

\item The integer $n$ is referred to as the {\it quantum number} of
  the particle. For this problem, specifying a value for $n$ uniquely 
  determines the state of the particle.  As we shall see in more
  complicated systems, more than one quantum number will often be needed 
  to completely specify a state.

\item The wavefunctions as given in Eq.~(\ref{eq:squarewell_7})
  and Fig.~\ref{fig:infinite_sq_well_solutions} should look
  familiar to you.  Aren't these the same functions we used when
  describing the standing wave patterns on a vibrating string fixed at
  both ends?  Indeed they are! In fact, standing waves on a string are
  determined by a differential equation similar to Eq.~(\ref{eq:squarewell_1}).
  Standing waves on a string exhibit their own form of quantization --
  in this case, the properties that are quantized are the wavelengths
  and frequencies.

\end{itemize}

\begin{example}{Electron trapped in a 1-D nanotube.}
\label{exam:nanoTube}
Carbon atoms can be bonded into a cylindrical arrangement called a
carbon nanotube.  Carbon nanotubes can be fabricated with
length-to-diameter ratios that are extremely large. These nanotubes
have a broad range of applications including uses as conducting nano-wires
and mechanical ``scaffolding'' for growing new bone cells.

A carbon nanotube of length $L$ can be approximated as an infinite 
square well of width $L$.  Consider a particular carbon nanotube 
of length $L = 10.0\units{nm}$ and negligible diameter.  Calculate 
the ground state energy for an electron trapped inside of this carbon nanotube.

{\bf Solution:} Approximating the carbon nanotube as an infinite
square well, the possible allowed energies for the electron are 
given by Eq.~(\ref{eq:squarewell_6}).  For the ground state
energy, where $n = 1$, the energy is given as
\begin{equation}
E_1 = (1)^2 \frac{h^2}{8 m_\text{e} L^2}. 
\end{equation}
Because the length is given in nm, and eV's are convenient units 
for energy at the microscopic scale, it makes calculations easier 
if we multiply both the numerator and denominator by $c^2$, 
giving 
\begin{equation}
E_1 = (1)^2 \frac{(hc)^2}{8 (m_\text{e}c^2) L^2}. 
\end{equation}
Then we can insert the values $hc = 1240\units{eV$\cdot$nm}$, 
$L = 10\units{nm}$, and $m_\text{e}c^2 = 0.511\units{MeV} 
= 511\times 10^3\units{eV}$, giving
\begin{eqnarray}
E_1 &=& (1)^2 \frac{(1240\units{eV$\cdot$nm})^2}
    {8\times 511\times 10^3\units{eV}\times (10\units{nm})^2  }. \nonumber \\
    &=& 3.8\units {meV} 
\end{eqnarray} 
It is, of course, valid to use SI units for $h$, $m_\text{e}$, and $L$, but 
this entails calculations with large exponents and more unit conversions.
\end{example}

% \section{Probability and Probability Density for the Particle in a Box}
% \label{sec:ProbDensity}

We can use the idea of normalization (see Sec.~\ref{sec:ProbDensity}) to
assign a value to the constant $A$ in the wavefunction solutions of
the infinite square well potential given in 
Eq.~(\ref{eq:squarewell_7}). Given that the wavefunction is zero outside
of the well, the total probability of finding the particle within the
well should be one.  Therefore,

\begin{equation}
\int_0^L |\psi_2(x)|^2\, dx 
  = \int_0^L A^2 \sin^2\left(\frac{n \pi}{L}x\right)\, dx = 1 .
\end{equation}
Using the table of integrals in Appendix A of your Wolfson text,
% or Wolfram Alpha online (www.wolframalpha.com), 
we can evaluate the integral using
\begin{equation}
\int_0^L \sin^2\left(\frac{n \pi}{L} x\right) = \frac{L}{2} .
\end{equation}
Therefore
\begin{equation}
A^2 \left( \frac{L}{2} \right) = 1 \hspace{0.3in} 
              \mbox{and} \hspace{0.3in} A = \sqrt{\frac{2}{L}} .
\end{equation}
Notice that this value of $A$ is the same for all of the solutions,
independent of $n$.

\begin{example}{Probability of finding a particle in an infinite square well potential.}
A particle of mass $m$ is placed in an infinite square well of width
$L$ in a quantum state for which $n = 2$. (a) In the vicinity of what position(s) is
the particle most likely to be found within the well?  (b) What is the
probability of finding the particle between positions $x = L/4$ and $x
= L/2$?

{\bf Solution: (a)} The wavefunction for this particle is shown as
$\psi_2(x)$ in Fig.~\ref{fig:infinite_sq_well_solutions}.  Using
Eq.~(\ref{eq:squarewell_7}), the probability density for this
wavefunction is
\begin{equation}
|\psi_2(x)|^2 = \left(\sqrt{\frac{2}{L}} \right)^2 
                      \sin^2\left(\frac{2 \pi}{L}x\right) ,
\end{equation}

\begin{figure}[!tb]
\begin{center}
\includegraphics[width=3in]{quantized_energies/ProbState_2}
\end{center}
\caption{Probability density for a particle in the $n=2$ state of an infinite square well.}
\label{fig:ProbState_2}
\end{figure}

\noindent as shown in Fig.~\ref{fig:ProbState_2}. The positions
near which the particle is most likely to be found are where the
probability density is the greatest. By examining the graph of
probability density, these positions are located at $x = L/4$ and $x =
3L/4$.  {\bf (b)} To calculate the probability of finding the particle
between $x = L/4$ and $x = L/2$, we use 
Eq.~(\ref{eq:IntegralprobDensity}):
\begin{eqnarray}
\mbox{Prob}\left[\frac{L}{4}, \frac{L}{2}\right] 
   &=& \int_{L/4}^{L/2} |\psi_2(x)|^2\, dx \nonumber \\
   &=& \int_{L/4}^{L/2} \left(\frac{2}{L}\right) 
              \sin^2\left(\frac{2 \pi}{L}x\right)\, dx \nonumber \\
   &=& \frac{2}{L} \left[\frac{x}{2} 
       - \frac{L}{8 \pi}\sin \left(\frac{4 \pi}{L}x\right) \right]_{L/4}^{L/2}
           \nonumber \\ 
   &=&  \frac{1}{4} 
\end{eqnarray}
\noindent where the integral can be evaluated using techniques that you
learned in your calculus class or the table of integrals in Appendix A of
your Wolfson text.
%, your calculator (if it can do integrals), or Wolfram Alpha online (www.wolframalpha.com).

Because of the symmetry of the probability density, the area under the
curve of $\left|\psi\right|^2$ between $x = L/4$ and $x = L/2$ is
one-quarter of the total area, therefore it makes sense that the
probability is $\frac{1}{4}$.
\end{example}

% If a particle is in a {\bf three-dimensional} box (i.e., a
% 3D infinite square-well potential) --- as is the case for quantum
% dots --- then the wavefunction is a 3D wave with an integer number
% of half-wavelengths in each of the three directions. The result is an
% energy that depends on integers $n_x$, $n_y$ and $n_z$ that denote the 
% number of half wavelengths in each direction:
% \begin{eqnarray}
% \label{eq:3Dquantumdot}
% E_{n_x,n_y,n_z} = (n_x^2+n_y^2+n_z^2) \frac{\pi^2 \hbar^2}{2mL^2} 
% = (n_x^2+n_y^2+n_z^2) \frac{h^2}{8 m L^2} ,
% end{eqnarray}
% with each of the quantum numbers $n_x, n_y$ or $n_z$ as a 
% positive integer (1, 2, 3, \dots)
% In Chapter~\ref{chapter:3D_and_semiconductors}, we'll say more about 
% wavefunctions in three-dimensional systems.


\section[Semi-infinite Square Well and Tunneling]{Semi-infinite Square Well Potential and Quantum Tunneling}
\label{sec:semi_infinite_sq_well}

If we want to determine the wavefunction solutions for a particle confined
in a region where the potential energy is some known function $U(x)$,
then we perform a procedure similar to what we did for the example of the
infinite square well potential. However, this time the potential is not
zero (or a constant) but changes with position.  Solving Schr\"odinger's
equation for the case of a general potential energy function $U(x)$
can be very complicated and certainly this is beyond the scope of this
course. In fact, there are very few potential energy functions for which
the Schr\"odinger equation can be solved exactly.

However, we can give a slightly more complicated example of solving
Schr\"odinger's equation where the potential energy has different
constant values in two different regions of space, as shown in
Fig.~\ref{fig:semiInfiniteSquareWell}. In this case, the potential
differs from the infinite square well potential in that the boundary
at $x = L$ is now set to the finite value $U(x) = U_0$ for $x > L$
instead of being infinite. If we consider a solution for a total
energy $E$ which is less than the potential energy step $U_0$, then we
see that Schr\"{o}dinger's equation is different in the two regions $(0
< x < L)$ and $(x > L)$.  Rearranging Schr\"{o}dinger's equation, we
obtain

\begin{figure}[!t]
\begin{center}
\includegraphics[width=3.2in]{quantized_energies/semiInfinite_sq_well}
\end{center}
\caption{The semi-infinite square well potential energy function.}
\label{fig:semiInfiniteSquareWell}
\end{figure}

\begin{equation}
\frac{d^2\psi(x)}{dx^2}  = \frac{2m}{\hbar^2}\left(U(x)-E\right)\psi(x).
\end{equation}
We see that in the region $0 < x < L$, Schr\"odinger's equation reduces
again to Eq.~(\ref{eq:squarewell_1}) so we should expect wavefunction
solutions in that region to be similar to sinusoidal functions as
given in Eq.~(\ref{eq:squarewell_3}).  However, in the region $(x >
L)$ the total energy is less than the potential energy ($E < U_0$)
and Schr\"odinger's equation becomes
\begin{equation}
\frac{d^2\psi(x)}{dx^2} =
\frac{2m}{\hbar^2}\left(U_0-E\right)\psi(x) \hspace{0.5in}\mbox{for  $x > L$}.
\label{eq:SchrodingerE<U}
\end{equation}

As we did before, we must determine a mathematical function $\psi(x)$
that is consistent with Eq.~(\ref{eq:SchrodingerE<U}),
remembering that the quantity $\left[U_0 - E\right]$ is a positive
quantity.  This time the second derivative of $\psi(x)$ is equal to a
{\it positive} number times the same function.  In this case,
sinusoidal functions will not work, but again, if we think back to our
calculus course, we recall that another set of functions we know is
the right choice here --- the exponential functions $C e^{+\kappa x}$
and $D e^{-\kappa x}$, where $C, D$, and $\kappa$ are constants,
with $\kappa > 0$.

\begin{example}{Wavefunction solutions for $E < U$.}
\label{ex:SolutionElessU}
  Given the potential shown in Fig.~\ref{fig:semiInfiniteSquareWell},
  show that the function $C e^{\kappa x}$ is a solution to
  Schr\"odinger's equation~(\ref{eq:SchrodingerE<U}) and determine
  what the constant $\kappa$ must be for this to be a solution.

\noindent {\bf Solution:} To see if $\psi(x) = C e^{+\kappa x}$ is a
solution, we first evaluate the left hand side of Eq.~(\ref{eq:SchrodingerE<U})
 by calculating the second derivative of $\psi(x)$:
\begin{eqnarray}
\frac{d \psi(x)}{dx} & = & \kappa  C e^{\kappa x} \nonumber \\
\frac{d^2 \psi(x)}{dx^2} & = & \kappa^2 C e^{\kappa x}. 
\end{eqnarray}

\noindent Inserting this into Schr\"odinger's equation~(\ref{eq:SchrodingerE<U})
we find \begin{equation}
\kappa^2 C e^{\kappa x} 
   = \frac{2m}{\hbar^2}\left( U_0 - E \right) C e^{\kappa x} .
\end{equation}
In order for $\psi(x) = C e^{+\kappa x}$ to be a solution,
the left side of this equation must equal the right side, and
therefore
\begin{equation}
\kappa^2 = \frac{2m}{\hbar^2}\left( U_0 - E \right) .
\end{equation}
\end{example}
\newpage

\begin{exampleb}{Complete wavefunction solution for the semi-infinite square well.}
\label{ex:SolutionSemiInf}
The complete wavefunction for the semi-infinite square well is made
by combining the three solutions to Schr\"odinger's equation in the
three regions, ($x<0$), ($0 <x < L$), and ($ x > L$):
\begin{equation}
\psi(x) =  \left\{\begin{array}{ll} 
              0                       & \mbox{for $x<0$}  \\
              A \sin(kx) + B \cos(kx) & \mbox{for $0<x<L$} \\ 
              C e^{+\kappa x} + D e^{- \kappa x} & \mbox{for $x > L$} 
              \end{array} \right.  .
\end{equation}
As we did in the previous section, we can further simplify this
solution by matching this solution to what the wavefunction should be
at $x = 0$ and as $x \rightarrow \infty$. At $x = 0$, the wavefunction
should be zero so that it matches $\psi(x)=0$ for $(x < 0)$.  This
requires that the constant $B = 0$ so the function $\cos{(kx)}$ does
not appear.  As $x \rightarrow \infty$, the function $e^{+ \kappa x}$
would become infinite and is therefore a not well-behaved solution.
So we choose $C = 0$ to eliminate that function from our solution and
the final solution becomes
\begin{equation}
\label{eq:two_regions_wavefunction}
\psi(x) =  \left\{\begin{array}{ll} 
            0                & \mbox{for $x<0$} \\
            A \sin(kx)       &  \mbox{for $0 < x < L$r} \\  
            D e^{- \kappa x} & \mbox{for $x > L$}
            \end{array} \right. .
\end{equation}
The wavefunctions in the two regions given in
Eq.~(\ref{eq:two_regions_wavefunction}) must match at position $x =
L$ such the the wavefunction and its first derivative make a smooth
transition from the region $x<L$ to the region $x>L$.  The mathematics
is beyond the scope of this course, but there are only certain values
of $k$, and hence $E$, for which the wavefunctions make a smooth
transition. Graphs of the wavefunctions for the two lowest energy states
$\psi_1(x)$ and $\psi_2(x)$ are shown in Fig.~\ref{fig:Semi_Inf_States}.

\begin{figure}
\begin{center}
\includegraphics[width=4.8in]{quantized_energies/Semi_Inf_States}
\end{center}
\caption{The lowest two energy states of a particle in a semi-infinite
  potential well.}
\label{fig:Semi_Inf_States}
\end{figure}
In some of the problems at the end of the chapter you will use an Excel 
spreadsheet to determine solutions to the semi-infinite square well using 
a numerical method.
\end{exampleb}
An examination of Fig.~\ref{fig:Semi_Inf_States} leaves us the
following important conclusions:

\begin{itemize}
\item Wavefunctions in the region where $E > U$ are sinusoidal
  functions similar to those in the infinite square well potential.
  The wavelength depends on the difference $\left|E - U\right|$.
\item Wavefunctions in the region where $E < U$ are exponential functions.
\item In the case of the semi-infinite well, the length of the well $L$
is not exactly an integer or half-integer multiple of the wavelength as in
the infinite square well (see Fig.~\ref{fig:infinite_sq_well_solutions}).
\end{itemize}

{\bf Note also} that in the region where $E < U$ the wavefunction $\psi(x)$
is {\it not zero}. This might not seem at first glance to be all that
surprising until you consider that a particle's kinetic energy $K = E - U$,
and the region where $E < U$ is a classically forbidden region (i.e., a
particle can't ever be found in that region, classically) since classical
mechanics forbids a particle ever from having a negative kinetic energy.
But even though classical mechanics says that a particle can never enter a
region where $E < U$, we can see from this example that quantum theory
predicts a non-zero probability for a particle to be found in this region.

\begin{figure}
\begin{center}
\includegraphics[width=3.0in]{quantized_energies/tunnel_potential}
\end{center}
\caption{For this potential, a particle with energy $E$ in the region
$0 < x < L_1$ would remain trapped forever, according to classical
physics. But quantum theory predicts that the particle can ``tunnel''
through the thin barrier and escape.}
\label{fig:tunneling}
\end{figure}

But what if there is a potential well such as the one shown in
Fig.~\ref{fig:tunneling}? If a particle confined in the region 
$0 < x < L_1$ has an energy $E < U_0$, then according to classical
theory, that particle will remain trapped between $x = 0$ and
$x = L$ forever, since it can never pass through the classically-forbidden
region $L_1 < x < L_2$. But we just saw in our previous example that
Schr\"{o}dinger's Equation predicts a non-zero probability of a particle
being found in a classically-forbidden region, so presumably, there is
a non-zero probability that the particle in Fig.~\ref{fig:tunneling}
can be found in the region $L_1 < x < L_2$. But if it can make it into
that region, then there is a non-zero probability that the particle
can be found {\bf outside} the confined region, i.e. for $x > L_2$!
In other words, a classically-trapped particle can ``tunnel'' out of
the trap in quantum mechanics.

We won't calculate the probabilities for tunneling. But examination of
Fig.~\ref{fig:Semi_Inf_States} indicates
that the probabilities tail off the wider the classically-forbidden region
is. So, the probabilities can become quite small unless the barrier is very
narrow and unless the energy $E$ is close to the barrier height $U_0$.
Despite these limitations, quantum mechanical tunneling can play a very
important role in various processes, and has already found its way into
some useful applications: 
\begin{itemize}
% \item Radioactivity can be considered as a quantum tunneling phenomena.
% The nucleus of an unstable atom can be thought of as having a
% subatomic particle (such as an electron
% or positron) that tunnels out, producing a high energy emission and leaving
% behind an altered nucleus with a different electric charge.
\item Scanning tunneling microscopes (STMs) have been developed that use
tunneling to produce images of surfaces that can resolve individual atoms.
\item Josephson junctions and tunneling diodes have become tools of modern
electronics. 
\item Tunneling is being proposed as a new mechanism for developing
transistors, which could improve the performance of integrated circuits
in the future.
\end{itemize}

\section{Energy Bands}
\label{sec:energy_bands}

Quantized energies are found for particles confined in any system, not
just the infinite or finite square well potentials. Schr\"{o}dinger's
Equation can be solved to determine the allowed energies for particles 
in any potential energy function. (In fact, determining allowed energies
for electrons in atoms is one fundamental aspect of physical 
chemistry.) Each element (and molecule) has its 
own pattern of energy levels. As we'll see later this chapter, this 
has implications for the light that is absorbed and emitted by different
elements and molecules.

But in addition to the case of a single electron trapped in a single 
potential well, there are many systems composed of multiple electrons
trapped in a repeating pattern of potential wells. Consider a chunk
of aluminum in a crystalline pattern, for example. The nuclei of
the aluminum atoms produce a potential energy function that has
repeating energy wells that mobile electrons experience.

An interesting thing happens to the quantized energy levels when there
is more than one side-by-side potential well. For simplicity, we will
consider the case of side-by-side, 1D, finite square-well potentials.
Figure~\ref{fig:PeriodicPotentials}) shows a representation of what
happens to the energy levels when more and more square-well potentials are
added to the system. Each energy level in the single finite-well potential
(Fig.~\ref{fig:PeriodicPotentials}a) splits into two nearby levels for the
double-well potential (Fig.~\ref{fig:PeriodicPotentials}b).  If there are
four side-by-side potential wells (Fig.~\ref{fig:PeriodicPotentials}c),
there are four energy levels clustered around each value. As the number
of side-by-side wells increases, the number of energy levels in each
``band'' increases accordingly.

\begin{figure}[!t]
\begin{center}
\includegraphics[width=4.5in]{quantized_energies/PeriodicPotentials}
\end{center}
\caption{Allowed energy levels for (a) infinite square well potential;
(b) double, side-by-side potential wells, separated by a thin, finite
barriers; and (c) four side-by-side potential wells.}
\label{fig:PeriodicPotentials}
\end{figure}

In a real solid (such as aluminum), there are many, many, many side-by-side
potential wells, so many that it becomes difficult to distinguish the
individual energy levels within each band (Fig.~\ref{fig:bands}). They
are still there --- in the figure, it may look like a continuous band of
allowed energies, but there are still a finite number of allowed
energy levels (although a {\bf very} large number of finite values)
in each energy band. So, even though the energies appear continuous
within each band, they are still quantized.

\begin{figure}[!t]
\begin{center}
\includegraphics[width=2.5in]{quantized_energies/EnergyBands}
\end{center}
\caption{Valence and conduction bands for a typical solid composed of
many atoms. Each of these bands contains a large (but finite) number of
discrete energy levels.}
\label{fig:bands}
\end{figure}

In Fig.~\ref{fig:bands}, we show only two bands of allowed energies,
although there are typically more. The valence band is the highest
energy band that is typically filled with electrons.\footnote{As we will
see in Chapter \ref{chapter:quantum_statistics}, the Pauli Exclusion
Principle argues that two electrons cannot occupy the same state,
so there is a finite number of electrons that can have energies in any
band in a solid.} The conduction band always has available energy levels
that are unoccupied.  In Chapter~\ref{chapter:3D_and_semiconductors},
we'll discuss how the band structure discussed here is important in
understanding electrical conductivity for solids, and how the quantum
properties of these bands have been used to develop some of the most
important building blocks of modern electronics.

\begin{figure}[!t]
\begin{center}
\includegraphics[width=2.5in]{quantized_energies/absorption_emission}
\end{center}
\caption{(a) Absorption of a single photon, causing an electron in
the material to jump up to a higher (unoccupied) energy state. (b) Emission
of a photon by an electron that drops from a higher to lower energy
state.}
\label{fig:absorption_emission}
\end{figure}

\section{Absorption, Emission, and Spectroscopy}
\label{sec:absorption_emission}

Okay, we are now ready to address some questions that you might have asked
when you were a kid: ``Why is a tomato red?" or ``How do glow-in-the-dark
shirts work?'' or ``Why is it that black-light illumination (which looks
kinda deep violet in color) cause certain pigments to glow orange'' or
``Why do certain laundry detergents make my clothes look so unnaturally
bright white?'' These questions all fall under the category of the topic
of photophysics or photochemistry, i.e., the physics and chemistry of
how light interacts with matter. In addition to providing answers to
these questions, this subject also relates to the important topic of
{\it spectroscopy} which is a critical diagnostic technique that spans
all fields of science and engineering.

There are a few different ways in which light (and any
electromagnetic radiation) can interact with matter, but two important
processes are absorption and emission of individual photons of
light. Figure~\ref{fig:absorption_emission} shows these two processes.
If a photon strikes a material, it can be absorbed by an electron,
resulting in an increase in the energy of that electron from an
initial (lower) value $E_1$ to a higher value $E_2$, as shown in
Fig.~\ref{fig:absorption_emission}a. This process must conserve energy,
which means that the energy gained by the electron must be the same as
the energy of the absorbed photon:
\begin{equation}
\label{eq:photon_energy}
E_\text{ph} = \Delta E = |E_\text{final} - E_\text{initial}|
\end{equation} 
Note that the upper energy level state must initially be unoccupied for
this process to work. As we'll see in Chapter \ref{chapter:quantum_statistics},
two electrons cannot occupy the same quantum state.

The process of emission is the same thing in reverse 
(Fig.~\ref{fig:absorption_emission}b). An electron in 
a higher energy state can drop to a lower-energy state (if initially
unoccupied), releasing a single photon with an energy given by
Eq.~(\ref{eq:photon_energy}).

These simple quantum principles now enable us to resolve the third of
the great failures of classical physics --- the observation in the early
1900s that when energy is pumped into atoms, they radiate EM waves only
at certain distinct wavelengths; i.e., the problem of discrete atomic
spectral lines discussed in Chapter \ref{chapter:beyond_classical}. The
explanation is actually quite simple: each element or molecule
has its own discrete, quantized energy level structure (e.g.,
Fig.~\ref{fig:energy_levels}). An electron in an atom can make transitions
between the energy levels.  Any transition from a higher to a lower
energy level will result in the emission of a photon. But since there are
only certain, well-defined, quantized energy levels for the electron,
then there are only certain well-defined {\it differences} $\Delta E$
allowed for the transitions between these levels. And since the photons
emitted by these transitions have energy $E_\text{ph} = |\Delta E|$,
that means that the emitted photons have only certain, well-defined
energies. And since the energy of the photon is related to the frequency
and wavelength of the electromagnetic radiation via $E_\text{ph} = hf =
hc/\lambda$, that means that the light emitted by atoms has only certain,
well-defined, discrete wavelengths.

\begin{figure}
\begin{center}
\includegraphics[width=1.0in]{quantized_energies/energy_levels}
\end{center}
\caption{Example of an atom or molecule with four quantized energy
levels.}
\label{fig:energy_levels}
\end{figure}
\newpage

\begin{example}{Light emitted from a molecule with four energy levels.}
\label{exam:emission}
For a molecule with the four-level system in Fig.~\ref{fig:energy_levels}, 
assume that $E_0 = -11.5\units{eV}$, $E_1=-6.2\units{eV}$, 
$E_2 = -5.4\units{eV}$, and $E_3 = -2.5\units{eV}$. 
(a) Determine how many different wavelengths of EM
radiation can be emitted from this molecule, and calculate (b) the 
largest and (c) the smallest wavelengths of
EM radiation that can be emitted from this molecule.

{\bf Solution:} (a) There are several possible transitions that an 
electron can make that will emit a photon in this system. An electron
can drop from energy level 3 down to level 0, 1 or 2; it can drop
from energy level 2 down to level 0 or 1; and it can drop from
energy level 1 down to level 0. Altogether, there are 6 possible
downward energy transitions, so there are 6 possible wavelengths of
light that can be emitted. 

\begin{figure}[!t]
\begin{center}
\includegraphics[width=2.0in]{quantized_energies/energy_levels2}
\end{center}
\caption{Possible transitions that result in emission of EM radiation
for the 4-level system of Example \ref{exam:emission}.}
\label{fig:energy_levels2}
\end{figure}
For parts (b) and (c), for any emitted photon, the energy 
$E_text{ph} = |\Delta E|$ and since $E_\text{ph} = hc/\lambda$, 
it follows that $hc/\lambda = |\Delta E|$ and
$\lambda = hc/\Delta E$. So, the largest and smallest wavelengths
correspond to the smallest and largest energy differences $\Delta E$.
(b) The smallest energy difference $\Delta E$ given the allowed electron
energies is the energy difference between levels 2 and 1. So, the largest
emitted wavelength is:
\begin{equation}
\lambda = \frac{hc}{\Delta E} 
        = \frac{1240 \units{eV$\cdot$nm}}{|-5.4 \units{eV} 
               - (-6.2 \units{eV})|} 
        = \frac{1240 \units{eV$\cdot$nm}}{0.8 \units{eV}} = 1550 \units{nm}, 
\end{equation}
where we have used the result $hc = 1240\units{eV$\cdot$nm}$, 
which is convenient for 
spectra problems where the energy levels and differences are expressed
in electron volts and the wavelengths of emitted and absorbed light are 
expressed in nanometers. We could have converted everything into SI units
and used typical values for $h$ and $c$, but that is more time-consuming.

Note that the longest wavelength emitted is a value beyond the visible spectrum
(this is in the infrared spectrum), so the human eye wouldn't be able
to see this emitted light. 
(c) For the shortest wavelength for emitted EM radiation, we want the
largest $\Delta E$ possible for these four energy levels, which corresponds
to a transition from energy level 3 down to energy level 0. 

\begin{equation}
\lambda = \frac{hc}{\Delta E} = \frac{1240 \units{eV$\cdot$nm}}{|-2.5 \units{eV} - (-11.5 \units{eV})|} 
= \frac{1240 \units{eV$\cdot$nm}}{9.0 \units{eV}} = 138 \units{nm} \nonumber
\end{equation}
This wavelength is also outside the visible spectrum (it's ultraviolet, in 
this case).
\end{example}

Everything in the previous example works the other way for absorption. 
But instead of an electron dropping to a lower energy level and emitting
a photon, a photon is absorbed and the electron jumps up to a higher
energy level. But this only works if the incoming photon has the right
energy. From a practical perspective, if you shine white light through
a gas of a particle element or molecule, all the light will pass through
{\bf except} for light with particular wavelengths and energies corresponding
to the energy differences $\Delta E$ for the sample.

Since each element and each molecule has its own distinctive energy-level
structure, the spectrum of EM radiation emitted is unique to each, as
is the spectrum of radiation absorbed. This
``fingerprint'' is often used to identify materials in systems as diverse
as a microscopic sample of DNA\footnote{Spectroscopy is often used in
the biological sciences. We have had many biology majors in PHYS 212
in the past who have commented that they see these techniques used quite
often in the biology labs at Bucknell.} up to astronomical-scale objects
such as stars and planets; large, nebular clouds that are collapsing to 
form new stars and planetary systems; and accretion disks of gas spiraling
into supermassive black holes at the centers of ``active'' 
galaxies.\footnote{Spectroscopy is the most important tool for identifying 
materials in distant astronomical objects. It's not as though we can go
collect a sample of gas from a nebula 100,000 light years away. The light
that we see from these objects is the only data that we have in most
cases.}

So, why is a tomato red? Answer: because it contains a pigment called
lycopene that has electronic energy levels with an energy difference
corresponding to the energy of red photons. When an electron in an excited
state in lycopene drops to a lower level, it emits red photons. (Lycopene 
is actually a fluorescent pigment; fluorescence is discussed in the next
section.)

Finally, absorption and emission also work for solids with band structures
similar to that in Fig.~\ref{fig:bands}. There are many, closely-spaced
energy levels within the bands, allowing for EM radiation with very 
large wavelengths to be emitted and absorbed. But it is also possible for
electrons to make transitions between the bands, resulting in larger
energy differences and smaller wavelengths for emitted or absorbed light.
				
\begin{example}{Absorption by a material with a band gap.}
\label{exam:absorption_bands}
A light detector uses a material with a filled valence band (all available
energy states occupied by an electron) and an empty conduction band separated
by a band gap energy of $3.2\units{eV}$. Calculate the largest wavelength of light 
most effectively absorbed by this detector.

{\bf Solution:} 
The energy of the photon $E_\text{ph} = |\Delta E|$
and since $E_\text{ph} = hc/\lambda$, it follows that $hc/\lambda = |\Delta E|$.
The largest wavelength corresponds to the smallest energy
difference $\Delta E$. Since the valence band is filled, the nearest 
unoccupied energy level is at the bottom of the conduction band, so
$\Delta E_\text{min} = 3.2\units{eV}$.
So $\lambda_\text{max} = hc/\Delta E 
            = (1240\units{eV$\cdot$nm})/(3.2 \units{eV}) = 387.5\units{nm}$. 
\end{example}

\section{Fluorescence and Phosphorescence}
\label{sec:fluorescence_phosphorescence}

Many materials can absorb one wavelength of EM radiation and emit EM
radiation with a different wavelength. Uranine dye (sodium fluorescein),
for example, readily absorbs near-UV radiation, but emits green light. The
principle --- which is called ``fluorescence'' --- is quite simple,
and is very common in materials with more than two possible electronic
energy levels.


\begin{figure}
\begin{center}
\includegraphics[width=1.5in]{quantized_energies/fluorescence}
\end{center}
\caption{{\bf Fluorescence: } Energy-level diagram for a 
three-level atom, showing
absorption of a photon with large energy and emission of two
photons with smaller energy. }
\label{fig:fluorescence}
\end{figure}
		
Figure \ref{fig:fluorescence} shows the basic idea for fluorescence.
A large energy photon (e.g., corresponding to ultraviolet light)
is absorbed by a low energy electron, which jumps up more than one
energy level.  The electron can later spontaneously drop back down to
a lower energy level, but in addition to dropping back down directly to
the original state, it can also drop back down to an intermediate state
and then later drop to the original state. For a three-level system
like in Fig.~\ref{fig:fluorescence}, there are three different energies
possible for the emitted light: $E_\text{ph} = \Delta E_{31}$, $\Delta E_{32}$,
or $\Delta E_{21}$.

So, why does black light illumination cause certain pigments to glow at
different colors?  Assume that the pigment has an energy level structure
similar to that in Fig.~\ref{fig:fluorescence}. An incoming photon from
the black light --- which is really near-UV radiation with a ``color''
just beyond the visible (although there is usually a little violet light
that comes out of a black light as well) --- has enough energy to cause an
electron in the ground ($n = 0$) state to become excited to the $n = 2$
state. After an undetermined amount of time, the electron spontaneously
drops back down to a lower energy level. If it drops directly back to
the $n = 0$ level, it emits another near-UV photon, which you don't
really see. But if it drops first to the $n = 1$ level and then later
to the $n = 0$ level, it emits two lower-energy photons, one of which
(and maybe both) has energy in the visible part of the EM spectrum.
(For many fluorescent pigments, one photon will be in the visible range,
e.g., orange or green, and the other will be an infrared photon which you
can't see.) So, that's why some materials fluoresce at a different
color than they are illuminated with.

Fluorescent pigments can be found in many applications. Of course,
there are many toys with fluorescent pigments, including Crayola$^{TM}$
crayons (with fluorescent colors such as {\it screamin' green}, {\it
atomic tangerine}, and {\it unmellow yellow}). Fluorescent dyes are also
used in laundry detergents to make clothes appear brighter (that's why
white clothes glow when illuminated by black light).

But fluorescence is also a {\it very} important tool used in scientific
studies.  Fluorescent ``tags'' are often used in biological studies. The
basic idea is that if you can attach a fluorescent marker to a particular
molecule in a biological system, then if you illuminate your sample with
near-UV radiation, only the tagged regions will fluoresce in the visible,
enabling you to see (usually under a microscope) the particular molecules
in question. In recent years, biologists have developed a fluorescent
protein called ``green-fluorescent protein'' (or GFP for short) that can
be attached to various molecules in a living cell. Better yet, techniques
have been developed to create targeted mutations in microorganisms that
result in the GFP tag being an {\it inherited} property of the organism
(i.e., they are born fluorescing.)

Fluorescence is also being used to revolutionize surgery with a new
(as of 2017) technique referred to as ``fluorescence image guided surgery.''
The idea is quite simple: a fluorescent dye that preferentially bonds to
cancerous cells is injected into a patient. During surgery, the region
is illuminated with high-frequency light (possibly near-UV), and the
fluorescent dye attached to the cancerous tumor glows, enabling the
surgeon to cut away the minimum amount of cancerous tissue without
removing healthy tissue. This is a technique that figures to significantly
improve survival rates for cancer patients during the next decade.

Another very new technology is that of the ``quantum dot,'' which
is a micro-engineered particle-in-a-box. We talk a lot about the
particle-in-a-box problem because it is easy to solve, {\it but these
things are actually being made!!} And the nice thing about a quantum dot
is that the energy level structure can be altered simply by changing the
size of the quantum dot (i.e., the width of the square-well potential).

A quantum dot is a a semiconductor that is manufactured to be
{\bf so} small (about 10 -- 50 atoms in diameter) that its charge
carriers experience effects of quantum confinement, including
quantized energy levels. The result is that the filled valence band
and empty conduction band for the semiconductor are replaced by a smaller
number of discrete energy levels, similar to those of the particle-in-a-box.
%as shown in Fig.~\ref{fig:QDot}. 
But since the electrons are confined
(due to the small size of the dot), the minimum conduction band energy
is shifted upward (away from the band) gap by a particle-in-box energy.
Similarly, the maximum valence band energy is shifted downward.
The result is the lowest energy $E_\text{ph}$ of an emitted photon for a 
quantum dot is
\begin{equation}
E_\text{ph}=E_g+\eta\, \frac{h^2}{8m_\text{e}R^2} , 
\label{eq:QDot}
\end{equation}
where $m_\text{e}$ is the electron
mass and $\eta$ is a factor that includes all of the material-specific 
semiconductor effects.\footnote{The constant $\eta$ includes deviations
from the standard particle-in-box energies due to the fact that this box
isn't empty; rather, the electron is moving within a semiconductor.}
Since there are higher, open energy levels an electron in the quantum 
dot can absorb smaller wavelength (higher energy) photons and then 
fluoresce by emitting larger wavelength (smaller energy) photons 
as it drops back down in energy.

\begin{example}{Quantum dot fluorescence.}
A quantum dot made from a CdTe semiconductor (with band gap energy
$1.38\units{eV}$  and $\eta = 20.5$) is fabricated with a radius 
$4.22\units{nm}$.  Calculate the longest wavelength emitted
by this quantum dot.

{\bf Solution:} The longest wavelength corresponds to the smallest
energy (Eq.~(\ref{eq:QDot})). 
Since, $E_\text{ph} = hc/\lambda$, and using the trick employed in
Example~\ref{exam:nanoTube} of multiplying the numerator and 
denominator of the second term  by $c^2$,  it follows that:
\begin{eqnarray}
\frac{hc}{\lambda} &=& E_g+\eta\,  \frac{h^2}{8m_\text{e}R^2} \nonumber \\
              &=& E_g + \eta\, \frac{(hc)^2}{8(m_\text{e}c^2) R^2} \nonumber \\
              &=& 1.38\units{eV} + 20.5\,  \frac{(1240\units{eV$\cdot$nm})^2}
                {8 \times (511\times 10^3\units{eV})\times (4.22\units{nm})^2} 
                   \nonumber \\
              &=& 1.81\units{eV}. 
\end{eqnarray}
Rearranging this gives
\begin{eqnarray}
\lambda &=& \frac{hc}{1.81\units{eV}} \nonumber \\
        &=& \frac{1240\units{eV$\cdot$nm}}{1.81\units{eV}}\nonumber \\
        &=& 684\units{nm}.
\end{eqnarray}
%\lambda &=& \frac{hc}{E_g+\eta\, \frac{(hc)^2}{8(m_\text{e}c^2)R^2}}\nonumber \\
%        &=& \frac{(6.63\times 10^{-34}\units{J$\cdot$s})
%             \cdot(3.0\times 10^8\units{m/s})}
%              {(1.38\units{eV})\cdot (1.6\times 10^{-19}\units{J/eV}) +20.5\frac{(6.63\times 10^{-34}\units{J$\cdot$s})^2} {8\cdot (9.11\times 10^{-31}\units{kg})\cdot (4.22\times 10^{-9}\units{m})^2}} \nonumber \\
%        &=& 6.85\times 10^{-7} \units{m} \nonumber \\
\end{example}

One final question in this chapter: {\it when will an excited electron
drop back down in energy, emitting a photon?} The answer: you can never
know. There is nothing in quantum theory that enables you to
predict the precise moment when the emission process will happen. {\bf However},
it is possible to calculate a probability that an emission process will
occur within a certain time range, similar to how we can calculate
a probability for a particle to be found in a certain region, and it
is also possible to define a {\it half-life} for emission processes; i.e.,
the duration over which (on the average) half of the atoms in an excited
state will have decayed back to the ground state, emitting photons.

The half-life for spontaneous emission can vary dramatically from one
material to another. Usually, the half-life is a very short time; e.g.,
nanoseconds and even picoseconds. But there are some materials that have
unnaturally long half-lives.  In these cases, the material will continue
to emit light. Some composites made from SrAl$_2$O$_4$ have been made
with half-lives of several minutes, which means that an excited sample of
this material will emit light for quite a long time after it is excited.

This is the principle (called {\it phosphorescence}) behind ``glow-in-the-dark''
pigments and materials. You ``charge up'' the material by exposing
it to visible or near-UV light, promoting electrons in the pigment
to excited energy levels.  These electrons then drop back to the
lower energy levels, but over the course of several minutes. During that
time, the pigment appears to ``glow,'' even if there are no lights
around.
%
\newpage


\section*{Problems}
\label{sec:quantized_energies_problems}
\markright{PROBLEMS}

%\vspace{1.5in}


%one

\begin{problem}
\begin{enumerate}
\item Draw a sketch of the wavefunction $\psi(x)$ for the $3^\text{rd}$ 
lowest energy
state for an electron trapped in a one-dimensional, infinite square well 
potential (i.e., the ``particle in a box'').
\item Interpreting your drawing from part (a) as a standing wave pattern, determine the 
wavelength $\lambda$ (in terms of the width $L$ of the box)
of this standing wave. 
\item Substitute the wavefunction $\psi(x) = \sqrt{2/L}\sin(2\pi x/\lambda)$
with your value of $\lambda$ from (b)
into Schr\"{o}dinger's Equation for the infinite square well potential. Verify 
that this is a solution for the
region $0 < x < L$, and determine the energy of the state. Compare your
result with what you get when using the formula from 
Eq.~(\ref{eq:squarewell_6}).
\end{enumerate}
\label{prob:PinBox}
\end{problem}

%two

\begin{problem}
% A \emph{quantum wire} is a conducting structure so thin that quantum
% effects are evident.  Electron energies in a quantum wire are quantized
% and so, therefore, are electrical properties such as resistivity.
% A particular quantum wire is made from carbon nanotubes $1.0 \units{nm}$
% in diameter.  Approximating the structure as a one-dimensional infinite
Find the energies (in eV) of an electron in (a) the ground
state; and (b) the first excited state of a $1.0 \units{nm}$ one-dimensional
infinite square well potential.
\label{prob:QuantumWire}
\end{problem}

%three

\begin{problem}
A particle is in the ground state of an infinite square well between $x = 0$
and $x = L$.  What is the
probability of finding the particle in the region between $x=0$ and $x=
L/3$ ?  ({\bf NOTE:} You may want to make use of the Table of Integrals
in Appendix A of your Wolfson text for this problem!) 
%Alternately, you
%can use Wolfram Alpha {\tt www.wolframalpha.com} to evaluate the integral.)
\end{problem}


%four

\begin{problem}
Sketch the probability density for the $n=2$ state (first excited state)
of an infinite square well
extending from $x=0$ to $x=L$. In the vicinity of what position(s) is 
the particle most likely to be found?
\label{prob:WhereInPInBox}
\end{problem}

% \newpage

%five

\begin{problem}

Let's apply what we have learned about infinite square wells to a
macromolecule confined to a biological cell.  Consider a protein of mass
$250,000 \units{u}$ (where $1\units{u} = 1.661 \times 10^{-27}\units{kg}$)
confined to a $10 \units{$\mu$m}$-diameter cell.  Treating this as a
particle in a one-dimensional square well, find the energy difference
between the ground state and the first excited state.  Given that
biochemical reactions typically involve energies on the order of 
$1\units{eV}$, what so you conclude about the role of quantization in
these reactions?

\end{problem}

\newpage

%Old 4-5, now six

\begin{problem}
  The potential energy for the one-dimensional {\em finite} square
  well is shown in Fig.~\ref{fig:finite_sq_well}, with dotted lines
  representing the energies of the ground state and the first excited
  state.
\begin{figure}[h]
\begin{center}
\includegraphics[width=2.6in]{quantized_energies/finite_sq_well}
\end{center}
\caption{Plot of $U(x)$ versus $x$, for Problem~\ref{prob:finite_sq_well}.}
\label{fig:finite_sq_well}
\end{figure}

\begin{enumerate}
\item Using general principles developed in Examples~\ref{ex:SolutionElessU} and \ref{ex:SolutionSemiInf}, sketch the ground state wavefunction versus position, including regions in which $E< U$.

\item Compared to the ground state wavefunction of the infinite square 
well with the same width, is the wavelength in the classically-allowed
region longer or shorter? Is 
the energy larger or smaller?  

\item Repeat a) and b) for the first excited state.

\item If you did (a) - (c) correctly, you should notice 
that the wavefunction isn't zero for $x < 0$ or for
$x > L$, so there is a non-zero probability that the particle could be
located in these regions. Why is this a violation of classical physics?
(Hint: what can you say about the kinetic energy and speed of the particle
when it is in one of these regions?)
\end{enumerate}
\label{prob:finite_sq_well}
\end{problem}

%\begin{problem}
%  For the potential energy in the top diagram of
%  Fig.~\ref{fig:sawtooth}, the wavefunction of the ground state
%  appears in the bottom diagram. Using the general properties of
%  wavefunctions, sketch the wavefunctions for the first and second
%  excited states. Pay attention to number of nodes, wavelength,
%  inflection points, and concavity.
%\begin{figure}[h]
%\begin{center}
%\includegraphics[width=3.0in]{wavefunctions/sawtooth}
%\end{center}
%\caption{Plot of $U(x)$ versus $x$, and of $\psi_1$ versus $x$, for
%  Problem~\ref{prob:sawtooth}.}
%\label{fig:sawtooth}
%\end{figure}
%\label{prob:sawtooth}
%\end{problem}

%seven

\begin{problem}
The wavefunction solution to the semi-infinite square well was given in
Eq.~(\ref{eq:two_regions_wavefunction}) for the two regions $0 < x < L$ and $x > L$:
\begin{equation}
\psi(x) =  \left\{\begin{array}{ll} 
                 A \sin(kx)       & \mbox{for $0 < x <  L$} \\ 
                 D e^{- \kappa x} & \mbox{for $x > L$} 
                \end{array} \right.
\end{equation}
In order for the wavefunction solution to be continuous and smooth across
the boundary at $x = L$, the value of the wavefunction and its first derivative
for the two solutions must match up at the boundary at $x = L$.

\begin{enumerate}

\item Using the solutions given in
Eq.~(\ref{eq:two_regions_wavefunction}), write an equation that states
that the two wavefunction solutions are equal at the boundary $x = L$.
(Your equation should only contain the symbols $k, \kappa, A, D$ and
numerical constants.)

\item Using the solutions given in
Eq.~(\ref{eq:two_regions_wavefunction}), write an equation that states
that the {\emph derivatives} of the two wavefunction solutions are equal
at the boundary $x = L$.  (Your equation should only contain the symbols
$k, \kappa, A, D$ and numerical constants.)

\item Using the Schr\"{o}dinger equation, it can be shown that the
quantities $k$ and $\kappa$ are related to the energy $E$ of the particle
according to
\begin{equation}
k^2 = \frac{2 m E}{\hbar^2} \hspace{0.5in} 
\mbox{and} \hspace{0.5in} \kappa^2 = \frac{2m \left( U_0 - E \right)}{\hbar^2}
\end{equation}
where $m$ is the mass of the particle and $U_0$ is the height of the
potential well.  Given these relations and the results of parts~(a) and
(b), do you think that the particle can have \emph{any} value of energy
$E$ in the well?  
\end{enumerate} 
\end{problem}

%\begin{problem}
% Consider a superposition of two energy states for the infinite square-well potential.  When states are superposed in this way, the wavefunction for the superposition state is
% \begin{equation}
% \psi(x) = a_1 \psi_1(x) + a_2 \psi_2(x) = a_1 \sqrt{\frac{2}{L}} \sin{\left(\frac{\pi x}{L}\right)} + a_2 \sqrt{\frac{2}{L}} \sin{\left(\frac{2\pi x}{L}\right)}. \nonumber
% \end{equation}
% \begin{enumerate}
% \item Determine the probability density function $|\psi(x)|^2$ for the above superpositions state in terms of $a_1$, $a_2$, and $L$, recognizing that the amplitudes $a_1$ and $a_2$ might be complex numbers.  In other words, calculate $\psi^*\psi$.
%
% \item Now, write down the probability density function $|\psi(x)|^2$ for the following three cases: (i) $a_1 = a_2 = 1/\sqrt{2}$; (ii) $a_1 = 1/\sqrt{2}$ and $a_2 = -1/\sqrt{2}$; and (iii) $a_1 = 1/\sqrt{2}$ and $a_2 = i/\sqrt{2}$.
%
% \item Set $L = 1$ and make graphs of the three functions that you determined in part (b).  You can use Excel or your calculator or whatever you want.  Either print out the graphs or sketch them on your paper.
%
% \item Comment on the graphs.  In particular, we have seen that adding together multiple energy states can localize the particle in the well.  Based on what you see in the graphs, what role do the amplitudes play in this localization process?
% \end{enumerate}
% \end{problem}

%eight

\begin{problem}
 {\bf Semi-infinite square-well potential.} Download the Excel worksheet
\verb+semi-finite.xls+ from either the {\it Handouts} page or from the
Calendar page.  This sheet shows the calculations for determining the
wavefunctions for a potential well that is infinite at $x=0$ but of 
finite magnitude on the right side of the well (which is at $x=5$ in
this problem).  You'll see two
graphs: the top one shows the semi-infinite potential well (in
purple) along with a non-normalized plot of the calculated
wavefunction so you can see it along with the potential.  The bottom
graph shows the normalized wavefunction, corresponding to the
second-to-last column in the worksheet.  

When you bring up the worksheet, the energy will be set for the
value for the ground state.  Some questions:

\begin{enumerate}

\item Sketch or print out (just the first page!) the wavefunctions 
that are displayed for the ground state along with at least two 
of the excited states.  To display
the 1$^{\rm st}$ and $2^{\rm nd}$ excited states, type in 0.64282 and 
1.4144 respectively in the framed box for energy.  

\item What happens if you type in an energy that {\bf isn't} one
of the well-defined energies for the problem?  Try it out, and
comment on what happens.  Had we not told you what the allowed
energies were, how might you figure them out?  (You'll be doing
this in lab later this semester.)

\end{enumerate}
\end{problem}

\newpage

%nine

\begin{problem}
For the potential energy function $U(x)$ and total mechanical energy $E$
in Fig.~\ref{fig:tunneling}:
\begin{enumerate}
\item Use classical physics to argue that if a particle starts out
in the region $0 < x < L_1$, it will remain there forever.

\item Use quantum arguments (specifically, refer to wavefunction) to
argue that a particle that starts in the region $0 < x < L_1$ can escape
(tunnel). Can you predict {\it the precise moment} when the particle will 
escape? (Hint: no.)
\end{enumerate}
\end{problem}

%ten


\begin{problem} 
Given the solution to Schr\"{o}dinger's Equation for an electron in the 
semi-infinite square well potential (see Examples \ref{ex:SolutionElessU}
and  \ref{ex:SolutionSemiInf}).
Assume that $U_0 = 12\units{eV}$ and the electron is in a state with 
an energy $E = 10\units{eV}$.
\begin{enumerate}
\item Calculate the value of the constant $\kappa$ in the exponent
$e^{-\kappa x}$ for the wavefunction
in the classically-forbidden region.
\item Calculate the distance into the classically-forbidden region 
beyond $x = L$ where the probability density is a factor of
2 smaller than that at $x = L$ (i.e., calculate the value of $d$
such that\\ $|\psi(L+d)|^2/|\psi(L)|^2 = e^{-2\kappa d} = 0.5$.
% \item Repeat part (b) if the electron is in a state with energy 
% $E = 8$ eV.
\item Now, let's assume that the particle is a ball with mass 
$0.5\units{kg}$, and assume that  $U_0$ and $E$ have everyday values 
of, say, $100\units{J}$ and $50\units{J}$, 
respectively. Repeat the calculations from parts (a) and (b).
\item Now consider the potential energy function in Fig.~\ref{fig:tunneling}.
What do you think is needed to get a reasonable (not ridiculously small)
probability for the particle to tunnel through the barrier?
Why do you think we never experience quantum tunneling for
everyday objects?
\end{enumerate}
\label{prob:TunnelSemiInfinite}
\end{problem}

%eleven

\begin{problem}
  Consider a proton ($m_p = 1.67 \times 10^{-27}\units{kg}$) confined
  in an infinite potential well of size $L =
  10^{-14}\units{m}$. Calculate the energy of a photon emitted when
  the proton makes a transition from the $n=4$ state to the $n=3$
  state.
\end{problem}

%twelve

\begin{problem}
Electrons in an ensemble of $10\units{nm}$ wide one-dimensional 
infinite square-well potentials are all initially in the 
$n=4$ state.  Find the wavelengths of all possible photons emitted 
as electrons make transitions to the ground state.
\label{prob:EmittedWavelengths}
\end{problem}

%thirteen

\begin{problem}
Calculate the radius of a quantum dot made from CdSe (band gap energy
$1.74\units{eV}$ and $\eta = 16.5$) so that the largest emitted wavelength
will be $629\units{nm}$.
\label{prob:QuantumDot}
\end{problem}

\newpage

%fourteen

\begin{problem}
\begin{enumerate}
\item Ozone (O$_3$) in the atmosphere absorbs ultraviolet radiation
that can damage the skin and cause cancer. It does this via a photochemistry
process $\mbox{O}_3 + \gamma \rightarrow \mbox{O}_2 + \mbox{O}$, 
where $\gamma$ is a photon.  If the dissociation energy of ozone is 
$3.94\units{eV}$, calculate the maximum wavelength of UV radiation
that is absorbed by this photochemical process.
\item Diatomic oxygen (O$_2$) also absorbs UV radiation with smaller
wavelengths via the process $\mbox{O}_2 + \gamma \rightarrow 
\mbox{O} + \mbox{O}$.
Given a dissociation energy of $5.17\units{eV}$ for an O$_2$ molecule,
calculate the maximum wavelength of UV radiation that is absorbed by
this process.
\end{enumerate}
\label{prob:UVAbsorption}
\end{problem}

%fifteen

\begin{problem}
Titanium dioxide is one of many different possible ingredients used in
sunblock lotions. It has a band gap energy of $3.2\units{eV}$. Calculate the
maximum wavelength of radiation that you would expect TiO$_2$ to absorb
via an absorption process that promotes electrons from the valence
to the conduction band. Is your answer consistent with the use of TiO$_2$
as an ingredient in sunblock lotions?
\end{problem}

%sixteen

\begin{problem}
Carbon dioxide has first and second excited vibrational states which 
are $0.083\units{eV}$
and $0.166\units{eV}$ above the ground state. Calculate the wavelengths of
electromagnetic radiation that you would expect to be most readily
absorbed by transitions between these vibrational energy levels in CO$_2$.

The wavelengths that you will find are in the infrared range of
$700\units{nm}$ to $1\units{mm}$. This is one of the reasons 
why carbon dioxide contributes
to global warming as IR radiation is one of the ways in which the Earth
radiates heat, assuming it isn't absorbed in the atmosphere. (Note, though,
that this is a simplification, as there are rotational energy levels
in CO$_2$ as well that aren't included in this problem.)
\label{prob:CO2}
\end{problem}

%seventeen

\begin{problem}
Class 3 (neutral phenol) Green Fluorescence Protein (GFP) used in cell
and molecular biology studies has an excitation wavelength of 
$399\units{nm}$
(near-UV) and an emission wavelength of $511\units{nm}$ (green, duh).
\begin{enumerate}
\item Determine a 3-level energy diagram (with ground state energy
zero) that is consistent with these data. (There are two answers
that are consistent with the given data.)
\item Based on your diagram, you might expect to see two different
colors of emitted light. Why do you see only green light emitted
from a sample tagged (or genetically mutated) with Class 3 GFP?
\end{enumerate}
\end{problem}

\newpage

%eighteen

\begin{problem}
The spec sheets for Lumidot$^{TM}$ quantum dots from Sigma-Aldrich
scientific supply company specifies the following core sizes (radius
of the quantum dots) and wavelengths for their various dots.
\begin{center}
\begin{tabular}{c | c}
Radius (nm) & Wavelength (nm) \\
\hline
3.0 & 510 \\
3.3 & 530 \\
3.4 & 560 \\
4.0 & 590 \\
5.2 & 610 \\
6.3 & 640 \\
\end{tabular}
\end{center}

\noindent Use this data to verify the $1/R^2$ dependence 
of Eq.~(\ref{eq:QDot}) and to find the band gap energy. (Hint: Calculate 
$E_\text{ph}$ for each of these wavelengths, and then plot $E_\text{ph}$ 
versus $1/R^2$; a computer can make the calculations, plotting, and 
fitting easier.)
\end{problem}
