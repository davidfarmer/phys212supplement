\chapter*{Answers to Selected Problems}
\label{chap:selected_answers}
\addcontentsline{toc}{chapter}{Answers to Selected Problems}
%\markboth{ANSWERS}{ANSWERS}

\noindent {\bf Additional Problems}

\noindent {\bf A\ref{prob:addEfields}}~(a)~$-5560\, \hat \jmath \units{N/C}$; 
(b)~$0$; (c)~$5560\, \hat \jmath \units{N/C}$.\\
{\bf A\ref{prob:integ1}}~(a)~$k\lambda / a$; (b)~$-k\lambda / a$.\\
{\bf A\ref{prob:arc1}}~$\frac{-2qk}{\pi R^2}(\hat\imath + \hat\jmath)$.\\
{\bf A\ref{prob:ComputerMonitors}}~(a)~$50,000\units{eV}$; 
(b)~$8.0 \times 10^{-15}\units{J}$;
(c)~$1.33 \times 10^8\units{m/s}$.\\
{\bf A\ref{prob:plates}}~left side:~$1600\units{V/m}$ to left; 
right side:~$300\units{V/m}$ to right.\\
{\bf A\ref{prob:railgun}}~$v_f = \sqrt{2IHBD/m}$.\\
{\bf A\ref{prob:PowerCords}}~$4 \times 10^{-5}\units{T}$.\\
{\bf A\ref{prob:solenoid}}~$5.8 \times 10^{-3}\units{T}$.\\
{\bf A\ref{prob:ElecGen}}~$0.013\units{s}$.\\
{\bf A\ref{prob:faraday}}~$\pi r^2 B_0(6t-1)/R$.\\
{\bf A\ref{prob:reradiation}}~(a)~$\pm z$, (b)~$\pm x$, (c)~$\pm z$,
(d)~$\pm z$, (e)~$\pm x$, $\pm y$.\\
{\bf A\ref{prob:OrganPipes}}~(a)~$20\units{m}$, $10\units{m}$, $6.67\units{m}$; 
(b)~$\pi /10,\ \pi /5,\ 3\pi /10$; 
(c)~$17\units{Hz}$, $34\units{Hz}$, $51\units{Hz}$.\\
{\bf A\ref{prob:dvds}}~(a)~$717\units{nm}$; (b)~$55,800$.\\
{\bf A\ref{prob:complex_traveling_wave}}~(a)~$\lambda = 1257\units{nm}$, 
$T = 4.19\times 10^{-15}\units{s}$, $v = 3.0\times 10^8\units{m/s}$, infrared,
(b)~$B = 0.2\units{mT}$;\\
%{\bf A\ref{prob:dice}}~c)~probabilities from 2 through 12 are: $\frac{1}{36}$,
%$\frac{1}{18}$,$\frac{1}{12}$,$\frac{1}{9}$,$\frac{5}{36}$,$\frac{1}{6}$,
%$\frac{5}{36}$,$\frac{1}{9}$,$\frac{1}{12}$,$\frac{1}{18}$, and $\frac{1}{36}$.\\
%{\bf A\ref{prob:parking_lot_1}}~a)~$1/600\units{m$^{-1}$}$; b)~$1/6$; 
%c)~$P_1 = 3/1000\units{m$^{-1}$}$, $P_2 = 1/1000\units{m$^{-1}$}$; d)~$0.8$.\\
{\bf A\ref{prob:proton_in_box}}~(b)~$2 \times 10^{-14}\units{m}$;
(c)~$p = 3.3 \times 10^{-20}\units{kg$\cdot$m/s}$, 
$K = .3.3 \times 10^{-13}\units{J}$.\\
{\bf A\ref{prob:moreuncertainty}}~$1.3 \times 10^{-21} \units{m/s}$.\\
{\bf A\ref{prob:tunneltime}}~(a)~$\Delta t = 6.6 \times 10^{-18}\units{s}$; 
(b)~$v = 4.2 \times 10^6\units{m/s}$; (c)~$L = 2.8 \times 10^{-11}\units{m}$;
(d)~$0.13$.\\
{\bf A\ref{prob:spinflip}}~$5.4\units{cm}$. \\
{\bf A\ref{prob:nmr}}~$19.4\units{mT}$. \\
{\bf A\ref{prob:TwoAntennas}}~$1.41A$.\\
{\bf A\ref{prob:speakers}}~(a)~minimum; (b)~$9I_0$.\\
{\bf A\ref{prob:wave_to_eq}}~$y(x,t) = 2\cos{(\pi x - 2\pi t)}$.\\
{\bf A\ref{prob:wave_to_phasor_2}}~(b)~$2.95$; (c)~$0.5\units{rad}$.\\
{\bf A\ref{prob:parking_lot_3}}~(a)~$\frac{1}{1600\pi}\units{m$^{-2}$}$; 
(b)~$1/4$; (c)~$P_A = \frac{1}{1000\pi}\units{m$^{-2}$}$, 
$P_B = \frac{1}{2000\pi}\units{m$^{-2}$}$; (d)~$13/20$.\\
{\bf A\ref{prob:ComplexPractice}}~(a)~$1 + i\theta - \theta^2/2! -i\theta^3/3! 
+ \theta^4/4! + i\theta^5/5! - \theta^6/6! + \dots$;
(c)~$1 - i\theta - \theta^2/2! + i\theta^3/3! 
+ \theta^4/4! - i\theta^5/5! - \theta^6/6! + \dots$;
(g)~$0.88 + 0.48i$; (h)~$0.88 -0.48i$; (i)~$1.76$; (j)~$0.96i$;
(k)~$\left(e^{0.3i} + e^{-0.3i}\right)/2$;  
(l)~$\left(e^{0.3i} - e^{-0.3i}\right)/2i$; (m)~$1$; (n)~$e^{0.6i}$.\\
{\bf A\ref{prob:timedependence_1}}~(a)~$11E_1/5$; 
(b)~$\sqrt{3/5}\ e^{-iE_1t/\hbar}|1\rangle  + \sqrt{2/5}\ e^{-i4E_1t/\hbar}|2\rangle $;
(c)~$11E_1/5$.\\
{\bf A\ref{prob:qm_precess_1}}~(a)~$|\Psi(0)\rangle  = |\mbox{$+$}x\rangle  
= \sqrt{1/2}\left(|\mbox{$+$}z\rangle  + |\mbox{$-$}z\rangle \right)$; (b)~$E_+ = -\mu B_0$,$E_- = +\mu B_0$;
(c)~$|\Psi(t)\rangle  = \sqrt{1/2}\left(e^{i\omega t/2}|\mbox{$+$}z\rangle  + e^{-i\omega t/2}|\mbox{$-$}z\rangle\right)$;
(d)~$\left(1 -\sin{\left(\omega t\right)}\right)/2$;
(e)~$\left(1 +\sin{\left(\omega t\right)}\right)/2$.\\
%{\bf A\ref{prob:time_dependent_wavefunctions}}~a)~Let $\Delta E = E_2 - E_1$. 
%Then $|\Psi(x,t)|^2 = [\sin^2(\pi x/L) + \sin(2\pi x/L)^2 + 
%2\cos(\Delta E t/\hbar) \sin(\pi x/L)  \sin(2\pi x/L)]/(2L)$; 
%b)~First define 
%$\omega = \Delta E/\hbar$, and the period $T = 2\pi /\omega$. 
%Then compare to Supp CH~\ref{chapter:states} Problem \ref{prob:particle_in_a_box_i}(b).
%At $t=0$, the result is the same as in part (i),
%at $t=T/4$, the result is the same as in  part (iii),
%at $t=T/2$, the result is the same as in  part (ii), and
%at $t=3T/4$, the result is again the same as in part (iii).\\
{\bf A\ref{prob:phasors_and_waveforms}}~(b)~$2$.\\
{\bf A\ref{prob:Feynman_fun}}~(a)~$e^+$, $W^+$; (b)~$e^+$, $e^+$; (c)~$Ru$, $Gd$.\\
{\bf A\ref{prob:two_wires}}~$4.3\times 10^{-6}\units{T}$.\\
{\bf A\ref{prob:ampere_wire}}~(a)~$0.1\units{mT}$; (b) $0.2\units{mT}$; (c)~$0.1\units{mT}$. 


\medskip

\noindent{\bf Chapter \ref{chapter:phasors}}

\noindent{\bf \ref{prob:waterwaves}.}~(b)~$x(t) = 30 e^{i(\frac{\pi}{3}t + \pi/2)}\units{cm}$.\\
\noindent{\bf \ref{prob:between_towers}.}~$3\sqrt{5}$.\\
\noindent{\bf \ref{prob:three_towers}.}~$1.00 A$.\\
\noindent{\bf \ref{prob:four_slits}.}~(b)~$\theta = 0.45^\circ$, $0.90^\circ$, $1.35^\circ$.

\medskip
%\newpage

\noindent{\bf Chapter \ref{chapter:beyond_classical}}

\noindent{\bf \ref{prob:photonenergy}.}~$E_{ph} = 2.48\units{eV}$, $K = 6.0 \times 10^{-6}\units{eV}$.\\
{\bf \ref{prob:cutoff_freq}.}~(a)~$f_c = 5.4 \times 10^{14}\units{Hz}$.\\
{\bf \ref{prob:T4virus}.}~(a)~$10^{-9} \units{m}$ or smaller, $E_{ph} = 1240 \units{eV}$; (b)~$10^{-9}\units{m}$ or smaller,
$K_{electron} = 1.5\units{eV}$.\\
{\bf \ref{prob:deBmomentum}.}~$p = 4140\units{eV/c}$ or $2.21 \times 10^{-24}\units{kg m/s}$.\\
{\bf \ref{prob:BindingEnergy}.}~$1.1\units{eV}$.\\
{\bf \ref{prob:EMFhazards}.}~(a)~$E_{ph} = 2.48 \times 10^{-13}\units{eV}$; (b)~$E_{ph} \ll U_{bind}$, so no chemical
bonds affected, including those in DNA.\\
{\bf \ref{prob:largest_mode}.}~(a)~$\frac{hc}{2L}j$; (b)~$\frac{Lk_BT}{hc}\frac{1}{j}$; (c)~$j_{max}=\frac{Lk_BT}{hc}$.\\
{\bf \ref{prob:IonizeHAtom}.}~$K_{max} = 7.1\units{eV}$.

\medskip

\noindent{\bf Chapter \ref{chapter:uncertainty}}

%\noindent{\bf \ref{prob:GausWavefunction}.}~(a)~$x = 0$; (b)~$x=\pm a\sqrt{ln2/2} = \pm 0.59a$.\\
\noindent{\bf \ref{prob:WavefunctionProbs}.}~(a) 0.42; (b) 0.27.\\
{\bf \ref{prob:UncertaintyHelectron}.}~(a)~$1.06 \times 10^{-24}\units{kg m/s}$; (b) $1.15 \times 10^6\units{m/s}$.\\
{\bf \ref{prob:UncertaintyBlowDart}.}~$2.1\times 10^{-26}\units{m/s}$.\\
{\bf \ref{prob:NanoscaleUncertainty}.}~(a)~$6.6\times 10^{-13}\units{m}$; (b)~$2.3\times 10^{-12}\units{m}$;
(c)~(Hint: do you think the binding energy for a proton in a nucleus is 1 eV?)\\
{\bf \ref{prob:MinK}.}~(a)~$6.1\times 10^{-19}\units{J}$ or $3.8\units{eV}$; (b)~$1.5\times 10^{-33}\units{J}$ or $9.6\times 10^{-15}\units{eV}$;
(c)~$6.9\times 10^{-61}\units{J}$ or $4.3\times 10^{-42}\units{eV}$.\\
{\bf \ref{prob:SubstitutionPractice}.}~(a)~Works if $A = 5/6$ and $B = 3$; (b)~Doesn't work for all values of $x$.\\
{\bf \ref{prob:TestSchrod}.}~(b)~Doesn't work; (c)~Works if $k = \pm \sqrt{2m(E-U_0)}/\hbar$, which is fine since $E > U_0$;
(d)~Would work if $\kappa = \pm \sqrt{2m(U_0-E)}/\hbar$, but $E > U_0$, so this would be an imaginary $\kappa$.\\
{\bf \ref{prob:harmonic_oscillator}.}~$E = 2$ for a solution.\\

 \medskip

\noindent{\bf Chapter \ref{chapter:quantized_energies}}

\noindent{\bf \ref{prob:PinBox}.}~(b)~$2L/3$; (c)~$E=\frac{9h^2}{8mL^2}$.\\
{\bf \ref{prob:QuantumWire}.}~(a)~$0.38\units{eV}$; (b)~$1.51 \units{eV}$.\\
{\bf \ref{prob:WhereInPInBox}.}~$L/4$ and $3L/4$.\\
{\bf \ref{prob:TunnelSemiInfinite}.}~(a)~$7.2\times 10^9 $m$^{-1}$; (b)~$4.8\times 10^{-11}\units{m}$;
(c)~$5.2\times 10^{-36}\units{m}$.\\
{\bf \ref{prob:EmittedWavelengths}.}~$47.2\units{$\mu$m}$, 
$27.5\units{$\mu$m}$, $22.0\units{$\mu$m}$, $66.0\units{$\mu$m}$, 
$41.2\units{$\mu$m}$, $110\units{$\mu$m}$.\\
{\bf \ref{prob:QuantumDot}.}~$5.18\units{nm}$.\\
{\bf \ref{prob:UVAbsorption}.}~(a)~$315\units{nm}$; (b)~$240\units{nm}$.\\
{\bf \ref{prob:CO2}.}~$15\units{$\mu$m}$ and $7.5\units{$\mu$m}$.

\medskip

\noindent{\bf Chapter \ref{chapter:spin}}

\noindent{\bf \ref{prob:SuperpositionProbs}.}~(a)~$\frac{1}{2}$; (b)~$0$; 
(c)~$\frac{1}{2}$.\\
% {\bf \ref{prob:spin_ii}.}~(a)~$0.98$ and $0.02$; (b)~$0.50$ and $0.50$.\\
{\bf \ref{prob:spin_iv}.}~(a)~$\frac{1}{5}$; (b)~$\frac{1}{10}$; (c) $\frac{1}{2}$; (d) 0.\\
{\bf \ref{prob:SuperpositionProbs2}.}~(a)~$\frac{1}{2}$; (b)~$\frac{1}{2}$.\\
{\bf \ref{prob:PhotonProbs}.}~(b)~$\cos^2\theta$, $\vert\mbox{X}\rangle$; 
(c)~$\cos^2\theta$, $\vert\mbox{$\theta$}\rangle$; 
(d)~$\sin^2\theta$; 
(e)~$\cos^2\theta\sin^2\theta$.\\
{\bf \ref{prob:ElectronSpinState}.}~(a)~$0.75$; (b)~$0.067$.\\
{\bf \ref{prob:ElectronSpinState2}.}~$0.50$.\\
% {\bf \ref{prob:ChemicalShift}.}~$9.41\units{T}$.\\
{\bf \ref{prob:H_hyperfine}.}~$21\units{cm}$.\\
%{\bf \ref{prob:ChemicalShift2}.}~$3.95\times 10^{-5}\units{T}$.\\
{\bf \ref{prob:energystates}.}~(a)~$\frac{13}{36}$; (b)~0; (c)~There are
numerous answers to this question.  The easiest are 
$\frac{\sqrt{2}}{2}$ or $-\frac{\sqrt{2}}{2}$ or $i \frac{\sqrt{2}}{2}$ or
$-i \frac{\sqrt{2}}{2}$. There are also numerous $a+i b$ combinations that
would work, as long as $|a+i b|^2 = \frac{1}{2}$.\\
{\bf \ref{prob:SG1}.}~(a) Two equal intensity beams (1/4 the intensity of the
intensity of the initial electron beam) will emerge, one with $S_z =
+\hbar/2$ and the other with $S_z = -\hbar/2$. (b) A beam will emerge 
from just one of the two exits with $S_x = -\hbar/2$.  This beam will be
1/2 the intensity of the initial electron beam.

\medskip

\noindent{\bf Chapter \ref{chapter:quantum_statistics}}

\noindent{\bf 1.}~(a) electrons are indistinguishable, 
(b)~$\frac{1}{\sqrt 2}\ket{E_1\uparrow\; E_1\downarrow} 
-\frac{1}{\sqrt 2}\ket{E_1\downarrow\; E_1\uparrow}$.\\
\noindent{\bf 2.}~Yes for both (a) and (b).  Electrons and muons are 
distinguishable.\\
\noindent{\bf 3.}~(b)~4 for classical, 3 for bosons, 1 for fermions; 
(c)~$\frac{1}{2}$ for classical, $\frac{2}{3}$ for bosons.\\
\noindent{\bf 4.}~$84\units{eV}$.\\
\noindent{\bf 8.}~$^3$He is a fermion.\\
\noindent{\bf 10.}~0 (that's the point!).\\

\medskip

\noindent{\bf Chapter \ref{chapter:3D_and_semiconductors}}

\noindent{\bf 3.}~$6.84 \times 10^{-34}\units{J$\cdot$s}$.\\
\noindent{\bf 4.}~$-1.51\units{eV}$.\\
\noindent{\bf 7.}~(a)$~2.8\times 10^{-18}$, $0.017$ and $0.40$, 
(b)~$0.026\units{eV}$, $0.26\units{eV}$ and $2.6\units{eV}$;
(c)~$0$, $0$, and $7$.\\
\noindent{\bf 9.}~Would expect poor conductivity at $300\units{K}$, 
moderate conductivity at $3,000\units{K}$, and
excellent conductivity at $30,000\units{K}$.\\
\noindent{\bf 11.}~(a)~$867\units{nm}$ (near-IR); 
(b)~$549\units{nm}$ (yellowish green); (c)~$365\units{nm}$ (near-UV).\\
\noindent{\bf 15.}~(b)$E = -\frac{mk^2e^4}{2\hbar^2}$.

\medskip

\noindent{\bf Chapter \ref{chapter:quantum_entanglement}}

\noindent{\bf 1.}~(a)~3/10, (b)~7/10.\\
\noindent{\bf 3.}~(a)~$c_+=\sqrt{3/10}$, $c_-=\sqrt{7/10}$,
$\ket{\phi_1} = \sqrt{1/3}\ket{\uparrow} + \sqrt{2/3}\ket{\downarrow}$,
$\ket{\phi_2} = \sqrt{2/7}\ket{\uparrow} + \sqrt{5/7}\ket{\downarrow}$.\\
\noindent{\bf 4.}~(a)~0.55; 
(b)~$\ket{\psi_\text{new}} = 1.0\ket{\uparrow}\ket{\phi_1}$;
(c)~$\frac{1}{3}$.\\
\noindent{\bf 7.}~0.854.\\
\noindent{\bf 9.}~(a)~0.75, (b)~0.67.\\

\medskip

%\newpage

\noindent{\bf Chapter \ref{chapter:particles}}

\noindent{\bf \ref{prob:mesons_vs_baryons}.}~Mesons are bosons, baryons are fermions.\\
{\bf \ref{prob:identify_particle_x}.}~(a)~$K^+$, (b)~$\nu_e$, (c)~$K^+$.\\
{\bf \ref{prob:strangeness_and_strong_force}.}~(a)~yes, (b)~no, (c)~no, (d)~yes.\\
{\bf \ref{prob:neutron_decay}.}~Does not conserve charge.\\
{\bf \ref{prob:Xi0_strangeness}.}~$S=-2$.\\
{\bf \ref{prob:quark_constituents}.}~(a)~$uds$, (b)~$dss$, (c)~$u\bar d$.\\
{\bf \ref{prob:quark_model}.}~This requires two $s$-quarks, with total charge $-2/3$.  No single quark can add $q=5/3$.\\
{\bf \ref{prob:identify_from_quarks}.}~(a)~$n$ or $\Delta ^0$, (b)~$\Sigma^+$ or $\Sigma^{*+}$, (c)~$K^0$.

\medskip

\noindent{\bf Chapter \ref{chapter:quarks}}

\noindent{\bf \ref{prob:pair_production}.}~(a)~1.022\units{MeV},
(b)~$6.5\times 10^{-22}\units{s}$.\\
% {\bf \ref{prob:Z_boson}.}~(a)~$96\units{GeV}$, 
% (b)~$7\times 10^{-27}\units{s}$, (c)~$2.1\times 10^{-18}\units{m}$.\\
{\bf \ref{prob:Z_boson}.}~(a)~$91.2\units{GeV}$, 
(b)~$7\times 10^{-27}\units{s}$, (c)~$2\times 10^{-18}\units{m}$.\\
{\bf \ref{prob:quark_color}.}~antigreen.

\medskip

\noindent{\bf Chapter \ref{chapter:interactions}}

\noindent{\bf \ref{prob:proton_decay}.}~About two protons should decay.\\
{\bf \ref{prob:Sigma_minus_decay}.}~Only $\Sigma^-\to n + \pi^-$.\\
{\bf \ref{prob:identify_interaction}.}~(a)~strong, (b)~electromagnetic, 
  (c)~weak.\\
{\bf \ref{prob:Omega_minus}.}~(a)~There are no lighter baryons with $S=-3$, (b)~weak; strangeness is not conserved, (c)~weak interaction
is slower than strong or electromagnetic.\\
{\bf \ref{prob:Xi_minus}.}~$\Xi^-(1535)$ decays much faster; it can 
  decay by the strong interaction; $\Xi^-(1535)\to \Xi^-(1322) +
  \pi^0$, while the lighter $\Xi^-$ must go by a weak interaction.\\
  {\bf \ref{prob:pion_decay}.}~Photons or leptons.  Because the weak 
  decay to leptons is much slower than the electromagnetic decay to photons.

\medskip

\noindent{\bf Chapter \ref{chapter:cosmology}}

\noindent{\bf \ref{prob:Lambda_pair_production}.}~about $10^{-7}$ to $10^{-6}\units{s}$ \\
{\bf \ref{prob:quark_confinement}.}~about $100\units{MeV}$, pions \\
% {\bf \ref{prob:transparent_universe}.}~between 320 and 32,000 years \\
{\bf \ref{prob:X_asymmetry}.}~a)~5730~$q$, 5640~$\bar q$, 330~$l$,
300~$\bar l$, b)~1910 baryons, 1880 antibaryons, c)~30 baryons, 30 leptons,
2180 photons, d)~72.7
