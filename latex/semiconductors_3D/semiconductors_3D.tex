\chapter[3D Wavefunctions and Semiconductors]{Three-Dimensional Wavefunctions and Semiconductors}
% \chapter[3D Wavefunctions and Time Dependence]{Three-Dimensional Wavefunctions and the Time Dependence of Wavefunctions}
%\label{chapter:wavefunctions3D}
\label{chapter:3D_and_semiconductors}
%\setcounter{ex}{0}

\section{Introduction}
\label{sec:3D_and_semiconductors_intro}

\indent The basics of quantum mechanics and its interpretation can be
understood by first looking at one-dimensional problems.  This is what
we have done so far in this unit.  But the real world has three spatial
dimensions, and most real systems have more than one particle to keep
track of.  The extension of the Schr\"{o}dinger equation to three
dimensions is straight-forward, but the math can get a little
complicated.

We will confine our study primarily to the case of electrons in atoms.  
A key feature of wavefunctions in
three dimensions (3D) is that the allowed quantum states are described by
{\bf three} quantum numbers instead of one.  For the states of
electrons in atoms, each quantum number has a distinct physical
interpretation, as we will see.

Also in this chapter we describe the basics of semiconductor physics,
which follows from the discussion of energy bands in Chapter
\ref{chapter:quantized_energies}. We will explain how the conductivity
of materials can be understood in terms of these bands, and we'll
discuss how these quantum properties have led to some really
cool (and tremendously important) techniques that form the basis
of modern electronics.

\section{Quantum in 3D}
\label{sec:3D_quantum}

In earlier chapters we have seen how to use the Schr\"odinger
equation to solve for the wavefunctions of a particle in one
dimension.  In three dimensions, the Schr\"{o}dinger equation
can be written in spherical coordinates $(r, \theta, \phi)$
as:
\begin{equation}
\label{eq:3Dschrodinger}
-\frac{\hbar^2}{2mr^2} \left[ \frac{\partial}{\partial r} \left( r^2
  \frac{\partial \psi}{\partial r} \right) + \frac{1}{\sin{\theta}}
  \frac{\partial}{\partial \theta} \left( \sin{\theta} \frac{\partial
    \psi}{\partial \theta} \right) + \frac{1}{\sin^2{\theta}}
  \frac{\partial^2 \psi}{\partial \phi^2} \right] + U \psi = E \psi.
% \frac{k e^2}{r} \psi = E \psi .
\end{equation}
At this point, you might be hyperventilating, but don't worry --- we
will restrict ourselves to cases where everything is
spherically-symmetric (with no $\theta$- or $\phi$-dependence), so
most of the terms in this equation drop out.  And we won't ask you to
{\it solve} this equation, but will instead use the approach from
earlier where we give you a solution and ask for you to test to see if
it does or does not satisfy the equation.

\begin{example}{Schr\"odinger equation for 3D Harmonic Oscillator}
Show that the test wavefunction $\psi(r) = Ae^{-br^2}$ is a solution
to Eq.~(\ref{eq:3Dschrodinger}) for the harmonic oscillator with
potential energy function $U =\frac{1}{2}kr^2$. If it is a solution
(which it is), determine the energy of the quantum state.

%{\bf Solution}: 
\begin{solution}
If you have never seen a partial derivative such as
$\partial \psi/\partial r$, fear not --- it is the same thing as a
regular derivative where everything else (in this case, $\theta$ and
$\phi$) is treated as constant.  And since the test wavefunction for
this example doesn't have any $\theta$ or $\phi$'s in it, we can
quickly simplify Schr\"{o}dinger's equation by dropping out any terms
that have derivatives with respect to $\theta$ or $\phi$. So, we now
have:
\begin{equation}
-\frac{\hbar^2}{2mr^2} \frac{\partial}{\partial r} \left( r^2
\frac{\partial \psi}{\partial r} \right) + \frac{1}{2}kr^2 \psi = E
\psi.
\label{eq:ReducedSE}
\end{equation}
The derivative $\partial \psi/\partial r = Ae^{-br^2}(-2br)$ and the
expression $\frac{\partial}{\partial r}(r^2\frac{\partial
  \psi}{\partial r}) = -6br^2Ae^{-br^2} +4b^2r^4Ae^{-br^2}$. So,
Eq.~(\ref{eq:ReducedSE}) becomes:
\begin{equation}
-\frac{\hbar^2}{2mr^2} \left[-6br^2Ae^{-br^2}+4b^2r^4Ae^{-br^2}\right]
+ \frac{1}{2}kr^2 \psi = E \psi.
\end{equation}
The terms $Ae^{-br^2}$ are present in each term, so they
cancel. Simplifying,
\begin{align}
\frac{3b\hbar^2}{m} - \frac{2b^2\hbar^2}{m}r^2 + \frac{1}{2}kr^2 =
E\\ \rightarrow r^2\left(\frac{1}{2}k-\frac{2b^2\hbar^2}{m}\right) +
\left(\frac{3b\hbar^2}{m} - E\right) = 0
\end{align}
and this is satisfied only if both the constant and $r^2$ terms
separately add to zero.  Setting the $r^2$ terms to zero gives us $b =
\frac{\sqrt{mk}}{2\hbar}$ and $E = \frac{3b\hbar^2}{m} =
\frac{3\hbar^2}{m}\frac{\sqrt{mk}}{2\hbar} =
\frac{3}{2}\hbar\sqrt{k/m}$.

So, {\bf yes}, the test function $\psi(r) = Ae^{-br^2}$ satisfies
Schr\"{o}dinger's equation for a harmonic oscillator as long as $b$
and $E$ are given by the values above.
\end{solution}
\end{example}

A particularly important case for the 3D Schr\"{o}dinger equation
(Eq.~(\ref{eq:3Dschrodinger})) is that of a hydrogen atom where the
potential energy function $U = -ke^2/r$.  The complete general
solutions for this equation, describing all possible states for an
electron in a hydrogen atom, are beyond the scope of this
course.\footnote{But not beyond the scope of a class in physical
  chemistry or upper-level quantum mechanics --- something to look
  forward to if you are a chemistry or physics major!}  We'll be
content with checking a few of the simplest cases (you'll be doing
this for homework, using an approach very similar to the example
above).  But certain general observations will be made.

The first observation is that the spatial wavefunction $\psi(x,y,z)$
for any single-particle, 3D quantum system involves three quantum
numbers.  For atoms, each of the three quantum numbers ($n, l,$ and
$m_l$) which arise in solving this particular 3D problem can be
associated with a particular spherical coordinate.  The principal
quantum number $n$ is associated with the radial coordinate $r$, and
is related to the number of nodes in the wavefunction as you move
outward from the origin on a radial line.  The orbital quantum number
$l$ is associated with the polar angle $\theta$ (measured from the
``North pole'').  When $l=0$, the associated wavefunction has no
dependence on $\theta$.  The orbital magnetic quantum number $m_l$ is
associated with the azimuthal angle $\phi$; when $m_l = 0$, there is
no $\phi$-dependence.\footnote{In other math classes you have taken
  you may have used $\phi$ for the polar angle and $\theta$ for the
  azimuthal angle.  This is just a matter of convention. In this
  course we will consistently use the definitions given in the text
  above.}

Actually, there is a fourth quantum number $m_s$ associated with the
spin angular momentum of the electron --- this is the same quantum
number that we discussed in the previous chapter.  So, a full
description of the state of an electron in an atom requires the
specification of four quantum numbers: $|n,l,m_l,m_s\rangle$.

The second observation is that the quantum numbers are directly
connected to measurable physical quantities of the atom.  For
solutions of Eq.~(\ref{eq:3Dschrodinger}), the energy of the electron
in hydrogen is determined to a high degree of accuracy by the
principal quantum number $n$.\footnote{For real atoms, there are
  slight modifications to the energy based on the other quantum
  numbers, but we won't discuss those slight corrections in this
  course.}  The energy formula for an electron in a state of quantum
number $n$ is

\begin{equation}
\label{eq:hydrogenE}
E_n = -\frac{1}{n^2} \frac{m k^2 e^4}{2 \hbar^2} 
= - \frac{13.6 \units{eV}}{n^2}  .
\end{equation}

Another useful physical quantity is the electron's orbital angular
momentum $\vec{L}$, whose magnitude is given in terms of the orbital
quantum number $l$ as

\begin{equation}
\label{eq:Lmag}
|\vec{L}| = \sqrt{l (l + 1)} \hbar  .
\end{equation}

The $z$-component of the orbital angular momentum $L_z$ is
given by
\begin{equation}
\label{eq:Lcomponent}
L_z = m_l \hbar  ,
\end{equation}
and the $z$-component of the spin angular moment $S_z$ is given by
\begin{equation}
\label{eq:Scomponent}
S_z = m_s \hbar .
\end{equation}

\section{The Hydrogen Atom}
\label{sec:hydrogen_atom}

We are now ready to talk about the actual wavefunctions of the
simplest atom: hydrogen.  Specifically, we consider the wavefunctions
for the electron that ``orbits'' around the single proton in the
nucleus.  We put the word ``orbits'' in quotes because the behavior of
the electron is quite different from the behavior of a satellite
orbiting around the Earth. Because of the small mass of an electron
and the small size of an atom, quantum wave-like effects are quite
significant.

Because of the Uncertainty Principle, we can't specify both the
position and the velocity of the electron. An electron orbiting in an
atom is typically smeared out into a probability wave around the atom,
so there really {\it isn't\/} a ``position'' per se.  And because of
the fact that the motion of the electron (whatever {\it motion\/}
really means in the quantum world) isn't in a straight line, it isn't
meaningful to specify the velocity or momentum of the
electron. Instead we solve the Schr\"{o}dinger equation
(\ref{eq:3Dschrodinger}) to find wavefunctions representing various
states of the hydrogen atom.  As usual, the square of the magnitude of
the wavefunction is to be interpreted as a probability density, in
this case, probability per unit {\bf {\em volume}}.

By solving Schr\"{o}dinger's equation (\ref{eq:3Dschrodinger}), each
hydrogen wavefunction solution is characterized by three quantum
numbers $\left(n, l, m_l\right)$, and is written as
$\psi_{nlm}(r,\theta,\phi)$.  (The quantum number $m_s$ isn't needed
to specify the wavefunction.)  For instance, the lowest energy
(ground) state wavefunction of hydrogen ($n=1, l=0, m_l=0$) is
calculated to be

\begin{equation}
\psi_{100} = \frac{1}{\sqrt{\pi a_0^3}}\ e^{-r/a_0} .
\label{eq:H100}
\end{equation}

\noindent The excited state ($n=2, l=1, m_l=0$) is given by

\begin{equation}
\psi_{210} = \frac{1}{\sqrt{32\pi a_0^3}} \left( \frac{r}{a_0} \right)
e^{-r/2a_0} \cos{\theta}
\label{eq:H210}
\end{equation}

\noindent and the excited states ($n=2, l=1, m_l=+1$) and ($n=2, l=1,
m_l=-1$) are
\begin{equation}
\psi_{211} = -\frac{1}{\sqrt{64\pi a_0^3}} \left( \frac{r}{a_0}
\right) e^{-r/2a_0} \sin{\theta}\, e^{i\phi}.
\label{eq:H211}
\end{equation}

\begin{equation}
\psi_{21-1} = \frac{1}{\sqrt{64\pi a_0^3}} \left( \frac{r}{a_0}
\right) e^{-r/2a_0} \sin{\theta}\, e^{-i\phi} ,
\label{eq:H21-1}
\end{equation}

\begin{figure}
\begin{center}
\includegraphics[width=3.0in]{semiconductors_3D/hydrogen100Dot}
\end{center}
\caption{Plot of probability density for an electron in state
  $\psi_{100}$ of hydrogen.}
\label{fig:hydrogen100Dot}
\end{figure}
\noindent respectively. We can use these wavefunctions to calculate
their probability densities $\left|\psi_{nlm_l}\right|^2$ and then
plot these probability densities to get a visualization of the
``location'' of the electron in this state.
Figure~\ref{fig:hydrogen100Dot} shows such a plot for the ground state
$\psi_{100}$.  In this plot, the probability density for finding the
electron at any location is related to the density of dots in the
plot.  Thus, the probability density is greatest at the center $r=0$
and drops off exponentially as you move away from the origin, in
agreement with the wavefunction in Eq.(\ref{eq:H100}). Also we notice
that the probability density is spherically symmetric, implying that
it doesn't depend on the direction in space, which follows from
Eq.~(\ref{eq:H100}) since there is no dependence on the coordinates
$\theta$ and $\phi$.

In the case of the excited states ($n=2$, $l=1$, $m_l=1$, 0,-1), it is
actually real linear combinations of the wavefunctions given in
equations (\ref{eq:H211}) and (\ref{eq:H21-1}) that make up the
familiar atomic $p$-orbitals that are important in chemical
bonding. For instance, the linear combination

\begin{align}
  \psi_{2p_x} &= -\frac{1}{\sqrt{2}} \left[ \psi_{211} - \psi_{21-1} \right]\\
              &= \frac{1}{\sqrt{32 \pi a_0^3}}\ \left( \frac{r}{a_0} \right) \ e^{-r/2a_0} \sin{\theta} \cos{\phi}
  \nonumber
\end{align}

\noindent corresponds to what is known as the ``$p_x$'' orbital that
you might have seen in chemistry textbooks. The probability density is
shown in Fig.~\ref{fig:hydrogen21xDot}.  In this case you will notice
that the probability density is not spherically symmetric as in the
ground state, but instead has two lobes that extend outward along the
$x$-axis.

\begin{figure}
\begin{center}
\includegraphics[width=3.0in]{semiconductors_3D/hydrogen21xDot}
\end{center}
\caption{Plot of probability density for an electron in an orbital state $\psi_{2p_x}$ of hydrogen.}
\label{fig:hydrogen21xDot}
\end{figure}

Finally, here are a few rules relating the atomic quantum numbers $n,
l, m_l$.  We won't derive them here in PHYS 212.  You'll have to take
PHYS 222 for more details about this.

\begin{itemize}
    \item The possible values for the principal quantum number $n$ are 
     \begin{equation}
     n = \mbox{1, 2, 3, \dots}
\end{equation}
\item Given a principle quantum number $n$ for an electron, the
  orbital quantum number $l$ can have the following values
    \begin{equation}
      l= 0,\, 1,\, \dots,\, n-1  .
    \label{eq:lquantumnumber}
    \end{equation}
    So, for example, if $n=1$, then $l$ can only be 0.  If $n=2$, then
    $l$ can be 0 or 1.  If $n=3$, then $l$ can be 0, 1, or 2.  Etc.
  \item Given an orbital quantum number $l$, the magnetic quantum
    number $m_l$ can have the following values
    \begin{equation}
      m_l=-l,\, -(l-1),\, \dots,\, -1,\, 0,\, 1,\, \dots,\, l-1,\, l  .
    \label{eq:mlquantumnumber}
    \end{equation}
    So, for example, if $l=0$, then $m_l$ can only be 0.  If $l=1$,
    then $m_l$ can be $-1$, 0, or 1.  If $l=2$, then $m_l$ can be
    $-2$, $-1$, 0, 1, 2,  etc.
\end{itemize}
The rule for the quantum number $m_s$ is the same as was discussed
in Chapter \ref{chapter:spin} for an electron: $m_s = -1/2$ or $+1/2$, 
regardless of the other three quantum numbers $n, l$ and $m_l$.

Some of the consequences of these rules will be explored in the
problems.

\section{Periodic Table}
\label{sec:periodic_table}

We now have the basic tools to explain the periodic table of the 
elements. Electrons are fermions since they have a half-integral
spin quantum number ($s = 1/2$ for an electron). Consequently,
electrons obey the Pauli Exclusion Principle (see Section
\ref{sec:pauli_exclusion_principle}): no two electrons can be in the same
state. So, if there is more than one electron in an atom, no
two of them can have the same combination of quantum numbers
$n, l, m_l$ and $m_s$.

Consider lithium, which has three protons in the nucleus
and three electrons orbiting it.  The first two electrons can go into
the ground state, but Pauli demands that the third electron go into a
new state.  If the lithium atom is in its lowest possible energy
state, that third electron must be in some $n=2$ orbital.  Most of the
chemical behavior of atoms is due to the shape and size of their outer
electron orbitals, so the Pauli exclusion principle is what makes
lithium act very differently than helium.

This process repeats many times over to make up the periodic table.
The more electrons there are, the higher the energy levels must be filled,
similar to the five particle system in Chapter~\ref{chapter:quantum_statistics}, Example~\ref{example:five_particles}.
For some specific orbital, meaning some specific values of $n$, $\ell$
and $m_\ell$, there are two possible single-electron states:
$|n\,\ell\,m_\ell\uparrow\rangle$ and
$|n\,\ell\,m_\ell\downarrow\rangle$, and we can put two electrons into
the antisymmetric combination of these two states.  Any additional
electrons will have to go into a new orbital.

Were it not for the Pauli Exclusion Principle (i.e., if it didn't apply
to electrons in atoms), then all elements would have all of their
electrons in the lowest energy state with $n = 1$ (except when
excited). Were that the case, then there would be no chemical diversity --
each element would behave almost identically to each other, and
there would be no chemical interactions. Chemistry would be really dull
in this case. Even worse, we wouldn't be around to notice that chemistry
had become dull.

\section{Conductors, Semiconductors, and Insulators}
\label{sec:conductors_semiconsuctors_insulators}

For the second half of this chapter, we return to our discussion from
Chapter~\ref{chapter:quantized_energies} of quantized energy levels
and band structure. It turns out that some clever manipulations of
these materials can result in some really important electronic
components. The material in this chapter is the basis behind some of
the most important developments in modern electronics.

\begin{figure}
\begin{center}
\includegraphics[width=5.0in]{semiconductors_3D/FermiLevel}
\end{center}
\caption{Highest valence and lowest conduction bands (with Fermi Level) for:
(a) an insulator; (b) a semiconductor; and (c) a conductor. The dark shading
denotes filled energy levels.}
\label{fig:FermiLevel}
\end{figure}

We return to the discussion of energy bands from 
Chapter~\ref{chapter:quantized_energies}.
First, we need to discuss the electrical conductivity of a material
and how it relates to the band structure. Figure~\ref{fig:FermiLevel}
shows the highest energy, filled valence band and the lowest energy,
unfilled conduction band for a typical material. We define an energy 
level referred to as
the ``Fermi energy'' $E_F$ that denotes the boundary between the
highest energy filled level and the lowest unfilled level for a
material at a temperature of absolute zero ($T = 0$ K). By convention,
for a situation like that in Fig.~\ref{fig:FermiLevel}, $E_F$ is in
the middle of the forbidden energy gap.

The electrical properties of a material depend critically on the band gap
energy $E_g$. For the case in Fig.~\ref{fig:FermiLevel}a, the band gap
energy is so large that electrons in the filled valence band rarely jump
up to the conduction band. Since the valence band is completely full,
there is no possibility for electrical conduction if an electric field
is applied to this material, because there is no way for any electron in
the valence band to change its energy since all the other energy levels
in the valence band are filled. (Recall: Pauli's Exclusion Principle
applies here, since electrons are fermions.) And there are no electrons
in the conduction band either. So, Fig.~\ref{fig:FermiLevel}a corresponds
to an electrical insulator.

An electrical conductor is shown in Fig.~\ref{fig:FermiLevel}c. In this
case, the band gap energy $E_g = 0$, so if an electrical field is applied, electrons
in the valence band can easily be excited into one of the available energy
levels in the conduction band.

Figure \ref{fig:FermiLevel}b corresponds to a ``semiconductor.'' The
energy band structure is similar to the insulator of Fig.~\ref{fig:FermiLevel}a,
except that the band gap energy $E_g$ is smaller. This is important because
the smaller but non-zero $E_g$ makes it possible to ``turn on'' and ``turn off''
electrical conductivity, a {\bf very} useful property.

We discuss two ways in which a semiconductor can be manipulated to
produce good conductivity.

\subsection{Temperature Dependence of Conductivity for a Semiconductor}

If the temperature of the material is not zero, then there is
thermal energy that can cause electrons in the valence band to 
become excited into the conduction band. The larger the temperature,
the larger the interatomic collision energies that can excite electrons.
The temperature dependence is captured well by the {\it Fermi-Dirac}
probability distribution function:
\begin{equation}
p(E) = \frac{1}{1+e^{(E-E_F)/k_BT}} .
\label{eq:Fermi-Dirac}
\end{equation}
This relation is plotted in Fig.~\ref{fig:Fermi-Dirac}a for three
different temperatures.  For a material at absolute zero ($T = 0$), all
of the energy levels for $E < E_F$ are filled (with probability $p(E) = 1$),
whereas all energy levels for $E > E_F$ are empty ($p(E) = 0$).
If the temperature is not zero, though, thermal excitations result in
a decrease in $p$ for $E < E_F$ and an increase for $E > E_F$.

\begin{figure}
\begin{center}
\includegraphics[width=5.0in]{semiconductors_3D/Fermi-Dirac}
\end{center} 
\caption{(a) Fermi-Dirac relation (Eq.~\ref{eq:Fermi-Dirac}) for
three temperatures. (b) Energy band diagram for a temperature large
enough to excite several electrons from the valence to conduction bands.} 
\label{fig:Fermi-Dirac} 
\end{figure}

If the temperature is large enough, then the probability $p(E)$ for electrons
to be found for energies in the conduction band 
may become non-negligible, and there may be sufficient electrons
excited into the conduction band for the material to become a good conductor.
An important question:  {\bf how large is ``large enough?''} The answer
to this question can be seen by noticing the factor of $k_BT$
in Eq.~(\ref{eq:Fermi-Dirac}). This factor should look familiar to you
from our thermodynamics unit from PHYS 211 --- this is the well-known
{\it Boltzmann factor} that gives a characteristic energy for thermal
collisions.  If the temperatue in a semiconductor is such that $k_BT$ is
comparable to or greater than the band gap energy $E_g$, then there will
be sufficient thermal energy for the semiconductor to be a 
good conductor. This is shown conceptually in Fig.~\ref{fig:Fermi-Dirac}b: for 
a sufficiently large temperature, some of the higher energy levels in the
valence band become empty and some of the lower energy bands in the conduction
band contain electrons. This enables conduction not only for electrons
in the conduction band but also for electrons in the valence band who now
have open energy levels if an electric field is applied.

\begin{example}{Probabilities from Fermi-Dirac.}
\label{ex:FD}
Lead sulfide (PbS) has a band gap energy of $0.37\units{eV}$. (a) Use an
approximation where $k_BT \approx E_g$ to determine a temperature for which you 
would expect PbS to be an excellent conductor.
(b) Calculate the
probability of finding an electron with energy at the bottom of the
conduction band for temperatures of 400K and 4000K.

%{\bf Solution}: 
\begin{solution}
(a) As a reasonable approximation, if the temperature is
such that $k_BT \approx E_g$, then there should be sufficient energy
from thermal collisions to excite electrons from the valence to the conduction
band.  So, 
$T \approx E_g/k_B 
= (0.37\units{eV})/(8.62\times 10^{-5}\units{eV$\cdot$K$^{-1}$})  
= 4290\units{K}$.

(b) We can calculate probabilities from the Fermi-Dirac equation
(Eq.~(\ref{eq:Fermi-Dirac})). Since the Fermi energy $E_F$ is in the center
of the band gap, the lowest conduction band energy $E = E_F + E_g/2$,
so for $T = 400\units{K}$,
\begin{align}
p(E) &= \frac{1}{1+e^{(E-E_F)/k_BT}} = \frac{1}{1+e^{E_g/2k_BT}}\nonumber \\
&= \frac{1}{1+e^{(0.37\units{eV$\cdot$K$^{-1}$})/(2 \times 8.62\times 10^{-5}\units{eV$\cdot$K$^{-1}$}\cdot 400\units{K})}}\nonumber \\ 
&= 0.0047 
\end{align}
This temperature is about 10 times lower than the value that we calculated in
part (a), so it makes sense that the probability is small for finding an 
electron at the bottom level of the conduction band.

If you repeat the calculation for $T = 4000\units{K}$, you'll find 
a probability of 0.37, which is quite appreciable. This is consistent 
with the result in part (a) which predicts a temperature of $4290\units{K}$ 
for appreciable conduction.

\end{solution}
\end{example}

In the previous example, it is tempting to say that if $p(E) = 0.0047$ for
an electron to be found in the lowest conduction-band energy state, then
the material will be an insulator. But remember how many electrons are
in a typical solid (a {\bf lot}!), so even if the probabilities are low,
there will still be some electrons that can conduct electricity. But 
with higher and higher temperatures there
would be more and more conduction band electrons.

So, semiconductors conduct electricity better at higher temperatures.
That can be both a good thing and a bad thing. On the good side, this 
principle can be used to make very sensitive and precise temperature 
measurement devices called {\it thermistors}.
On the bad side, the electronic devices that we'll discuss in the next
section can fail if the semiconductor gets too hot. This is why it
is so critical for electronic devices to be cooled.

\subsection{Doping and n- and p-Type Semiconductors}

\begin{figure}
\begin{center}
\includegraphics[width=3.5in]{semiconductors_3D/Doping}
\end{center} 
\caption{(a) n-type semiconductor; donor atom impurities are added to provide 
some extra electrons which occupy some of the levels in the conduction band. 
(b) p-type semiconductor; acceptor atom impurities are added that remove
electrons from the valence band, leaving conducting ``holes'' behind. } 
\label{fig:Doping} 
\end{figure}



There is another way that you can make a semiconductor conduct: doping, 
i.e., add impurities to a semiconductor with a different number of valence
electrons than the primary semiconductor material. For example, silicon
is a common ``Group IV'' semiconductor element with 4 valence electrons; i.e., 
4 electrons in the highest energy (unfilled) shell. If a silicon semiconductor
is doped with trace amounts of  an impurity ``Group V'' element such as arsenic
with 5 valence electrons, 4 of the valence electrons in the impurity
will fill the valence band along with the silicon atoms, but there will be
one additional outer-shell electron for each impurity atom. Arsenic is
considered to be a ``donor impurity'' for silicon semiconductors since
it donates an additional electron. The additional donated electrons
are mobile charge carriers in the conduction band, as shown in 
Fig.~\ref{fig:Doping}a.\footnote{This is a simplification: the impurities
alter the band structure, but the basic principle is the same.}
Semiconductors doped with donor atoms are called ``n-type'' semiconductors,
because they have mobile charge carriers that have negative charge 
(electrons).

Doping a semiconductor with ``acceptor impurities'' results in
``p-type'' semiconductors, as shown in Fig.~\ref{fig:Doping}b.
An example of a p-type semiconductor would be silicon (Group IV)
doped with impurities from Group III (with 3 valence electrons), such as
boron. Since the impurity atoms have only 3 valence electrons (unlike
the 4 for the bulk silicon atoms), each impurity atom would fill only
3 out of the 4 available energy levels, leaving an unfilled ``hole''
in the valence band. This hole provides other electrons in the valence 
band with an opportunity to change their energies in response to
an applied electric field. Consequently, the presence of holes in
the valence band allows for conductivity. An applied electric field
results in electrons in the valence band moving opposite the
field into available holes; the result is that the holes move
in the direction of the electric field. Conceptually, the holes themselves
act like positive, mobile charge carriers; hence the term {\it p-type}
to describe these semiconductors.

Things become very interesting if a p-type semiconductor is joined
with an n-type semiconductor. Figure \ref{fig:pnJunction}a shows
an n-type semiconductor (on the left) with mobile electrons (dark
circles) connected to a p-type semiconductor (on the right) with
mobile holes (white circles). But in the absence of an applied
electric field, the mobile electrons and holes diffuse around the 
material, moving randomly in the sample, including moving across 
the p-n junction. Invariably, mobile electrons and holes near the
junction come into contact with each other and {\it annihilate},
with the mobile electron dropping into the open energy level left
by the hole. This annihilation process removes both the electron and
the hole from the conduction process. The result is that in
the vicinity of the junction between the n- and p-type materials, there
is a {\it depletion zone} (DZ) with no mobile charge 
carriers, as shown in Fig.~\ref{fig:pnJunction}b. The annihilation process at the junction produces
a {\it bias} electric field that eventually stops the diffusion of the
electrons and holes across the junction, stabilizing the width
of the DZ.

\begin{figure}
\begin{center}
\includegraphics[width=3.0in]{semiconductors_3D/pnJunction}
\end{center} 
\caption{(a) A pn junction (in the absence of diffusion or a depletion
zone) made by connecting an n-type semiconductor on the left (with
negative mobile charges --- black dots) and a p-type semiconductor 
on the right (with positive mobile charges --- white dots). 
(b) A more realistic pn junction; n-type electrons and p-type holes diffuse
across the junction and annihilate each other, resulting in a depletion
zone (DZ) around the junction with no mobile charge carriers.
} 
\label{fig:pnJunction} 
\end{figure}

\section{Diodes, Photodiodes, and LEDs}
\label{sec:diodes}

A p-n junction like that shown in Fig.~\ref{fig:pnJunction} forms what
is know in electronics as a {\it diode} --- a device that enables
current to pass only in one direction.  Diodes are used for
{\it rectifiers} which turn AC (alternating current) power from the
outlets in your house into DC (direct current) power needed for most
devices.  Diodes are used to protect sensitive electronic devices from
accidental reverse voltages.  And diodes form some of the fundamental
elements in digital computer circuitry.\footnote{For a full appreciation of
the importance of diodes --- and other electronic elements --- consider
PHYS 235 (``Applied Electronics'').}

To understand how a p-n junction works as a diode, consider 
Figs.~\ref{fig:pnJunction} and \ref{fig:BiasedpnJunction}.  The
key is the depletion zone (DZ) at the junction. Since there are no 
mobile charge carriers in the DZ, then that part of the device
is a good electrical insulator. For a p-n junction to conduct, the DZ has to
be eliminated.

If you want to run an electrical 
current through a p-n junction, you need to apply an electric field.
If the field points from the p-type to the n-type semiconductor
(as in Fig.~\ref{fig:BiasedpnJunction}a), the holes in the
p-type material are pulled in the direction of the electric field
toward the junction, and the electrons in the n-type material are
pulled in the direction opposite the electric field, also toward
the junction.  Consequently, the DZ shrinks.
If the field is strong enough\footnote{It turns out that
for silicon semiconductors, there is a ``turn-on voltage'' of
0.6 V that needs to be exceeded for a diode to conduct in the
forward-biased direction.} the holes and electrons meet at the junction,
and the DZ is completely gone; consequently, the entire device
conducts and the current can pass. So, a strong enough electric field
in this {\it forward-biased} direction will eliminate the DZ and
produce an electrical current.

\begin{figure}
\begin{center}
\includegraphics[width=3.0in]{semiconductors_3D/BiasedpnJunction}
\end{center} 
\caption{(a) Forward-biased pn junction with an electric field
pointing from the p-type toward the n-type semiconductor. This electric
field pulls n-type charges (electrons) to the right and p-type charges
(holes) to the left, decreasing the size of the depletion zone.
(b) Reverse-biased pn junction with an enlarged depletion zone.
} 
\label{fig:BiasedpnJunction} 
\end{figure}

On the other hand, if an electric field is applied pointing from the
n-type to the p-type semiconductor, the field pushes the holes and
mobile electrons {\it away} from the junction, making the DZ even
larger. With the DZ intact, the p-n junction won't conduct electricity
in this {\it reverse-biased} direction. So, a p-n junction works as
a diode --- it allows current to flow in one direction but
not the other.\footnote{It isn't an ideal diode because of the turn-on voltage,
but there are ways of getting around this limitation.}

But more can be done with a diode (p-n junction) than just passing
current only in one direction. When a forward-biased diode is conducting
a current, mobile electrons and holes are continually annihilating each
other at the junction. Each time an electron-hole pair annihilates,
a photon is emitted, as shown in Fig.~\ref{fig:LED_PhotoExcitation}a.
This is the principle behind what are called {\it light-emitting
diodes} (LEDs), which can be found in almost any piece of modern
electronics. The little red and blue indicator lights on your cell phone
are LEDs. The ``flash'' that your cell phone uses to take pictures at
night (which you probably use more as a flashlight when walking back
to your dorm at night) is an LED. Most scoreboards at games are made with
LEDs.  And you can now buy LED light bulbs which are {\it significantly}
more efficient and reliable than standard, incandescent light bulbs.

\begin{figure}
\begin{center}
\includegraphics[width=3.0in]{semiconductors_3D/LED_PhotoExcitation}
\end{center} 
\caption{(a) Photon emitted when an n-type charge carrier (an electron)
annihilates a p-type charge carrier (a hole); conceptually, the electron
drops into the open, lower-energy state represented by the hole.
(b) Excitation of a valence electron by an incoming photon. 
The absorption of the photon results in the formation of an  n- 
and p- charge carrier pair. 
} 
\label{fig:LED_PhotoExcitation} 
\end{figure}

The process works in reverse. If a photon with suitable wavelength and
energy hits a p-n junction (Fig.~\ref{fig:LED_PhotoExcitation}b), it 
can excite an electron from the
valence to the conduction band, producing an electron-hole
pair in the DZ. Some residual electric fields in the DZ\footnote{We won't
explain the origin of this residual electric field here.} sweep this
electron and hole away from the DZ and produce a small electrical current
that can be measured. As a result, a diode can be used as a light
detector --- it is referred to as a {\it photodiode} when used this
way. This is the basic idea behind the detectors used in modern
digital cameras, including the camera in your cell phone.

Finally, an introduction to semiconductor physics would not be complete
with a discussion of the {\it transistor}, the development of
which completely revolutionized modern technology. A basic transistor
can be made out of a sandwich of a p-type semiconductor between two pieces
of n-type semiconductors (Fig.~\ref{fig:transistor}). For your homework
problems, you will explain (by considering the depletion zones at
each of the two p-n junctions) how a transistor works. The result is
a device that will not pass an electrical current in {\it either} 
direction unless a very small electrical current is provided at
one of the p-n junctions. That very small, second current effectively
``turns on'' or ``turns off'' the larger current flowing lengthwise
through the device. As you will also see in your homework, the
current can also be turned on by light shining on the p-n junction,
as shown in Fig.~\ref{fig:transistor}b for a device referred to
as a phototransistor.

\begin{figure}
\begin{center}
\includegraphics[width=3.0in]{semiconductors_3D/transistor}
\end{center} 
\caption{(a) An npn transistor. (b) Phototransistor.
} 
\label{fig:transistor} 
\end{figure}

The use of a transistor as an electronic ``on-off switch'' has
tremendous implications for modern technolgoy. Without electronic 
switches, most of modern electronic technology would be useless.
Simple examples of the use of transistor switches abound. 
A transistor switch turns on the blue indicator light on your
phone when you have a new email or the red indicator light if
your phone is charging. A transistor switch turns on and off
the display of your phone or the LED light that you use to
walk home at night. In a computer, billions of transistor switches
are rapidly turning on and off small currents that enable
the computer to perform its tasks.

Transistors are also very important for power amplification. As an example,
your cell phone detects electromagnetic waves transmitted by a nearby
cell tower, but the detected signal is very weak. A transistor amplifier
makes the signal stronger so that your cell phone can decode the signal
and produce a time-varying voltage which is then amplified by another
transistor circuit to produce a sound loud enough for your ear to hear.

The npn transistor was the first truly electronic switch that didn't depend
on vacuum-tube technology.\footnote{Vacuum tubes were common in all
radios and TVs up until the 1950s, but you will be hard-pressed to find
them now.} There are many advantages to transistors over vacuum tubes, the
most significant being that they can be miniaturized and printed into
integrated circuits. Current computer CPUs have over a {\bf billion}
transistors.  Think about that: that is a {\bf lot} of transistors.

\newpage

\section*{Problems}
\label{sec:3D_and_semiconductors_problems}
\markright{PROBLEMS}

%\vspace{1.5in}


%three  (old 5-8)

\begin{problem}
  The so-named Paschen series of radiation emitted by hydrogen
  corrersponds to transitions from states of principle quantum numbers
  $n = 4$, 5, 6, \dots, to the state $n = 3$.
\begin{enumerate}
\item Calculate the frequencies and wavelengths of the three lowest
  energy photons emitted from hydrogen in the Paschen series.
\item In what region of the electromagnetic spectrum are these
  photons?
\end{enumerate}
\end{problem}

% four (old 5-11)

\begin{problem}
  An electron in a hydrogen atom is in a state of principle quantum
  number $n = 4$.  Using the notation ($n, l, m_l, m_s$) to specify a
  state, write all of the possible states for this electron. [ HINT:
  There should be a total of $32$ states.]
\end{problem}

% five (old 5-12)

\begin{problem}
  Find the maximum possible magnitude for the orbital angular momentum
  $|\vec{L}|$ of an electron in the state of principle quantum number
  $n = 7$ of a hydrogen atom.
\end{problem}

% six (Old 5-13)

\begin{problem}
  The orbital angular momentum of the electron in a hydrogen atom has
  a magnitude $|\vec{L}| = 2.585 \times 10^{-34}
  \units{J$\cdot$s}$. What is the minimum possible energy for this
  electron in the hydrogen atom?
\end{problem}

% seven (Old 5-14)

\begin{problem}
  Determine the principle quantum number $n$ and orbital quantum
  number $l$ for a hydrogen atom whose electron has energy $-0.850
  \units{eV}$ and orbital angular momentum $|\vec{L}| = \sqrt{12}
  \hbar$.
\end{problem}

% eight

\begin{problem}
Show that the probability that a state with energy $\Delta E$ above the
Fermi energy will be occupied is equivalent to the probability that a state
with energy $\Delta E$ {\it below} the Fermi energy will be
{\it unoccupied}. Explain in 1 or 2 sentences why this result makes
sense intuitively, when considering how conduction works in a semiconductor.
\end{problem}

\newpage

% nine

\begin{problem}
A one-dimensional periodic potential composed of 10 side-by-side
finite square well potentials has a valence band with 10 energy
levels from a low of $0.81\units{ eV}$ to a high of $1.10\units{eV}$ 
and a conduction band with 10 levels ranging from $3.19\units{eV}$ to 
$4.39\units{eV}$ .
The Fermi energy (in the middle of the band
gap) is $2.145\units{eV}$ for this system. Assume that there are 20 electrons
in this system (this is analogous to 10 atoms with 2 valence electrons
each).
\begin{enumerate}
\item Calculate the probability that an electron will be found in the 
energy state $E_{11} = 3.19\units{eV}$ (i.e., in the lowest conduction
band state) for $T = 300\units{K} $(room temperature), $3,000\units{K}$, and
$30,000\units{K}$.
\item Calculate the values of $k_BT$ for each of the temperatures
from part (a). Do the probabilities that you calculated in (a) make
sense, given these values for $k_BT$?
\item Now, download the Excel sheet ``FDForTenWells.xlsx'' from
the calendar page.  This worksheet calculates Fermi-Dirac probabilities
for electrons to occupy any of the 10 conduction band energies.
The sheet also calculates an expected average number of electrons
in each of these levels (by multiplying the probability by 2, since
there are two energy levels per state) and adds up the total number of
electrons expected in the conduction band. Use this worksheet to
calculate the estimated number of conduction band electrons for
$T = 300$, 3000, and $30000\units{K}$. For each of these temperatures,
would you expect this system to be an electrical insulator or
electrical conductor?
\item if there were only 10 electrons (e.g., 10 atoms with only
1 valence electron instead of 2), how would that affect the
conduction properties of this system?
\end{enumerate}
\end{problem}

% ten

\begin{problem}
Lead sulfide (PbS) has a band gap energy of $0.37\units{eV}$.
\begin{enumerate}
\item Calculate the value of $k_BT$ for temperatures $30\units{K}$,
$300\units{K}$ (room temperature), $3,000\units{K}$ and $30,000\units{K}$. 
At these temperatures would you expect PbS to be a poor, moderate or 
good electrical conductor?
\item Use the Fermi-Dirac relation to calculate the probability of the
lowest conduction band energy state being occupied for temperatures
30, 300, 3,000 and $30,000\units{K}$.  Are these results consistent with your
answers from (a)? (Note: it's not necessary for {\it every} conduction
band level to be filled for a material to be a reasonable conductor,
but the probabilities should at least be on the order of $10^{-5}$
or better.)
\end{enumerate}
\end{problem}

% eleven

\begin{problem}
Some recent studies have investigated the possibility of using DNA
strands as fundamental building blocks of nanotechnology. One
possibility is the use of DNA for nanoelectronic devices.
Experimental studies of a particular DNA sequence
(Poly(dA)*Poly(dT) with a B-type structure\footnote{``DNA
Electronics,'' M. Taniguchi and T. Kawai, Physica E {\bf 33},
1-12 (2006).}) measured a band gap of $2.7\units{eV}$ between
valence and conduction bands. Would you expect
this DNA strand to be a good electrical conductor at room 
temperature? Explain. (A ``$k_BT$'' analysis is sufficient ---
you don't need to use Fermi-Dirac distributions here.) If it is
a good conductor, approximate how low a temperature would be
needed to make it a poor conductor. If it is a poor conductor,
approximate how large a temperature would be needed to make it
a good conductor.
\end{problem}

% twelve

\begin{problem}
A silicon photodiode has a band gap energy of $1.17\units{eV}$. 
\begin{enumerate}
\item Calculate
the maximum wavelength of electromagnetic radiation that this photodiode
can detect. 
\item Why can't a silicon photodiode detect EM radiation
with wavelengths larger than the value calculated in part (a)?
\item Give an argument as to how a silicon photodiode {\it can}
detect EM radiation with wavelengths {\it smaller} than the value
calculated in part~(a).
\end{enumerate}
\end{problem}

% thirteen

\begin{problem}
Calculate the wavelength (and state the color --- or ``IR'' or
``UV'' if infrared or ultraviolet) of electromagnetic radiation 
emitted from a light-emitting diode (LED) made from the
following materials: 
\begin{enumerate}
\item Gallium arsenide (band gap $1.43\units{eV}$);
\item Gallium (III) Phosphide (band gap $2.26 \units{eV}$); 
\item Gallium nitride (band gap $3.4\units{eV}$).
\end{enumerate}
\end{problem}

% fourteen

\begin{problem}
White light LED bulbs can be made by using blue or violet LEDs
that are coated with phosphor (fluorescent) coatings. (a) Explain
in a few sentences how this kind of approach can produce light that
appears white. (b) Why would it {\it not} be possible to make a
white light LED with a red LED as the source? (Note: white-light
LED bulbs can also be made with coatings of quantum dots or can be 
made with combinations of red, green and blue LEDs.)
\end{problem}

\newpage

% fifteen

\begin{problem}
\begin{enumerate}
\item Draw a sketch of a p-n junction in the absence of any imposed
electric fields. In the sketch, show the depletion zone (DZ) and
sketch mobile carriers within the p- and n-type semiconductors
away from the DZ. Make sure that you have the correct type of
charge carriers (+ or -) in the correct material.
\item Now, make a similar sketch, but for the case with an applied
electric field pointing from the n- to the p-type semiconductor.
Is the DZ larger or smaller in this case than for the case with 
$E = 0$ from part (a) Explain why. Do you expect this p-n junction
to conduct electricity readily in the direction of this electric
field? Explain.
\item Repeat part (b), but for an electric field pointing from the p-
to the n-type semiconductor, answering the same questions.
\item Does this device act like a diode; i.e., a device that passes
current readily in one direction but not in the other? In what ways does
a p-n junction deviate from an ``ideal'' diode?
\end{enumerate}
\end{problem}

% sixteen

\begin{problem}
{\bf Transistors.} Figure \ref{fig:transistor}(a) is a sketch of a sandwich made of
2 n-type semiconductors with a thin p-type layer in between.
\begin{enumerate}
\item A wire is connected to each of the ends of this npn sandwich.
Give a short explanation as to why this npn device would not be
expected to conduct electricity in either direction. (Hint: consider what
happens to the DZ at each junction if you apply an electric field in either 
direction.)
\item A third wire is connected to the right pn junction. If
a very small current is fed into the device through this third wire,
there can be a significant conduction of electricity between the
two end wires if the current is going to the left. Explain why. 
(Hint: consider the depletion zone and
what happens if you inject some electrons or holes into it.)
\item Instead of a third wire, you can make a phototransistor that turns
on and off the current between the ends by shining light on the
p-type layer. Explain why light can turn on the current. (Hint: this
will only work if the light has a wavelength smaller than a particular
critical value.)
\end{enumerate}
\end{problem}

% seventeen

\newpage

%one

\begin{problem}
  The wavefunction for the ground state of the hydrogen atom 
\hbox{($n=1, l=0, m_l=0$)} is 
\[ \psi_{100} = \frac{1}{\sqrt{\pi a_0^3}}\,  e^{-r/a_0}, \] 
where $a_0$ is a constant which can be written in terms of 
fundamental constants
as $a_0 = \frac{\hbar^2}{m_e k e^2}$.
\begin{enumerate}
\item Show that this wavefunction is a solution to the 3D
  Schr\"{o}dinger equation %(\ref{eq:3Dschrodinger})
  \[ -\frac{\hbar^2}{2mr^2} \left[ \frac{\partial}{\partial r} \left(
      r^2 \frac{\partial \psi}{\partial r} \right) +
    \frac{1}{\sin{\theta}} \frac{\partial}{\partial \theta} \left(
      \sin{\theta} \frac{\partial \psi}{\partial \theta} \right) +
    \frac{1}{\sin^2{\theta}} \frac{\partial^2 \psi}{\partial \phi^2}
  \right] - \frac{k e^2}{r} \psi = E \psi \]
(Hint: since there are no $\theta$ or $\phi$ terms in the wavefunction,
the partial derivatives $\partial \psi/\partial \theta$ and 
$\partial \psi/\partial \phi$
are both zero. That will simplify things a {\bf lot}.)

\item Determine the value of $E$ required for $\psi_{100}$ to be a
  solution.  Compare your result with Eq.~(\ref{eq:hydrogenE}).
\end{enumerate}
\end{problem}

% two

\begin{problem}
  The ground state wavefunction of the electron in the hydrogen atom
  is spherically symmetric which means that the wavefunction $\psi(r)$
  can be written solely in terms of the radial coordinate $r$
  representing the distance between the proton and electron.

\begin{enumerate}
\item What does the quantity $|\psi(r)|^2$ mean physically?

\item Show that the volume of a thin spherical shell of radius $r$
  and thickness $dr$ is $4\pi r^2\, dr$. (You can use the approximation
for small $dr$ that the volume is the surface area of the sphere
times $dr$.)

\item In spherical coordinates, the ground state solution of the
  Schr\"{o}dinger equation for the hydrogen atom is
\[ \psi_{100} = \frac{1}{\sqrt{\pi a_0^3}}\,  e^{-r/a_0}, \] 
where $a_0$ is the same constant as from the previous problem.  Use the
  result of part~(b) to write an expression for the probability that
  the electron is in a spherical shell of radius $r$ and thickness
  $dr$.

\item Calculate the radius of the shell (of constant thickness $dr$)
  where the electron is most likely to be found.
\end{enumerate}
\label{prob:hydrogen_wavefn}
\end{problem}


